\href{http://travis-ci.org/nodeca/argparse}{\tt }

C\+LI arguments parser for node.\+js. Javascript port of python\textquotesingle{}s \href{http://docs.python.org/dev/library/argparse.html}{\tt argparse} module (original version 3.\+2). That\textquotesingle{}s a full port, except some very rare options, recorded in issue tracker.

{\bfseries NB.} Method names changed to camel\+Case. See \href{http://nodeca.github.com/argparse/}{\tt generated docs}.

\section*{Example }

test.\+js file\+:


\begin{DoxyCode}
#!/usr/bin/env node
'use strict';

var ArgumentParser = require('../lib/argparse').ArgumentParser;
var parser = new ArgumentParser(\{
  version: '0.0.1',
  addHelp:true,
  description: 'Argparse example'
\});
parser.addArgument(
  [ '-f', '--foo' ],
  \{
    help: 'foo bar'
  \}
);
parser.addArgument(
  [ '-b', '--bar' ],
  \{
    help: 'bar foo'
  \}
);
var args = parser.parseArgs();
console.dir(args);
\end{DoxyCode}


Display help\+:


\begin{DoxyCode}
$ ./test.js -h
usage: example.js [-h] [-v] [-f FOO] [-b BAR]

Argparse example

Optional arguments:
  -h, --help         Show this help message and exit.
  -v, --version      Show program's version number and exit.
  -f FOO, --foo FOO  foo bar
  -b BAR, --bar BAR  bar foo
\end{DoxyCode}


Parse arguments\+:


\begin{DoxyCode}
$ ./test.js -f=3 --bar=4
\{ foo: '3', bar: '4' \}
\end{DoxyCode}


More \href{https://github.com/nodeca/argparse/tree/master/examples}{\tt examples}.

\section*{Argument\+Parser objects }


\begin{DoxyCode}
new ArgumentParser(\{paramters hash\});
\end{DoxyCode}


Creates a new Argument\+Parser object.

{\bfseries Supported params\+:}


\begin{DoxyItemize}
\item {\ttfamily description} -\/ Text to display before the argument help.
\item {\ttfamily epilog} -\/ Text to display after the argument help.
\item {\ttfamily add\+Help} -\/ Add a -\/h/–help option to the parser. (default\+: true)
\item {\ttfamily argument\+Default} -\/ Set the global default value for arguments. (default\+: null)
\item {\ttfamily parents} -\/ A list of Argument\+Parser objects whose arguments should also be included.
\item {\ttfamily prefix\+Chars} -\/ The set of characters that prefix optional arguments. (default\+: ‘-\/‘)
\item {\ttfamily formatter\+Class} -\/ A class for customizing the help output.
\item {\ttfamily prog} -\/ The name of the program (default\+: {\ttfamily path.\+basename(process.\+argv\mbox{[}1\mbox{]})})
\item {\ttfamily usage} -\/ The string describing the program usage (default\+: generated)
\item {\ttfamily conflict\+Handler} -\/ Usually unnecessary, defines strategy for resolving conflicting optionals.
\end{DoxyItemize}

{\bfseries Not supportied yet}


\begin{DoxyItemize}
\item {\ttfamily fromfile\+Prefix\+Chars} -\/ The set of characters that prefix files from which additional arguments should be read.
\end{DoxyItemize}

Details in \href{http://docs.python.org/dev/library/argparse.html#argumentparser-objects}{\tt original Argument\+Parser guide}

\section*{add\+Argument() method }


\begin{DoxyCode}
ArgumentParser.addArgument([names or flags], \{options\})
\end{DoxyCode}


Defines how a single command-\/line argument should be parsed.


\begin{DoxyItemize}
\item {\ttfamily name or flags} -\/ Either a name or a list of option strings, e.\+g. foo or -\/f, --foo.
\end{DoxyItemize}

Options\+:


\begin{DoxyItemize}
\item {\ttfamily action} -\/ The basic type of action to be taken when this argument is encountered at the command line.
\item {\ttfamily nargs}-\/ The number of command-\/line arguments that should be consumed.
\item {\ttfamily constant} -\/ A constant value required by some action and nargs selections.
\item {\ttfamily default\+Value} -\/ The value produced if the argument is absent from the command line.
\item {\ttfamily type} -\/ The type to which the command-\/line argument should be converted.
\item {\ttfamily choices} -\/ A container of the allowable values for the argument.
\item {\ttfamily required} -\/ Whether or not the command-\/line option may be omitted (optionals only).
\item {\ttfamily help} -\/ A brief description of what the argument does.
\item {\ttfamily metavar} -\/ A name for the argument in usage messages.
\item {\ttfamily dest} -\/ The name of the attribute to be added to the object returned by parse\+Args().
\end{DoxyItemize}

Details in \href{http://docs.python.org/dev/library/argparse.html#the-add-argument-method}{\tt original add\+\_\+argument guide}

\section*{Action (some details) }

Argument\+Parser objects associate command-\/line arguments with actions. These actions can do just about anything with the command-\/line arguments associated with them, though most actions simply add an attribute to the object returned by parse\+Args(). The action keyword argument specifies how the command-\/line arguments should be handled. The supported actions are\+:


\begin{DoxyItemize}
\item {\ttfamily store} -\/ Just stores the argument’s value. This is the default action.
\item {\ttfamily store\+Const} -\/ Stores value, specified by the const keyword argument. (Note that the const keyword argument defaults to the rather unhelpful None.) The \textquotesingle{}store\+Const\textquotesingle{} action is most commonly used with optional arguments, that specify some sort of flag.
\item {\ttfamily store\+True} and {\ttfamily store\+False} -\/ Stores values True and False respectively. These are special cases of \textquotesingle{}store\+Const\textquotesingle{}.
\item {\ttfamily append} -\/ Stores a list, and appends each argument value to the list. This is useful to allow an option to be specified multiple times.
\item {\ttfamily append\+Const} -\/ Stores a list, and appends value, specified by the const keyword argument to the list. (Note, that the const keyword argument defaults is None.) The \textquotesingle{}append\+Const\textquotesingle{} action is typically used when multiple arguments need to store constants to the same list.
\item {\ttfamily count} -\/ Counts the number of times a keyword argument occurs. For example, used for increasing verbosity levels.
\item {\ttfamily help} -\/ Prints a complete help message for all the options in the current parser and then exits. By default a help action is automatically added to the parser. See Argument\+Parser for details of how the output is created.
\item {\ttfamily version} -\/ Prints version information and exit. Expects a {\ttfamily version=} keyword argument in the add\+Argument() call.
\end{DoxyItemize}

Details in \href{http://docs.python.org/dev/library/argparse.html#action}{\tt original action guide}

\section*{Sub-\/commands }

Argument\+Parser.\+add\+Subparsers()

Many programs split their functionality into a number of sub-\/commands, for example, the svn program can invoke sub-\/commands like {\ttfamily svn checkout}, {\ttfamily svn update}, and {\ttfamily svn commit}. Splitting up functionality this way can be a particularly good idea when a program performs several different functions which require different kinds of command-\/line arguments. {\ttfamily Argument\+Parser} supports creation of such sub-\/commands with {\ttfamily add\+Subparsers()} method. The {\ttfamily add\+Subparsers()} method is normally called with no arguments and returns an special action object. This object has a single method {\ttfamily add\+Parser()}, which takes a command name and any {\ttfamily Argument\+Parser} constructor arguments, and returns an {\ttfamily Argument\+Parser} object that can be modified as usual.

Example\+:

sub\+\_\+commands.\+js 
\begin{DoxyCode}
#!/usr/bin/env node
'use strict';

var ArgumentParser = require('../lib/argparse').ArgumentParser;
var parser = new ArgumentParser(\{
  version: '0.0.1',
  addHelp:true,
  description: 'Argparse examples: sub-commands',
\});

var subparsers = parser.addSubparsers(\{
  title:'subcommands',
  dest:"subcommand\_name"
\});

var bar = subparsers.addParser('c1', \{addHelp:true\});
bar.addArgument(
  [ '-f', '--foo' ],
  \{
    action: 'store',
    help: 'foo3 bar3'
  \}
);
var bar = subparsers.addParser(
  'c2',
  \{aliases:['co'], addHelp:true\}
);
bar.addArgument(
  [ '-b', '--bar' ],
  \{
    action: 'store',
    type: 'int',
    help: 'foo3 bar3'
  \}
);

var args = parser.parseArgs();
console.dir(args);
\end{DoxyCode}


Details in \href{http://docs.python.org/dev/library/argparse.html#sub-commands}{\tt original sub-\/commands guide}

\section*{Contributors }


\begin{DoxyItemize}
\item \href{https://github.com/shkuropat}{\tt Eugene Shkuropat}
\item \href{https://github.com/hpaulj}{\tt Paul Jacobson}
\end{DoxyItemize}

\href{https://github.com/nodeca/argparse/graphs/contributors}{\tt others}

\section*{License }

Copyright (c) 2012 \href{https://github.com/puzrin}{\tt Vitaly Puzrin}. Released under the M\+IT license. See \href{https://github.com/nodeca/argparse/blob/master/LICENSE}{\tt L\+I\+C\+E\+N\+SE} for details. 