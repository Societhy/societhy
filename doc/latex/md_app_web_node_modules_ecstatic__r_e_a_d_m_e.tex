

A simple static file server middleware. Use it with a raw http server, express/connect, or flatiron/union!

\section*{Examples\+:}

\subsection*{express 3.\+0.\+x}


\begin{DoxyCode}
var http = require('http');
var express = require('express');
var ecstatic = require('ecstatic');

var app = express();
app.use(ecstatic(\{ root: \_\_dirname + '/public' \}));
http.createServer(app).listen(8080);

console.log('Listening on :8080');
\end{DoxyCode}


\subsection*{union}


\begin{DoxyCode}
var union = require('union');
var ecstatic = require('ecstatic');

union.createServer(\{
  before: [
    ecstatic(\{ root: \_\_dirname + '/public' \}),
  ]
\}).listen(8080);

console.log('Listening on :8080');
\end{DoxyCode}


\subsection*{stock http server}


\begin{DoxyCode}
var http = require('http');
var ecstatic = require('ecstatic');

http.createServer(
  ecstatic(\{ root: \_\_dirname + '/public' \})
).listen(8080);

console.log('Listening on :8080');
\end{DoxyCode}
 \subsubsection*{fall through}

To allow fall through to your custom routes\+:


\begin{DoxyCode}
ecstatic(\{ root: \_\_dirname + '/public', handleError: false \})
\end{DoxyCode}


\section*{A\+PI\+:}

\subsection*{ecstatic(opts);}

Pass ecstatic an options hash, and it will return your middleware!


\begin{DoxyCode}
var opts = \{
             root          : \_\_dirname + '/public',
             baseDir       : '/',
             cache         : 3600,
             showDir       : false,
             autoIndex     : false,
             humanReadable : true,
             si            : false,
             defaultExt    : 'html',
             gzip          : false
           \}
\end{DoxyCode}


If {\ttfamily opts} is a string, the string is assigned to the root folder and all other options are set to their defaults.

\subsubsection*{{\ttfamily opts.\+root}}

{\ttfamily opts.\+root} is the directory you want to serve up.

\subsubsection*{{\ttfamily opts.\+base\+Dir}}

{\ttfamily opts.\+base\+Dir} is {\ttfamily /} by default, but can be changed to allow your static files to be served off a specific route. For example, if {\ttfamily opts.\+base\+Dir === \char`\"{}blog\char`\"{}} and {\ttfamily opts.\+root = \char`\"{}./public\char`\"{}}, requests for {\ttfamily localhost\+:8080/blog/index.\+html} will resolve to {\ttfamily ./public/index.html}.

\subsubsection*{{\ttfamily opts.\+cache}}

Customize cache control with {\ttfamily opts.\+cache} , if it is a number then it will set max-\/age in seconds. Other wise it will pass through directly to cache-\/control. Time defaults to 3600 s (ie, 1 hour).

\subsubsection*{{\ttfamily opts.\+show\+Dir}}

Turn {\bfseries on} directory listings with {\ttfamily opts.\+show\+Dir === true}. Defaults to {\bfseries false}.

\subsubsection*{{\ttfamily opts.\+auto\+Index}}

Serve {\ttfamily /path/index.html} when {\ttfamily /path/} is requested. Turn {\bfseries off} auto\+Indexing with {\ttfamily opts.\+auto\+Index === false}. Defaults to {\bfseries true}.

\subsubsection*{{\ttfamily opts.\+human\+Readable}}

If auto\+Indexing is enabled, add human-\/readable file sizes. Defaults to {\bfseries true}. Aliases are {\ttfamily humanreadable} and {\ttfamily human-\/readable}.

\subsubsection*{{\ttfamily opts.\+si}}

If auto\+Indexing and human\+Readable are enabled, print file sizes with base 1000 instead of base 1024. Name is inferred from cli options for {\ttfamily ls}. Aliased to {\ttfamily index}, the equivalent option in Apache.

\subsubsection*{{\ttfamily opts.\+default\+Ext}}

Turn on default file extensions with {\ttfamily opts.\+default\+Ext}. If {\ttfamily opts.\+default\+Ext} is true, it will default to {\ttfamily html}. For example if you want a request to {\ttfamily /a-\/file} to resolve to {\ttfamily ./public/a-\/file.html}, set this to {\ttfamily true}. If you want {\ttfamily /a-\/file} to resolve to {\ttfamily ./public/a-\/file.json} instead, set {\ttfamily opts.\+default\+Ext} to {\ttfamily json}.

\subsubsection*{{\ttfamily opts.\+gzip}}

Set {\ttfamily opts.\+gzip === true} in order to turn on \char`\"{}gzip mode,\char`\"{} wherein ecstatic will serve {\ttfamily ./public/some-\/file.js.\+gz} in place of {\ttfamily ./public/some-\/file.js} when the gzipped version exists and ecstatic determines that the behavior is appropriate.

\subsubsection*{{\ttfamily opts.\+handle\+Error}}

Turn {\bfseries off} handle\+Errors to allow fall-\/through with {\ttfamily opts.\+handle\+Error === false}, Defaults to {\bfseries true}.

\subsection*{middleware(req, res, next);}

This works more or less as you\textquotesingle{}d expect.

\subsubsection*{ecstatic.\+show\+Dir(folder);}

This returns another middleware which will attempt to show a directory view. Turning on auto-\/indexing is roughly equivalent to adding this middleware after an ecstatic middleware with autoindexing disabled.

\subsubsection*{{\ttfamily ecstatic} command}

to start a standalone static http server, run {\ttfamily npm install -\/g ecstatic} and then run {\ttfamily ecstatic \mbox{[}dir?\mbox{]} \mbox{[}options\mbox{]} -\/-\/port P\+O\+RT} all options work as above, passed in \href{https://github.com/substack/node-optimist}{\tt optimist} style. {\ttfamily port} defaults to {\ttfamily 8000}. If a {\ttfamily dir} or {\ttfamily -\/-\/root dir} argument is not passed, ecsatic will serve the current dir.

\section*{Tests\+:}

\begin{DoxyVerb}npm test
\end{DoxyVerb}


\section*{License\+:}

M\+IT. 