A dead simple way to do inheritance in JS. \begin{DoxyVerb}var inherits = require("inherits")

function Animal () {
  this.alive = true
}
Animal.prototype.say = function (what) {
  console.log(what)
}

inherits(Dog, Animal)
function Dog () {
  Dog.super.apply(this)
}
Dog.prototype.sniff = function () {
  this.say("sniff sniff")
}
Dog.prototype.bark = function () {
  this.say("woof woof")
}

inherits(Chihuahua, Dog)
function Chihuahua () {
  Chihuahua.super.apply(this)
}
Chihuahua.prototype.bark = function () {
  this.say("yip yip")
}

// also works
function Cat () {
  Cat.super.apply(this)
}
Cat.prototype.hiss = function () {
  this.say("CHSKKSS!!")
}
inherits(Cat, Animal, {
  meow: function () { this.say("miao miao") }
})
Cat.prototype.purr = function () {
  this.say("purr purr")
}


var c = new Chihuahua
assert(c instanceof Chihuahua)
assert(c instanceof Dog)
assert(c instanceof Animal)
\end{DoxyVerb}


The actual function is laughably small. 10-\/lines small. 