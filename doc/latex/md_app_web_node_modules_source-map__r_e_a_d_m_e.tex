This is a library to generate and consume the source map format \href{https://docs.google.com/document/d/1U1RGAehQwRypUTovF1KRlpiOFze0b-_2gc6fAH0KY0k/edit}{\tt described here}.

This library is written in the Asynchronous Module Definition format, and works in the following environments\+:


\begin{DoxyItemize}
\item Modern Browsers supporting E\+C\+M\+A\+Script 5 (either after the build, or with an A\+MD loader such as Require\+JS)
\item Inside Firefox (as a J\+SM file, after the build)
\item With Node\+JS versions 0.\+8.\+X and higher
\end{DoxyItemize}

\subsection*{Node}

\begin{DoxyVerb}$ npm install source-map
\end{DoxyVerb}


\subsection*{Building from Source (for everywhere else)}

Install Node and then run \begin{DoxyVerb}$ git clone https://fitzgen@github.com/mozilla/source-map.git
$ cd source-map
$ npm link .
\end{DoxyVerb}


Next, run \begin{DoxyVerb}$ node Makefile.dryice.js
\end{DoxyVerb}


This should spew a bunch of stuff to stdout, and create the following files\+:


\begin{DoxyItemize}
\item {\ttfamily dist/source-\/map.\+js} -\/ The unminified browser version.
\item {\ttfamily dist/source-\/map.\+min.\+js} -\/ The minified browser version.
\item {\ttfamily dist/\+Source\+Map.\+jsm} -\/ The Java\+Script Module for inclusion in Firefox source.
\end{DoxyItemize}

\subsection*{Examples}

\subsubsection*{Consuming a source map}


\begin{DoxyCode}
var rawSourceMap = \{
  version: 3,
  file: 'min.js',
  names: ['bar', 'baz', 'n'],
  sources: ['one.js', 'two.js'],
  sourceRoot: 'http://example.com/www/js/',
  mappings: 'CAAC,IAAI,IAAM,SAAUA,GAClB,OAAOC,IAAID;CCDb,IAAI,IAAM,SAAUE,GAClB,OAAOA'
\};

var smc = new SourceMapConsumer(rawSourceMap);

console.log(smc.sources);
// [ 'http://example.com/www/js/one.js',
//   'http://example.com/www/js/two.js' ]

console.log(smc.originalPositionFor(\{
  line: 2,
  column: 28
\}));
// \{ source: 'http://example.com/www/js/two.js',
//   line: 2,
//   column: 10,
//   name: 'n' \}

console.log(smc.generatedPositionFor(\{
  source: 'http://example.com/www/js/two.js',
  line: 2,
  column: 10
\}));
// \{ line: 2, column: 28 \}

smc.eachMapping(function (m) \{
  // ...
\});
\end{DoxyCode}


\subsubsection*{Generating a source map}

In depth guide\+: \href{https://hacks.mozilla.org/2013/05/compiling-to-javascript-and-debugging-with-source-maps/}{\tt {\bfseries Compiling to Java\+Script, and Debugging with Source Maps}}

\paragraph*{With Source\+Node (high level A\+PI)}


\begin{DoxyCode}
function compile(ast) \{
  switch (ast.type) \{
  case 'BinaryExpression':
    return new SourceNode(
      ast.location.line,
      ast.location.column,
      ast.location.source,
      [compile(ast.left), " + ", compile(ast.right)]
    );
  case 'Literal':
    return new SourceNode(
      ast.location.line,
      ast.location.column,
      ast.location.source,
      String(ast.value)
    );
  // ...
  default:
    throw new Error("Bad AST");
  \}
\}

var ast = parse("40 + 2", "add.js");
console.log(compile(ast).toStringWithSourceMap(\{
  file: 'add.js'
\}));
// \{ code: '40 + 2',
//   map: [object SourceMapGenerator] \}
\end{DoxyCode}


\paragraph*{With Source\+Map\+Generator (low level A\+PI)}


\begin{DoxyCode}
var map = new SourceMapGenerator(\{
  file: "source-mapped.js"
\});

map.addMapping(\{
  generated: \{
    line: 10,
    column: 35
  \},
  source: "foo.js",
  original: \{
    line: 33,
    column: 2
  \},
  name: "christopher"
\});

console.log(map.toString());
//
       '\{"version":3,"file":"source-mapped.js","sources":["foo.js"],"names":["christopher"],"mappings":";;;;;;;;;mCAgCEA"\}'
\end{DoxyCode}


\subsection*{A\+PI}

Get a reference to the module\+:


\begin{DoxyCode}
// NodeJS
var sourceMap = require('source-map');

// Browser builds
var sourceMap = window.sourceMap;

// Inside Firefox
let sourceMap = \{\};
Components.utils.import('resource:///modules/devtools/SourceMap.jsm', sourceMap);
\end{DoxyCode}


\subsubsection*{Source\+Map\+Consumer}

A Source\+Map\+Consumer instance represents a parsed source map which we can query for information about the original file positions by giving it a file position in the generated source.

\paragraph*{new Source\+Map\+Consumer(raw\+Source\+Map)}

The only parameter is the raw source map (either as a string which can be {\ttfamily J\+S\+O\+N.\+parse}\textquotesingle{}d, or an object). According to the spec, source maps have the following attributes\+:


\begin{DoxyItemize}
\item {\ttfamily version}\+: Which version of the source map spec this map is following.
\item {\ttfamily sources}\+: An array of U\+R\+Ls to the original source files.
\item {\ttfamily names}\+: An array of identifiers which can be referrenced by individual mappings.
\item {\ttfamily source\+Root}\+: Optional. The U\+RL root from which all sources are relative.
\item {\ttfamily sources\+Content}\+: Optional. An array of contents of the original source files.
\item {\ttfamily mappings}\+: A string of base64 V\+L\+Qs which contain the actual mappings.
\item {\ttfamily file}\+: Optional. The generated filename this source map is associated with.
\end{DoxyItemize}

\paragraph*{Source\+Map\+Consumer.\+prototype.\+compute\+Column\+Spans()}

Compute the last column for each generated mapping. The last column is inclusive.

\paragraph*{Source\+Map\+Consumer.\+prototype.\+original\+Position\+For(generated\+Position)}

Returns the original source, line, and column information for the generated source\textquotesingle{}s line and column positions provided. The only argument is an object with the following properties\+:


\begin{DoxyItemize}
\item {\ttfamily line}\+: The line number in the generated source.
\item {\ttfamily column}\+: The column number in the generated source.
\end{DoxyItemize}

and an object is returned with the following properties\+:


\begin{DoxyItemize}
\item {\ttfamily source}\+: The original source file, or null if this information is not available.
\item {\ttfamily line}\+: The line number in the original source, or null if this information is not available.
\item {\ttfamily column}\+: The column number in the original source, or null or null if this information is not available.
\item {\ttfamily name}\+: The original identifier, or null if this information is not available.
\end{DoxyItemize}

\paragraph*{Source\+Map\+Consumer.\+prototype.\+generated\+Position\+For(original\+Position)}

Returns the generated line and column information for the original source, line, and column positions provided. The only argument is an object with the following properties\+:


\begin{DoxyItemize}
\item {\ttfamily source}\+: The filename of the original source.
\item {\ttfamily line}\+: The line number in the original source.
\item {\ttfamily column}\+: The column number in the original source.
\end{DoxyItemize}

and an object is returned with the following properties\+:


\begin{DoxyItemize}
\item {\ttfamily line}\+: The line number in the generated source, or null.
\item {\ttfamily column}\+: The column number in the generated source, or null.
\end{DoxyItemize}

\paragraph*{Source\+Map\+Consumer.\+prototype.\+all\+Generated\+Positions\+For(original\+Position)}

Returns all generated line and column information for the original source and line provided. The only argument is an object with the following properties\+:


\begin{DoxyItemize}
\item {\ttfamily source}\+: The filename of the original source.
\item {\ttfamily line}\+: The line number in the original source.
\end{DoxyItemize}

and an array of objects is returned, each with the following properties\+:


\begin{DoxyItemize}
\item {\ttfamily line}\+: The line number in the generated source, or null.
\item {\ttfamily column}\+: The column number in the generated source, or null.
\end{DoxyItemize}

\paragraph*{Source\+Map\+Consumer.\+prototype.\+source\+Content\+For(source\mbox{[}, return\+Null\+On\+Missing\mbox{]})}

Returns the original source content for the source provided. The only argument is the U\+RL of the original source file.

If the source content for the given source is not found, then an error is thrown. Optionally, pass {\ttfamily true} as the second param to have {\ttfamily null} returned instead.

\paragraph*{Source\+Map\+Consumer.\+prototype.\+each\+Mapping(callback, context, order)}

Iterate over each mapping between an original source/line/column and a generated line/column in this source map.


\begin{DoxyItemize}
\item {\ttfamily callback}\+: The function that is called with each mapping. Mappings have the form {\ttfamily \{ source, generated\+Line, generated\+Column, original\+Line, original\+Column, name \}}
\item {\ttfamily context}\+: Optional. If specified, this object will be the value of {\ttfamily this} every time that {\ttfamily callback} is called.
\item {\ttfamily order}\+: Either {\ttfamily Source\+Map\+Consumer.\+G\+E\+N\+E\+R\+A\+T\+E\+D\+\_\+\+O\+R\+D\+ER} or {\ttfamily Source\+Map\+Consumer.\+O\+R\+I\+G\+I\+N\+A\+L\+\_\+\+O\+R\+D\+ER}. Specifies whether you want to iterate over the mappings sorted by the generated file\textquotesingle{}s line/column order or the original\textquotesingle{}s source/line/column order, respectively. Defaults to {\ttfamily Source\+Map\+Consumer.\+G\+E\+N\+E\+R\+A\+T\+E\+D\+\_\+\+O\+R\+D\+ER}.
\end{DoxyItemize}

\subsubsection*{Source\+Map\+Generator}

An instance of the Source\+Map\+Generator represents a source map which is being built incrementally.

\paragraph*{new Source\+Map\+Generator(\mbox{[}start\+Of\+Source\+Map\mbox{]})}

You may pass an object with the following properties\+:


\begin{DoxyItemize}
\item {\ttfamily file}\+: The filename of the generated source that this source map is associated with.
\item {\ttfamily source\+Root}\+: A root for all relative U\+R\+Ls in this source map.
\item {\ttfamily skip\+Validation}\+: Optional. When {\ttfamily true}, disables validation of mappings as they are added. This can improve performance but should be used with discretion, as a last resort. Even then, one should avoid using this flag when running tests, if possible.
\end{DoxyItemize}

\paragraph*{Source\+Map\+Generator.\+from\+Source\+Map(source\+Map\+Consumer)}

Creates a new Source\+Map\+Generator based on a Source\+Map\+Consumer


\begin{DoxyItemize}
\item {\ttfamily source\+Map\+Consumer} The Source\+Map.
\end{DoxyItemize}

\paragraph*{Source\+Map\+Generator.\+prototype.\+add\+Mapping(mapping)}

Add a single mapping from original source line and column to the generated source\textquotesingle{}s line and column for this source map being created. The mapping object should have the following properties\+:


\begin{DoxyItemize}
\item {\ttfamily generated}\+: An object with the generated line and column positions.
\item {\ttfamily original}\+: An object with the original line and column positions.
\item {\ttfamily source}\+: The original source file (relative to the source\+Root).
\item {\ttfamily name}\+: An optional original token name for this mapping.
\end{DoxyItemize}

\paragraph*{Source\+Map\+Generator.\+prototype.\+set\+Source\+Content(source\+File, source\+Content)}

Set the source content for an original source file.


\begin{DoxyItemize}
\item {\ttfamily source\+File} the U\+RL of the original source file.
\item {\ttfamily source\+Content} the content of the source file.
\end{DoxyItemize}

\paragraph*{Source\+Map\+Generator.\+prototype.\+apply\+Source\+Map(source\+Map\+Consumer\mbox{[}, source\+File\mbox{[}, source\+Map\+Path\mbox{]}\mbox{]})}

Applies a Source\+Map for a source file to the Source\+Map. Each mapping to the supplied source file is rewritten using the supplied Source\+Map. Note\+: The resolution for the resulting mappings is the minimium of this map and the supplied map.


\begin{DoxyItemize}
\item {\ttfamily source\+Map\+Consumer}\+: The Source\+Map to be applied.
\item {\ttfamily source\+File}\+: Optional. The filename of the source file. If omitted, source\+Map\+Consumer.\+file will be used, if it exists. Otherwise an error will be thrown.
\item {\ttfamily source\+Map\+Path}\+: Optional. The dirname of the path to the Source\+Map to be applied. If relative, it is relative to the Source\+Map.

This parameter is needed when the two Source\+Maps aren\textquotesingle{}t in the same directory, and the Source\+Map to be applied contains relative source paths. If so, those relative source paths need to be rewritten relative to the Source\+Map.

If omitted, it is assumed that both Source\+Maps are in the same directory, thus not needing any rewriting. (Supplying `\textquotesingle{}.\textquotesingle{}` has the same effect.)
\end{DoxyItemize}

\paragraph*{Source\+Map\+Generator.\+prototype.\+to\+String()}

Renders the source map being generated to a string.

\subsubsection*{Source\+Node}

Source\+Nodes provide a way to abstract over interpolating and/or concatenating snippets of generated Java\+Script source code, while maintaining the line and column information associated between those snippets and the original source code. This is useful as the final intermediate representation a compiler might use before outputting the generated JS and source map.

\paragraph*{new Source\+Node(\mbox{[}line, column, source\mbox{[}, chunk\mbox{[}, name\mbox{]}\mbox{]}\mbox{]})}


\begin{DoxyItemize}
\item {\ttfamily line}\+: The original line number associated with this source node, or null if it isn\textquotesingle{}t associated with an original line.
\item {\ttfamily column}\+: The original column number associated with this source node, or null if it isn\textquotesingle{}t associated with an original column.
\item {\ttfamily source}\+: The original source\textquotesingle{}s filename; null if no filename is provided.
\item {\ttfamily chunk}\+: Optional. Is immediately passed to {\ttfamily Source\+Node.\+prototype.\+add}, see below.
\item {\ttfamily name}\+: Optional. The original identifier.
\end{DoxyItemize}

\paragraph*{Source\+Node.\+from\+String\+With\+Source\+Map(code, source\+Map\+Consumer\mbox{[}, relative\+Path\mbox{]})}

Creates a Source\+Node from generated code and a Source\+Map\+Consumer.


\begin{DoxyItemize}
\item {\ttfamily code}\+: The generated code
\item {\ttfamily source\+Map\+Consumer} The Source\+Map for the generated code
\item {\ttfamily relative\+Path} The optional path that relative sources in {\ttfamily source\+Map\+Consumer} should be relative to.
\end{DoxyItemize}

\paragraph*{Source\+Node.\+prototype.\+add(chunk)}

Add a chunk of generated JS to this source node.


\begin{DoxyItemize}
\item {\ttfamily chunk}\+: A string snippet of generated JS code, another instance of {\ttfamily Source\+Node}, or an array where each member is one of those things.
\end{DoxyItemize}

\paragraph*{Source\+Node.\+prototype.\+prepend(chunk)}

Prepend a chunk of generated JS to this source node.


\begin{DoxyItemize}
\item {\ttfamily chunk}\+: A string snippet of generated JS code, another instance of {\ttfamily Source\+Node}, or an array where each member is one of those things.
\end{DoxyItemize}

\paragraph*{Source\+Node.\+prototype.\+set\+Source\+Content(source\+File, source\+Content)}

Set the source content for a source file. This will be added to the {\ttfamily Source\+Map} in the {\ttfamily sources\+Content} field.


\begin{DoxyItemize}
\item {\ttfamily source\+File}\+: The filename of the source file
\item {\ttfamily source\+Content}\+: The content of the source file
\end{DoxyItemize}

\paragraph*{Source\+Node.\+prototype.\+walk(fn)}

Walk over the tree of JS snippets in this node and its children. The walking function is called once for each snippet of JS and is passed that snippet and the its original associated source\textquotesingle{}s line/column location.


\begin{DoxyItemize}
\item {\ttfamily fn}\+: The traversal function.
\end{DoxyItemize}

\paragraph*{Source\+Node.\+prototype.\+walk\+Source\+Contents(fn)}

Walk over the tree of Source\+Nodes. The walking function is called for each source file content and is passed the filename and source content.


\begin{DoxyItemize}
\item {\ttfamily fn}\+: The traversal function.
\end{DoxyItemize}

\paragraph*{Source\+Node.\+prototype.\+join(sep)}

Like {\ttfamily Array.\+prototype.\+join} except for Source\+Nodes. Inserts the separator between each of this source node\textquotesingle{}s children.


\begin{DoxyItemize}
\item {\ttfamily sep}\+: The separator.
\end{DoxyItemize}

\paragraph*{Source\+Node.\+prototype.\+replace\+Right(pattern, replacement)}

Call {\ttfamily String.\+prototype.\+replace} on the very right-\/most source snippet. Useful for trimming whitespace from the end of a source node, etc.


\begin{DoxyItemize}
\item {\ttfamily pattern}\+: The pattern to replace.
\item {\ttfamily replacement}\+: The thing to replace the pattern with.
\end{DoxyItemize}

\paragraph*{Source\+Node.\+prototype.\+to\+String()}

Return the string representation of this source node. Walks over the tree and concatenates all the various snippets together to one string.

\paragraph*{Source\+Node.\+prototype.\+to\+String\+With\+Source\+Map(\mbox{[}start\+Of\+Source\+Map\mbox{]})}

Returns the string representation of this tree of source nodes, plus a Source\+Map\+Generator which contains all the mappings between the generated and original sources.

The arguments are the same as those to {\ttfamily new Source\+Map\+Generator}.

\subsection*{Tests}

\href{https://travis-ci.org/mozilla/source-map}{\tt }

Install Node\+JS version 0.\+8.\+0 or greater, then run {\ttfamily node test/run-\/tests.\+js}.

To add new tests, create a new file named {\ttfamily test/test-\/$<$your new test name$>$.js} and export your test functions with names that start with \char`\"{}test\char`\"{}, for example


\begin{DoxyCode}
exports["test doing the foo bar"] = function (assert, util) \{
  ...
\};
\end{DoxyCode}


The new test will be located automatically when you run the suite.

The {\ttfamily util} argument is the test utility module located at {\ttfamily test/source-\/map/util}.

The {\ttfamily assert} argument is a cut down version of node\textquotesingle{}s assert module. You have access to the following assertion functions\+:


\begin{DoxyItemize}
\item {\ttfamily does\+Not\+Throw}
\item {\ttfamily equal}
\item {\ttfamily ok}
\item {\ttfamily strict\+Equal}
\item {\ttfamily throws}
\end{DoxyItemize}

(The reason for the restricted set of test functions is because we need the tests to run inside Firefox\textquotesingle{}s test suite as well and so the assert module is shimmed in that environment. See {\ttfamily build/assert-\/shim.\+js}.) 