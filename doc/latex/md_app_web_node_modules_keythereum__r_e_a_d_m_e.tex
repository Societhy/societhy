\href{https://travis-ci.org/ethereumjs/keythereum}{\tt } \href{https://coveralls.io/github/ethereumjs/keythereum?branch=master}{\tt } \href{http://badge.fury.io/js/keythereum}{\tt }

Keythereum is a Java\+Script tool to generate, import and export Ethereum keys. This provides a simple way to use the same account locally and in web wallets. It can be used for verifiable cold storage wallets.

Keythereum uses the same key derivation functions (P\+B\+K\+D\+F2-\/\+S\+H\+A256 or scrypt), symmetric ciphers (A\+E\+S-\/128-\/\+C\+TR or A\+E\+S-\/128-\/\+C\+BC), and message authentication codes as \href{https://github.com/ethereum/go-ethereum}{\tt geth}. You can export your generated key to file, copy it to your data directory\textquotesingle{}s keystore, and immediately start using it in your local Ethereum client.

\subsection*{Installation }

\begin{DoxyVerb}$ npm install keythereum
\end{DoxyVerb}


\subsection*{Usage }

To use keythereum in Node.\+js, just {\ttfamily require} it\+: 
\begin{DoxyCode}
var keythereum = require("keythereum");
\end{DoxyCode}
 A minified, browserified file {\ttfamily dist/keythereum.\+min.\+js} is included for use in the browser. Including this file simply attaches the {\ttfamily keythereum} object to {\ttfamily window}\+: 
\begin{DoxyCode}
<script src="dist/keythereum.min.js" type="text/javascript"></script>
\end{DoxyCode}


\subsubsection*{Key creation}

Generate a new random private key (256 bit), as well as the salt (256 bit) used by the key derivation function, and the initialization vector (128 bit) used to A\+E\+S-\/128-\/\+C\+TR encrypt the key. {\ttfamily create} is asynchronous if it is passed a callback function, and synchronous otherwise. 
\begin{DoxyCode}
// optional private key and initialization vector sizes in bytes
// (if params is not passed to create, keythereum.constants is used by default)
var params = \{ keyBytes: 32, ivBytes: 16 \};

// synchronous
var dk = keythereum.create(params);
// dk:
\{
    privateKey: <Buffer ...>,
    iv: <Buffer ...>,
    salt: <Buffer ...>
\}

// asynchronous
keythereum.create(params, function (dk) \{
    // do stuff!
\});
\end{DoxyCode}


\subsubsection*{Key export}

You will need to specify a password and (optionally) a key derivation function. If unspecified, P\+B\+K\+D\+F2-\/\+S\+H\+A256 will be used to derive the A\+ES secret key. 
\begin{DoxyCode}
var password = "wheethereum";
var kdf = "pbkdf2"; // or "scrypt" to use the scrypt kdf
\end{DoxyCode}
 The {\ttfamily dump} function is used to export key info to keystore \href{https://github.com/ethereum/wiki/wiki/Web3-Secret-Storage-Definition}{\tt \char`\"{}secret-\/storage\char`\"{} format}. If a callback function is supplied as the sixth parameter to {\ttfamily dump}, it will run asynchronously\+: 
\begin{DoxyCode}
// if options is not passed to dump, it will use the values in
// keythereum.constants by default
var options = \{
    kdf: "pbkdf2",
    cipher: "aes-128-ctr",
    kdfparams: \{
        c: 262144,
        dklen: 32,
        prf: "hmac-sha256"
    \}
\};

// synchronous
var keyObject = keythereum.dump(password, dk.privateKey, dk.salt, dk.iv, options);
// keyObject:
\{
    address: '008aeeda4d805471df9b2a5b0f38a0c3bcba786b',
    Crypto: \{
        cipher: 'aes-128-ctr',
        ciphertext: '5318b4d5bcd28de64ee5559e671353e16f075ecae9f99c7a79a38af5f869aa46',
        cipherparams: \{
            iv: '6087dab2f9fdbbfaddc31a909735c1e6'
        \},
        mac: '517ead924a9d0dc3124507e3393d175ce3ff7c1e96529c6c555ce9e51205e9b2',
        kdf: 'pbkdf2',
        kdfparams: \{
            c: 262144,
            dklen: 32,
            prf: 'hmac-sha256',
            salt: 'ae3cd4e7013836a3df6bd7241b12db061dbe2c6785853cce422d148a624ce0bd'
        \}
    \},
    id: 'e13b209c-3b2f-4327-bab0-3bef2e51630d',
    version: 3
\}

// asynchronous
keythereum.dump(password, dk.privateKey, dk.salt, dk.iv, options, function (keyObject) \{
    // do stuff!
\});
\end{DoxyCode}
 {\ttfamily dump} creates an object and not a J\+S\+ON string. In Node, the {\ttfamily export\+To\+File} method provides an easy way to export this formatted key object to file. It creates a J\+S\+ON file in the {\ttfamily keystore} sub-\/directory, and uses geth\textquotesingle{}s current file-\/naming convention (I\+SO timestamp concatenated with the key\textquotesingle{}s derived Ethereum address). 
\begin{DoxyCode}
keythereum.exportToFile(keyObject);
\end{DoxyCode}
 After successful key export, you will see a message like\+: 
\begin{DoxyCode}
Saved to file:
keystore/UTC--2015-08-11T06:13:53.359Z--008aeeda4d805471df9b2a5b0f38a0c3bcba786b

To use with geth, copy this file to your Ethereum keystore folder
(usually ~/.ethereum/keystore).
\end{DoxyCode}


\subsubsection*{Key import}

Importing a key from geth\textquotesingle{}s keystore can only be done on Node. The J\+S\+ON file is parsed into an object with the same structure as {\ttfamily key\+Object} above. 
\begin{DoxyCode}
// specify a data directory (optional; defaults to ~/.ethereum)
var datadir = "/home/jack/.ethereum-test";

// synchronous
var keyObject = keythereum.importFromFile(address, datadir);

// asynchronous
keythereum.importFromFile(address, datadir, function (keyObject) \{
    // do stuff
\});
\end{DoxyCode}
 This has been tested with version 3 and version 1, but not version 2, keys. (Please send me a version 2 keystore file if you have one, so I can test it!)

To recover the plaintext private key from the key object, use {\ttfamily keythereum.\+recover}. The private key is returned as a Buffer. 
\begin{DoxyCode}
// synchronous
var privateKey = keythereum.recover(password, keyObject);
// privateKey:
<Buffer ...>

// asynchronous
keythereum.recover(password, keyObject, function (privateKey) \{
    // do stuff
\});
\end{DoxyCode}


\subsubsection*{Hashing rounds}

By default, keythereum uses 65536 hashing rounds in its key derivation functions, compared to the 262144 geth uses by default. (Keythereum\textquotesingle{}s J\+S\+ON output files are still compatible with geth, however, since they tell geth how many rounds to use.) These values are user-\/editable\+: {\ttfamily keythereum.\+constants.\+pbkdf2.\+c} is the number of rounds for P\+B\+K\+D\+F2, and {\ttfamily keythereum.\+constants.\+scrypt.\+n} is the number of rounds for scrypt.

\subsection*{Tests }

Unit tests are in the {\ttfamily test} directory, and can be run with mocha\+: \begin{DoxyVerb}$ npm test
\end{DoxyVerb}


{\ttfamily test/geth.\+js} is an integration test, which is run (along with {\ttfamily test/keys.\+js}) using\+: \begin{DoxyVerb}$ npm run geth
\end{DoxyVerb}


{\ttfamily geth.\+js} generates 1000 random private keys, encrypts each key using a randomly-\/generated passphrase, dumps the encrypted key info to a J\+S\+ON file, then spawns a geth instance and attempts to unlock each account using its passphrase and J\+S\+ON file. The passphrases are between 1 and 100 random bytes. Each passphrase is tested in both hexadecimal and base-\/64 encodings, and with P\+B\+K\+D\+F2-\/\+S\+H\+A256 and scrypt key derivation functions.

By default, the flags passed to geth are\+: \begin{DoxyVerb}$ geth --etherbase <account> --unlock <account> --nodiscover --networkid "10101" --port 30304 --rpcport 8547 --datadir test/fixtures --password test/fixtures/.password
\end{DoxyVerb}


{\ttfamily test/fixtures/.password} is a file which contains the passphrase. The {\ttfamily .password} file, as well as the J\+S\+ON key files generated by {\ttfamily geth.\+js}, are automatically deleted after the test.

(Note\+: {\ttfamily geth.\+js} conducts 4000 tests, each of which can take up to 5 seconds, so running this file can take up to 5.\+56 hours.) 