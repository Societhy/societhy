\href{https://travis-ci.org/mishoo/UglifyJS2}{\tt }

Uglify\+JS is a Java\+Script parser, minifier, compressor or beautifier toolkit.

This page documents the command line utility. For \href{http://lisperator.net/uglifyjs/}{\tt A\+PI and internals documentation see my website}. There\textquotesingle{}s also an \href{http://lisperator.net/uglifyjs/#demo}{\tt in-\/browser online demo} (for Firefox, Chrome and probably Safari).

\subsection*{Install }

First make sure you have installed the latest version of \href{http://nodejs.org/}{\tt node.\+js} (You may need to restart your computer after this step).

From N\+PM for use as a command line app\+: \begin{DoxyVerb}npm install uglify-js -g
\end{DoxyVerb}


From N\+PM for programmatic use\+: \begin{DoxyVerb}npm install uglify-js
\end{DoxyVerb}


From Git\+: \begin{DoxyVerb}git clone git://github.com/mishoo/UglifyJS2.git
cd UglifyJS2
npm link .
\end{DoxyVerb}


\subsection*{Usage }

\begin{DoxyVerb}uglifyjs [input files] [options]
\end{DoxyVerb}


Uglify\+J\+S2 can take multiple input files. It\textquotesingle{}s recommended that you pass the input files first, then pass the options. Uglify\+JS will parse input files in sequence and apply any compression options. The files are parsed in the same global scope, that is, a reference from a file to some variable/function declared in another file will be matched properly.

If you want to read from S\+T\+D\+IN instead, pass a single dash instead of input files.

If you wish to pass your options before the input files, separate the two with a double dash to prevent input files being used as option arguments\+: \begin{DoxyVerb}uglifyjs --compress --mangle -- input.js
\end{DoxyVerb}


The available options are\+:


\begin{DoxyCode}
--source-map                  Specify an output file where to generate source
                              map.
--source-map-root             The path to the original source to be included
                              in the source map.
--source-map-url              The path to the source map to be added in //#
                              sourceMappingURL.  Defaults to the value passed
                              with --source-map.
--source-map-include-sources  Pass this flag if you want to include the
                              content of source files in the source map as
                              sourcesContent property.
--in-source-map               Input source map, useful if you're compressing
                              JS that was generated from some other original
                              code.
--screw-ie8                   Pass this flag if you don't care about full
                              compliance with Internet Explorer 6-8 quirks
                              (by default UglifyJS will try to be IE-proof).
--expr                        Parse a single expression, rather than a
                              program (for parsing JSON)
-p, --prefix                  Skip prefix for original filenames that appear
                              in source maps. For example -p 3 will drop 3
                              directories from file names and ensure they are
                              relative paths. You can also specify -p
                              relative, which will make UglifyJS figure out
                              itself the relative paths between original
                              sources, the source map and the output file.
-o, --output                  Output file (default STDOUT).
-b, --beautify                Beautify output/specify output options.
-m, --mangle                  Mangle names/pass mangler options.
-r, --reserved                Reserved names to exclude from mangling.
-c, --compress                Enable compressor/pass compressor options. Pass
                              options like -c
                              hoist\_vars=false,if\_return=false. Use -c with
                              no argument to use the default compression
                              options.
-d, --define                  Global definitions
-e, --enclose                 Embed everything in a big function, with a
                              configurable parameter/argument list.
--comments                    Preserve copyright comments in the output. By
                              default this works like Google Closure, keeping
                              JSDoc-style comments that contain "@license" or
                              "@preserve". You can optionally pass one of the
                              following arguments to this flag:
                              - "all" to keep all comments
                              - a valid JS regexp (needs to start with a
                              slash) to keep only comments that match.
                              Note that currently not *all* comments can be
                              kept when compression is on, because of dead
                              code removal or cascading statements into
                              sequences.
--preamble                    Preamble to prepend to the output.  You can use
                              this to insert a comment, for example for
                              licensing information.  This will not be
                              parsed, but the source map will adjust for its
                              presence.
--stats                       Display operations run time on STDERR.
--acorn                       Use Acorn for parsing.
--spidermonkey                Assume input files are SpiderMonkey AST format
                              (as JSON).
--self                        Build itself (UglifyJS2) as a library (implies
                              --wrap=UglifyJS --export-all)
--wrap                        Embed everything in a big function, making the
                              “exports” and “global” variables available. You
                              need to pass an argument to this option to
                              specify the name that your module will take
                              when included in, say, a browser.
--export-all                  Only used when --wrap, this tells UglifyJS to
                              add code to automatically export all globals.
--lint                        Display some scope warnings
-v, --verbose                 Verbose
-V, --version                 Print version number and exit.
--noerr                       Don't throw an error for unknown options in -c,
                              -b or -m.
--bare-returns                Allow return outside of functions.  Useful when
                              minifying CommonJS modules and Userscripts that
                              may be anonymous function wrapped (IIFE) by the
                              .user.js engine `caller`.
--keep-fnames                 Do not mangle/drop function names.  Useful for
                              code relying on Function.prototype.name.
--reserved-file               File containing reserved names
--reserve-domprops            Make (most?) DOM properties reserved for
                              --mangle-props
--mangle-props                Mangle property names (default `0`). Set to 
                              `true` or `1` to mangle all property names. Set
                              to `unquoted` or `2` to only mangle unquoted 
                              property names. Mode `2` also enables the
                              `keep\_quoted\_props` beautifier option to 
                              preserve the quotes around property names and
                              disables the `properties` compressor option to
                              prevent rewriting quoted properties with dot
                              notation. You can override these by setting
                              them explicitly on the command line.
--mangle-regex                Only mangle property names matching the regex
--name-cache                  File to hold mangled names mappings
--pure-funcs                  List of functions that can be safely removed if
                              their return value is not used           [array]
\end{DoxyCode}


Specify {\ttfamily -\/-\/output} ({\ttfamily -\/o}) to declare the output file. Otherwise the output goes to S\+T\+D\+O\+UT.

\subsection*{Source map options}

Uglify\+J\+S2 can generate a source map file, which is highly useful for debugging your compressed Java\+Script. To get a source map, pass {\ttfamily -\/-\/source-\/map output.\+js.\+map} (full path to the file where you want the source map dumped).

Additionally you might need {\ttfamily -\/-\/source-\/map-\/root} to pass the U\+RL where the original files can be found. In case you are passing full paths to input files to Uglify\+JS, you can use {\ttfamily -\/-\/prefix} ({\ttfamily -\/p}) to specify the number of directories to drop from the path prefix when declaring files in the source map.

For example\+: \begin{DoxyVerb}uglifyjs /home/doe/work/foo/src/js/file1.js \
         /home/doe/work/foo/src/js/file2.js \
         -o foo.min.js \
         --source-map foo.min.js.map \
         --source-map-root http://foo.com/src \
         -p 5 -c -m
\end{DoxyVerb}


The above will compress and mangle {\ttfamily file1.\+js} and {\ttfamily file2.\+js}, will drop the output in {\ttfamily foo.\+min.\+js} and the source map in {\ttfamily foo.\+min.\+js.\+map}. The source mapping will refer to {\ttfamily \href{http://foo.com/src/js/file1.js}{\tt http\+://foo.\+com/src/js/file1.\+js}} and {\ttfamily \href{http://foo.com/src/js/file2.js}{\tt http\+://foo.\+com/src/js/file2.\+js}} (in fact it will list {\ttfamily \href{http://foo.com/src}{\tt http\+://foo.\+com/src}} as the source map root, and the original files as {\ttfamily js/file1.\+js} and {\ttfamily js/file2.\+js}).

\subsubsection*{Composed source map}

When you\textquotesingle{}re compressing JS code that was output by a compiler such as Coffee\+Script, mapping to the JS code won\textquotesingle{}t be too helpful. Instead, you\textquotesingle{}d like to map back to the original code (i.\+e. Coffee\+Script). Uglify\+JS has an option to take an input source map. Assuming you have a mapping from Coffee\+Script → compiled JS, Uglify\+JS can generate a map from Coffee\+Script → compressed JS by mapping every token in the compiled JS to its original location.

To use this feature you need to pass {\ttfamily -\/-\/in-\/source-\/map /path/to/input/source.map}. Normally the input source map should also point to the file containing the generated JS, so if that\textquotesingle{}s correct you can omit input files from the command line.

\subsection*{Mangler options}

To enable the mangler you need to pass {\ttfamily -\/-\/mangle} ({\ttfamily -\/m}). The following (comma-\/separated) options are supported\+:


\begin{DoxyItemize}
\item {\ttfamily toplevel} — mangle names declared in the toplevel scope (disabled by default).
\item {\ttfamily eval} — mangle names visible in scopes where {\ttfamily eval} or {\ttfamily with} are used (disabled by default).
\end{DoxyItemize}

When mangling is enabled but you want to prevent certain names from being mangled, you can declare those names with {\ttfamily -\/-\/reserved} ({\ttfamily -\/r}) — pass a comma-\/separated list of names. For example\+: \begin{DoxyVerb}uglifyjs ... -m -r '$,require,exports'
\end{DoxyVerb}


to prevent the {\ttfamily require}, {\ttfamily exports} and {\ttfamily \$} names from being changed.

\subsubsection*{Mangling property names ({\ttfamily -\/-\/mangle-\/props})}

{\bfseries Note\+:} this will probably break your code. Mangling property names is a separate step, different from variable name mangling. Pass {\ttfamily -\/-\/mangle-\/props}. It will mangle all properties that are seen in some object literal, or that are assigned to. For example\+:


\begin{DoxyCode}
var x = \{
  foo: 1
\};

x.bar = 2;
x["baz"] = 3;
x[condition ? "moo" : "boo"] = 4;
console.log(x.something());
\end{DoxyCode}


In the above code, {\ttfamily foo}, {\ttfamily bar}, {\ttfamily baz}, {\ttfamily moo} and {\ttfamily boo} will be replaced with single characters, while {\ttfamily something()} will be left as is.

In order for this to be of any use, we should avoid mangling standard JS names. For instance, if your code would contain {\ttfamily x.\+length = 10}, then {\ttfamily length} becomes a candidate for mangling and it will be mangled throughout the code, regardless if it\textquotesingle{}s being used as part of your own objects or accessing an array\textquotesingle{}s length. To avoid that, you can use {\ttfamily -\/-\/reserved-\/file} to pass a filename that should contain the names to be excluded from mangling. This file can be used both for excluding variable names and property names. It could look like this, for example\+:


\begin{DoxyCode}
\{
  "vars": [ "define", "require", ... ],
  "props": [ "length", "prototype", ... ]
\}
\end{DoxyCode}


{\ttfamily -\/-\/reserved-\/file} can be an array of file names (either a single comma-\/separated argument, or you can pass multiple {\ttfamily -\/-\/reserved-\/file} arguments) — in this case it will exclude names from all those files.

A default exclusion file is provided in {\ttfamily tools/domprops.\+json} which should cover most standard JS and D\+OM properties defined in various browsers. Pass {\ttfamily -\/-\/reserve-\/domprops} to read that in.

You can also use a regular expression to define which property names should be mangled. For example, {\ttfamily -\/-\/mangle-\/regex=\char`\"{}/$^\wedge$\+\_\+/\char`\"{}} will only mangle property names that start with an underscore.

When you compress multiple files using this option, in order for them to work together in the end we need to ensure somehow that one property gets mangled to the same name in all of them. For this, pass {\ttfamily -\/-\/name-\/cache filename.\+json} and Uglify\+JS will maintain these mappings in a file which can then be reused. It should be initially empty. Example\+:


\begin{DoxyCode}
rm -f /tmp/cache.json  # start fresh
uglifyjs file1.js file2.js --mangle-props --name-cache /tmp/cache.json -o part1.js
uglifyjs file3.js file4.js --mangle-props --name-cache /tmp/cache.json -o part2.js
\end{DoxyCode}


Now, {\ttfamily part1.\+js} and {\ttfamily part2.\+js} will be consistent with each other in terms of mangled property names.

Using the name cache is not necessary if you compress all your files in a single call to Uglify\+JS.

\subsection*{Compressor options}

You need to pass {\ttfamily -\/-\/compress} ({\ttfamily -\/c}) to enable the compressor. Optionally you can pass a comma-\/separated list of options. Options are in the form {\ttfamily foo=bar}, or just {\ttfamily foo} (the latter implies a boolean option that you want to set {\ttfamily true}; it\textquotesingle{}s effectively a shortcut for {\ttfamily foo=true}).


\begin{DoxyItemize}
\item {\ttfamily sequences} -- join consecutive simple statements using the comma operator
\item {\ttfamily properties} -- rewrite property access using the dot notation, for example {\ttfamily foo\mbox{[}\char`\"{}bar\char`\"{}\mbox{]} → foo.\+bar}
\item {\ttfamily dead\+\_\+code} -- remove unreachable code
\item {\ttfamily drop\+\_\+debugger} -- remove {\ttfamily debugger;} statements
\item {\ttfamily unsafe} (default\+: false) -- apply \char`\"{}unsafe\char`\"{} transformations (discussion below)
\item {\ttfamily unsafe\+\_\+comps} (default\+: false) -- Reverse {\ttfamily $<$} and {\ttfamily $<$=} to {\ttfamily $>$} and {\ttfamily $>$=} to allow improved compression. This might be unsafe when an at least one of two operands is an object with computed values due the use of methods like {\ttfamily get}, or {\ttfamily value\+Of}. This could cause change in execution order after operands in the comparison are switching. Compression only works if both {\ttfamily comparisons} and {\ttfamily unsafe\+\_\+comps} are both set to true.
\item {\ttfamily conditionals} -- apply optimizations for {\ttfamily if}-\/s and conditional expressions
\item {\ttfamily comparisons} -- apply certain optimizations to binary nodes, for example\+: {\ttfamily !(a $<$= b) → a $>$ b} (only when {\ttfamily unsafe\+\_\+comps}), attempts to negate binary nodes, e.\+g. {\ttfamily a = !b \&\& !c \&\& !d \&\& !e → a=!(b$\vert$$\vert$c$\vert$$\vert$d$\vert$$\vert$e)} etc.
\item {\ttfamily evaluate} -- attempt to evaluate constant expressions
\item {\ttfamily booleans} -- various optimizations for boolean context, for example {\ttfamily !!a ? b \+: c → a ? b \+: c}
\item {\ttfamily loops} -- optimizations for {\ttfamily do}, {\ttfamily while} and {\ttfamily for} loops when we can statically determine the condition
\item {\ttfamily unused} -- drop unreferenced functions and variables
\item {\ttfamily hoist\+\_\+funs} -- hoist function declarations
\item {\ttfamily hoist\+\_\+vars} (default\+: false) -- hoist {\ttfamily var} declarations (this is {\ttfamily false} by default because it seems to increase the size of the output in general)
\item {\ttfamily if\+\_\+return} -- optimizations for if/return and if/continue
\item {\ttfamily join\+\_\+vars} -- join consecutive {\ttfamily var} statements
\item {\ttfamily cascade} -- small optimization for sequences, transform {\ttfamily x, x} into {\ttfamily x} and {\ttfamily x = something(), x} into {\ttfamily x = something()}
\item {\ttfamily collapse\+\_\+vars} -- default {\ttfamily false}. Collapse single-\/use {\ttfamily var} and {\ttfamily const} definitions when possible.
\item {\ttfamily warnings} -- display warnings when dropping unreachable code or unused declarations etc.
\item {\ttfamily negate\+\_\+iife} -- negate \char`\"{}\+Immediately-\/\+Called Function Expressions\char`\"{} where the return value is discarded, to avoid the parens that the code generator would insert.
\item {\ttfamily pure\+\_\+getters} -- the default is {\ttfamily false}. If you pass {\ttfamily true} for this, Uglify\+JS will assume that object property access (e.\+g. {\ttfamily foo.\+bar} or {\ttfamily foo\mbox{[}\char`\"{}bar\char`\"{}\mbox{]}}) doesn\textquotesingle{}t have any side effects.
\item {\ttfamily pure\+\_\+funcs} -- default {\ttfamily null}. You can pass an array of names and Uglify\+JS will assume that those functions do not produce side effects. D\+A\+N\+G\+ER\+: will not check if the name is redefined in scope. An example case here, for instance {\ttfamily var q = Math.\+floor(a/b)}. If variable {\ttfamily q} is not used elsewhere, Uglify\+JS will drop it, but will still keep the {\ttfamily Math.\+floor(a/b)}, not knowing what it does. You can pass `pure\+\_\+funcs\+: \mbox{[} \textquotesingle{}Math.\+floor\textquotesingle{} \mbox{]}` to let it know that this function won\textquotesingle{}t produce any side effect, in which case the whole statement would get discarded. The current implementation adds some overhead (compression will be slower).
\item {\ttfamily drop\+\_\+console} -- default {\ttfamily false}. Pass {\ttfamily true} to discard calls to {\ttfamily console.$\ast$} functions.
\item {\ttfamily keep\+\_\+fargs} -- default {\ttfamily true}. Prevents the compressor from discarding unused function arguments. You need this for code which relies on {\ttfamily Function.\+length}.
\item {\ttfamily keep\+\_\+fnames} -- default {\ttfamily false}. Pass {\ttfamily true} to prevent the compressor from mangling/discarding function names. Useful for code relying on {\ttfamily Function.\+prototype.\+name}.
\item {\ttfamily passes} -- default {\ttfamily 1}. Number of times to run compress. Use an integer argument larger than 1 to further reduce code size in some cases. Note\+: raising the number of passes will increase uglify compress time.
\end{DoxyItemize}

\subsubsection*{The {\ttfamily unsafe} option}

It enables some transformations that {\itshape might} break code logic in certain contrived cases, but should be fine for most code. You might want to try it on your own code, it should reduce the minified size. Here\textquotesingle{}s what happens when this flag is on\+:


\begin{DoxyItemize}
\item {\ttfamily new Array(1, 2, 3)} or {\ttfamily Array(1, 2, 3)} → {\ttfamily \mbox{[} 1, 2, 3 \mbox{]}}
\item {\ttfamily new Object()} → {\ttfamily \{\}}
\item {\ttfamily String(exp)} or {\ttfamily exp.\+to\+String()} → {\ttfamily \char`\"{}\char`\"{} + exp}
\item {\ttfamily new Object/\+Reg\+Exp/\+Function/\+Error/\+Array (...)} → we discard the {\ttfamily new}
\item {\ttfamily typeof foo == \char`\"{}undefined\char`\"{}} → {\ttfamily foo === void 0}
\item {\ttfamily void 0} → {\ttfamily undefined} (if there is a variable named \char`\"{}undefined\char`\"{} in scope; we do it because the variable name will be mangled, typically reduced to a single character)
\end{DoxyItemize}

\subsubsection*{Conditional compilation}

You can use the {\ttfamily -\/-\/define} ({\ttfamily -\/d}) switch in order to declare global variables that Uglify\+JS will assume to be constants (unless defined in scope). For example if you pass {\ttfamily -\/-\/define D\+E\+B\+UG=false} then, coupled with dead code removal Uglify\+JS will discard the following from the output\+: 
\begin{DoxyCode}
if (DEBUG) \{
    console.log("debug stuff");
\}
\end{DoxyCode}


Uglify\+JS will warn about the condition being always false and about dropping unreachable code; for now there is no option to turn off only this specific warning, you can pass {\ttfamily warnings=false} to turn off {\itshape all} warnings.

Another way of doing that is to declare your globals as constants in a separate file and include it into the build. For example you can have a {\ttfamily build/defines.\+js} file with the following\+: ```javascript const D\+E\+B\+UG = false; const P\+R\+O\+D\+U\+C\+T\+I\+ON = true; // Alternative for environments that don\textquotesingle{}t support {\ttfamily const} /$\ast$$\ast$  $\ast$/ var S\+T\+A\+G\+I\+NG = false; // etc. 
\begin{DoxyCode}
and build your code like this:

    uglifyjs build/defines.js js/foo.js js/bar.js... -c

UglifyJS will notice the constants and, since they cannot be altered, it
will evaluate references to them to the value itself and drop unreachable
code as usual.  The build will contain the `const` declarations if you use
them. If you are targeting < ES6 environments, use `/** @const */ var`.

<a name="codegen-options"></a>

#### Conditional compilation, API
You can also use conditional compilation via the programmatic API. With the difference that the
property name is `global\_defs` and is a compressor property:

```js
uglifyJS.minify([ "input.js"], \{
    compress: \{
        dead\_code: true,
        global\_defs: \{
            DEBUG: false
        \}
    \}
\});
\end{DoxyCode}


\subsection*{Beautifier options}

The code generator tries to output shortest code possible by default. In case you want beautified output, pass {\ttfamily -\/-\/beautify} ({\ttfamily -\/b}). Optionally you can pass additional arguments that control the code output\+:


\begin{DoxyItemize}
\item {\ttfamily beautify} (default {\ttfamily true}) -- whether to actually beautify the output. Passing {\ttfamily -\/b} will set this to true, but you might need to pass {\ttfamily -\/b} even when you want to generate minified code, in order to specify additional arguments, so you can use {\ttfamily -\/b beautify=false} to override it.
\item {\ttfamily indent-\/level} (default 4)
\item {\ttfamily indent-\/start} (default 0) -- prefix all lines by that many spaces
\item {\ttfamily quote-\/keys} (default {\ttfamily false}) -- pass {\ttfamily true} to quote all keys in literal objects
\item {\ttfamily space-\/colon} (default {\ttfamily true}) -- insert a space after the colon signs
\item {\ttfamily ascii-\/only} (default {\ttfamily false}) -- escape Unicode characters in strings and regexps (affects directives with non-\/ascii characters becoming invalid)
\item {\ttfamily inline-\/script} (default {\ttfamily false}) -- escape the slash in occurrences of {\ttfamily $<$/script} in strings
\item {\ttfamily width} (default 80) -- only takes effect when beautification is on, this specifies an (orientative) line width that the beautifier will try to obey. It refers to the width of the line text (excluding indentation). It doesn\textquotesingle{}t work very well currently, but it does make the code generated by Uglify\+JS more readable.
\item {\ttfamily max-\/line-\/len} (default 32000) -- maximum line length (for uglified code)
\item {\ttfamily bracketize} (default {\ttfamily false}) -- always insert brackets in {\ttfamily if}, {\ttfamily for}, {\ttfamily do}, {\ttfamily while} or {\ttfamily with} statements, even if their body is a single statement.
\item {\ttfamily semicolons} (default {\ttfamily true}) -- separate statements with semicolons. If you pass {\ttfamily false} then whenever possible we will use a newline instead of a semicolon, leading to more readable output of uglified code (size before gzip could be smaller; size after gzip insignificantly larger).
\item {\ttfamily preamble} (default {\ttfamily null}) -- when passed it must be a string and it will be prepended to the output literally. The source map will adjust for this text. Can be used to insert a comment containing licensing information, for example.
\item {\ttfamily quote\+\_\+style} (default {\ttfamily 0}) -- preferred quote style for strings (affects quoted property names and directives as well)\+:
\begin{DoxyItemize}
\item {\ttfamily 0} -- prefers double quotes, switches to single quotes when there are more double quotes in the string itself.
\item {\ttfamily 1} -- always use single quotes
\item {\ttfamily 2} -- always use double quotes
\item {\ttfamily 3} -- always use the original quotes
\end{DoxyItemize}
\item {\ttfamily keep\+\_\+quoted\+\_\+props} (default {\ttfamily false}) -- when turned on, prevents stripping quotes from property names in object literals.
\end{DoxyItemize}

\subsubsection*{Keeping copyright notices or other comments}

You can pass {\ttfamily -\/-\/comments} to retain certain comments in the output. By default it will keep J\+S\+Doc-\/style comments that contain \char`\"{}@preserve\char`\"{}, \char`\"{}@license\char`\"{} or \char`\"{}@cc\+\_\+on\char`\"{} (conditional compilation for IE). You can pass {\ttfamily -\/-\/comments all} to keep all the comments, or a valid Java\+Script regexp to keep only comments that match this regexp. For example `--comments \textquotesingle{}/foo$\vert$bar/\textquotesingle{}` will keep only comments that contain \char`\"{}foo\char`\"{} or \char`\"{}bar\char`\"{}.

Note, however, that there might be situations where comments are lost. For example\+: 
\begin{DoxyCode}
function f() \{
    /** @preserve Foo Bar */
    function g() \{
      // this function is never called
    \}
    return something();
\}
\end{DoxyCode}


Even though it has \char`\"{}@preserve\char`\"{}, the comment will be lost because the inner function {\ttfamily g} (which is the A\+ST node to which the comment is attached to) is discarded by the compressor as not referenced.

The safest comments where to place copyright information (or other info that needs to be kept in the output) are comments attached to toplevel nodes.

\subsection*{Support for the Spider\+Monkey A\+ST}

Uglify\+J\+S2 has its own abstract syntax tree format; for \href{http://lisperator.net/blog/uglifyjs-why-not-switching-to-spidermonkey-ast/}{\tt practical reasons} we can\textquotesingle{}t easily change to using the Spider\+Monkey A\+ST internally. However, Uglify\+JS now has a converter which can import a Spider\+Monkey A\+ST.

For example \href{https://github.com/ternjs/acorn}{\tt Acorn} is a super-\/fast parser that produces a Spider\+Monkey A\+ST. It has a small C\+LI utility that parses one file and dumps the A\+ST in J\+S\+ON on the standard output. To use Uglify\+JS to mangle and compress that\+: \begin{DoxyVerb}acorn file.js | uglifyjs --spidermonkey -m -c
\end{DoxyVerb}


The {\ttfamily -\/-\/spidermonkey} option tells Uglify\+JS that all input files are not Java\+Script, but JS code described in Spider\+Monkey A\+ST in J\+S\+ON. Therefore we don\textquotesingle{}t use our own parser in this case, but just transform that A\+ST into our internal A\+ST.

\subsubsection*{Use Acorn for parsing}

More for fun, I added the {\ttfamily -\/-\/acorn} option which will use Acorn to do all the parsing. If you pass this option, Uglify\+JS will {\ttfamily require(\char`\"{}acorn\char`\"{})}.

Acorn is really fast (e.\+g. 250ms instead of 380ms on some 650K code), but converting the Spider\+Monkey tree that Acorn produces takes another 150ms so in total it\textquotesingle{}s a bit more than just using Uglify\+JS\textquotesingle{}s own parser.

\subsubsection*{Using Uglify\+JS to transform Spider\+Monkey A\+ST}

Now you can use Uglify\+JS as any other intermediate tool for transforming Java\+Script A\+S\+Ts in Spider\+Monkey format.

Example\+:


\begin{DoxyCode}
function uglify(ast, options, mangle) \{
  // Conversion from SpiderMonkey AST to internal format
  var uAST = UglifyJS.AST\_Node.from\_mozilla\_ast(ast);

  // Compression
  uAST.figure\_out\_scope();
  uAST = uAST.transform(UglifyJS.Compressor(options));

  // Mangling (optional)
  if (mangle) \{
    uAST.figure\_out\_scope();
    uAST.compute\_char\_frequency();
    uAST.mangle\_names();
  \}

  // Back-conversion to SpiderMonkey AST
  return uAST.to\_mozilla\_ast();
\}
\end{DoxyCode}


Check out \href{http://rreverser.com/using-mozilla-ast-with-uglifyjs/}{\tt original blog post} for details.

\subsection*{A\+PI Reference }

Assuming installation via N\+PM, you can load Uglify\+JS in your application like this\+: 
\begin{DoxyCode}
var UglifyJS = require("uglify-js");
\end{DoxyCode}


It exports a lot of names, but I\textquotesingle{}ll discuss here the basics that are needed for parsing, mangling and compressing a piece of code. The sequence is (1) parse, (2) compress, (3) mangle, (4) generate output code.

\subsubsection*{The simple way}

There\textquotesingle{}s a single toplevel function which combines all the steps. If you don\textquotesingle{}t need additional customization, you might want to go with {\ttfamily minify}. Example\+: 
\begin{DoxyCode}
var result = UglifyJS.minify("/path/to/file.js");
console.log(result.code); // minified output
// if you need to pass code instead of file name
var result = UglifyJS.minify("var b = function () \{\};", \{fromString: true\});
\end{DoxyCode}


You can also compress multiple files\+: 
\begin{DoxyCode}
var result = UglifyJS.minify([ "file1.js", "file2.js", "file3.js" ]);
console.log(result.code);
\end{DoxyCode}


To generate a source map\+: 
\begin{DoxyCode}
var result = UglifyJS.minify([ "file1.js", "file2.js", "file3.js" ], \{
    outSourceMap: "out.js.map"
\});
console.log(result.code); // minified output
console.log(result.map);
\end{DoxyCode}


Note that the source map is not saved in a file, it\textquotesingle{}s just returned in {\ttfamily result.\+map}. The value passed for {\ttfamily out\+Source\+Map} is only used to set the {\ttfamily file} attribute in the source map (see \href{https://docs.google.com/document/d/1U1RGAehQwRypUTovF1KRlpiOFze0b-_2gc6fAH0KY0k/edit}{\tt the spec}).

You can also specify source\+Root property to be included in source map\+: 
\begin{DoxyCode}
var result = UglifyJS.minify([ "file1.js", "file2.js", "file3.js" ], \{
    outSourceMap: "out.js.map",
    sourceRoot: "http://example.com/src"
\});
\end{DoxyCode}


If you\textquotesingle{}re compressing compiled Java\+Script and have a source map for it, you can use the {\ttfamily in\+Source\+Map} argument\+: ```javascript var result = Uglify\+J\+S.\+minify(\char`\"{}compiled.\+js\char`\"{}, \{ in\+Source\+Map\+: \char`\"{}compiled.\+js.\+map\char`\"{}, out\+Source\+Map\+: \char`\"{}minified.\+js.\+map\char`\"{} \}); // same as before, it returns {\ttfamily code} and {\ttfamily map} 
\begin{DoxyCode}
If your input source map is not in a file, you can pass it in as an object
using the `inSourceMap` argument:

```javascript
var result = UglifyJS.minify("compiled.js", \{
    inSourceMap: JSON.parse(my\_source\_map\_string),
    outSourceMap: "minified.js.map"
\});
\end{DoxyCode}


The {\ttfamily in\+Source\+Map} is only used if you also request {\ttfamily out\+Source\+Map} (it makes no sense otherwise).

Other options\+:


\begin{DoxyItemize}
\item {\ttfamily warnings} (default {\ttfamily false}) — pass {\ttfamily true} to display compressor warnings.
\item {\ttfamily from\+String} (default {\ttfamily false}) — if you pass {\ttfamily true} then you can pass Java\+Script source code, rather than file names.
\item {\ttfamily mangle} — pass {\ttfamily false} to skip mangling names.
\item {\ttfamily mangle\+Properties} (default {\ttfamily false}) — pass an object to specify custom mangle property options.
\item {\ttfamily output} (default {\ttfamily null}) — pass an object if you wish to specify additional \href{http://lisperator.net/uglifyjs/codegen}{\tt output options}. The defaults are optimized for best compression.
\item {\ttfamily compress} (default {\ttfamily \{\}}) — pass {\ttfamily false} to skip compressing entirely. Pass an object to specify custom \href{http://lisperator.net/uglifyjs/compress}{\tt compressor options}.
\item {\ttfamily parse} (default \{\}) — pass an object if you wish to specify some additional \href{http://lisperator.net/uglifyjs/parser}{\tt parser options}. (not all options available... see below)
\end{DoxyItemize}

\subparagraph*{mangle}


\begin{DoxyItemize}
\item {\ttfamily except} -\/ pass an array of identifiers that should be excluded from mangling
\end{DoxyItemize}

\subparagraph*{mangle\+Properties options}


\begin{DoxyItemize}
\item {\ttfamily regex} — Pass a Reg\+Exp to only mangle certain names (maps to the {\ttfamily -\/-\/mangle-\/regex} C\+LI arguments option)
\item {\ttfamily ignore\+\_\+quoted} – Only mangle unquoted property names (maps to the {\ttfamily -\/-\/mangle-\/props 2} C\+LI arguments option)
\end{DoxyItemize}

We could add more options to {\ttfamily Uglify\+J\+S.\+minify} — if you need additional functionality please suggest!

\subsubsection*{The hard way}

Following there\textquotesingle{}s more detailed A\+PI info, in case the {\ttfamily minify} function is too simple for your needs.

\#\#\#\# The parser 
\begin{DoxyCode}
var toplevel\_ast = UglifyJS.parse(code, options);
\end{DoxyCode}


{\ttfamily options} is optional and if present it must be an object. The following properties are available\+:


\begin{DoxyItemize}
\item {\ttfamily strict} — disable automatic semicolon insertion and support for trailing comma in arrays and objects
\item {\ttfamily bare\+\_\+returns} — Allow return outside of functions. (maps to the {\ttfamily -\/-\/bare-\/returns} C\+LI arguments option and available to {\ttfamily minify} {\ttfamily parse} other options object)
\item {\ttfamily filename} — the name of the file where this code is coming from
\item {\ttfamily toplevel} — a {\ttfamily toplevel} node (as returned by a previous invocation of {\ttfamily parse})
\end{DoxyItemize}

The last two options are useful when you\textquotesingle{}d like to minify multiple files and get a single file as the output and a proper source map. Our C\+LI tool does something like this\+: 
\begin{DoxyCode}
var toplevel = null;
files.forEach(function(file)\{
    var code = fs.readFileSync(file, "utf8");
    toplevel = UglifyJS.parse(code, \{
        filename: file,
        toplevel: toplevel
    \});
\});
\end{DoxyCode}


After this, we have in {\ttfamily toplevel} a big A\+ST containing all our files, with each token having proper information about where it came from.

\paragraph*{Scope information}

Uglify\+JS contains a scope analyzer that you need to call manually before compressing or mangling. Basically it augments various nodes in the A\+ST with information about where is a name defined, how many times is a name referenced, if it is a global or not, if a function is using {\ttfamily eval} or the {\ttfamily with} statement etc. I will discuss this some place else, for now what\textquotesingle{}s important to know is that you need to call the following before doing anything with the tree\+: 
\begin{DoxyCode}
toplevel.figure\_out\_scope()
\end{DoxyCode}


\paragraph*{Compression}

Like this\+: 
\begin{DoxyCode}
var compressor = UglifyJS.Compressor(options);
var compressed\_ast = toplevel.transform(compressor);
\end{DoxyCode}


The {\ttfamily options} can be missing. Available options are discussed above in “\+Compressor options”. Defaults should lead to best compression in most scripts.

The compressor is destructive, so don\textquotesingle{}t rely that {\ttfamily toplevel} remains the original tree.

\paragraph*{Mangling}

After compression it is a good idea to call again {\ttfamily figure\+\_\+out\+\_\+scope} (since the compressor might drop unused variables / unreachable code and this might change the number of identifiers or their position). Optionally, you can call a trick that helps after Gzip (counting character frequency in non-\/mangleable words). Example\+: 
\begin{DoxyCode}
compressed\_ast.figure\_out\_scope();
compressed\_ast.compute\_char\_frequency();
compressed\_ast.mangle\_names();
\end{DoxyCode}


\paragraph*{Generating output}

A\+ST nodes have a {\ttfamily print} method that takes an output stream. Essentially, to generate code you do this\+: 
\begin{DoxyCode}
var stream = UglifyJS.OutputStream(options);
compressed\_ast.print(stream);
var code = stream.toString(); // this is your minified code
\end{DoxyCode}


or, for a shortcut you can do\+: 
\begin{DoxyCode}
var code = compressed\_ast.print\_to\_string(options);
\end{DoxyCode}


As usual, {\ttfamily options} is optional. The output stream accepts a lot of options, most of them documented above in section “\+Beautifier options”. The two which we care about here are {\ttfamily source\+\_\+map} and {\ttfamily comments}.

\paragraph*{Keeping comments in the output}

In order to keep certain comments in the output you need to pass the {\ttfamily comments} option. Pass a Reg\+Exp or a function. If you pass a Reg\+Exp, only those comments whose body matches the regexp will be kept. Note that body means without the initial {\ttfamily //} or {\ttfamily /$\ast$}. If you pass a function, it will be called for every comment in the tree and will receive two arguments\+: the node that the comment is attached to, and the comment token itself.

The comment token has these properties\+:


\begin{DoxyItemize}
\item {\ttfamily type}\+: \char`\"{}comment1\char`\"{} for single-\/line comments or \char`\"{}comment2\char`\"{} for multi-\/line comments
\item {\ttfamily value}\+: the comment body
\item {\ttfamily pos} and {\ttfamily endpos}\+: the start/end positions (zero-\/based indexes) in the original code where this comment appears
\item {\ttfamily line} and {\ttfamily col}\+: the line and column where this comment appears in the original code
\item {\ttfamily file} — the file name of the original file
\item {\ttfamily nlb} — true if there was a newline before this comment in the original code, or if this comment contains a newline.
\end{DoxyItemize}

Your function should return {\ttfamily true} to keep the comment, or a falsy value otherwise.

\paragraph*{Generating a source mapping}

You need to pass the {\ttfamily source\+\_\+map} argument when calling {\ttfamily print}. It needs to be a {\ttfamily Source\+Map} object (which is a thin wrapper on top of the \href{https://github.com/mozilla/source-map}{\tt source-\/map} library).

Example\+: 
\begin{DoxyCode}
var source\_map = UglifyJS.SourceMap(source\_map\_options);
var stream = UglifyJS.OutputStream(\{
    ...
    source\_map: source\_map
\});
compressed\_ast.print(stream);

var code = stream.toString();
var map = source\_map.toString(); // json output for your source map
\end{DoxyCode}


The {\ttfamily source\+\_\+map\+\_\+options} (optional) can contain the following properties\+:


\begin{DoxyItemize}
\item {\ttfamily file}\+: the name of the Java\+Script output file that this mapping refers to
\item {\ttfamily root}\+: the {\ttfamily source\+Root} property (see the \href{https://docs.google.com/document/d/1U1RGAehQwRypUTovF1KRlpiOFze0b-_2gc6fAH0KY0k/edit}{\tt spec})
\item {\ttfamily orig}\+: the \char`\"{}original source map\char`\"{}, handy when you compress generated JS and want to map the minified output back to the original code where it came from. It can be simply a string in J\+S\+ON, or a J\+S\+ON object containing the original source map. 
\end{DoxyItemize}