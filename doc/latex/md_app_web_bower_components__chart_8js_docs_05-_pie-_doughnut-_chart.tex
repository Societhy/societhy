

 title\+: Pie \& Doughnut Charts \subsection*{anchor\+: doughnut-\/pie-\/chart }

\subsubsection*{Introduction}

Pie and doughnut charts are probably the most commonly used chart there are. They are divided into segments, the arc of each segment shows the proportional value of each piece of data.

They are excellent at showing the relational proportions between data.

Pie and doughnut charts are effectively the same class in Chart.\+js, but have one different default value -\/ their {\ttfamily percentage\+Inner\+Cutout}. This equates what percentage of the inner should be cut out. This defaults to {\ttfamily 0} for pie charts, and {\ttfamily 50} for doughnuts.

They are also registered under two aliases in the {\ttfamily Chart} core. Other than their different default value, and different alias, they are exactly the same.

 $<$canvas width=\char`\"{}250\char`\"{} height=\char`\"{}125\char`\"{}$>$$<$/canvas$>$ 

 $<$canvas width=\char`\"{}250\char`\"{} height=\char`\"{}125\char`\"{}$>$$<$/canvas$>$ 

\subsubsection*{Example usage}


\begin{DoxyCode}
// For a pie chart
var myPieChart = new Chart(ctx[0]).Pie(data,options);

// And for a doughnut chart
var myDoughnutChart = new Chart(ctx[1]).Doughnut(data,options);
\end{DoxyCode}


\subsubsection*{Data structure}


\begin{DoxyCode}
var data = [
    \{
        value: 300,
        color:"#F7464A",
        highlight: "#FF5A5E",
        label: "Red"
    \},
    \{
        value: 50,
        color: "#46BFBD",
        highlight: "#5AD3D1",
        label: "Green"
    \},
    \{
        value: 100,
        color: "#FDB45C",
        highlight: "#FFC870",
        label: "Yellow"
    \}
]
\end{DoxyCode}


For a pie chart, you must pass in an array of objects with a value and a color property. The value attribute should be a number, Chart.\+js will total all of the numbers and calculate the relative proportion of each. The color attribute should be a string. Similar to C\+SS, for this string you can use H\+EX notation, R\+GB, R\+G\+BA or H\+SL.

\subsubsection*{Chart options}

These are the customisation options specific to Pie \& Doughnut charts. These options are merged with the \href{#getting-started-global-chart-configuration}{\tt global chart configuration options}, and form the options of the chart.


\begin{DoxyCode}
\{
    //Boolean - Whether we should show a stroke on each segment
    segmentShowStroke : true,

    //String - The colour of each segment stroke
    segmentStrokeColor : "#fff",

    //Number - The width of each segment stroke
    segmentStrokeWidth : 2,

    //Number - The percentage of the chart that we cut out of the middle
    percentageInnerCutout : 50, // This is 0 for Pie charts

    //Number - Amount of animation steps
    animationSteps : 100,

    //String - Animation easing effect
    animationEasing : "easeOutBounce",

    //Boolean - Whether we animate the rotation of the Doughnut
    animateRotate : true,

    //Boolean - Whether we animate scaling the Doughnut from the centre
    animateScale : false,
    \{% raw %\}
    //String - A legend template
    legendTemplate : "<ul class=\(\backslash\)"<%=name.toLowerCase()%>-legend\(\backslash\)"><% for (var i=0; i<segments.length;
       i++)\{%><li><span
       style=\(\backslash\)"background-color:<%=segments[i].fillColor%>\(\backslash\)"></span><%if(segments[i].label)\{%><%=segments[i].label%><%\}%></li><%\}%></ul>"
    \{% endraw %\}
\}
\end{DoxyCode}
 You can override these for your {\ttfamily Chart} instance by passing a second argument into the {\ttfamily Doughnut} method as an object with the keys you want to override.

For example, we could have a doughnut chart that animates by scaling out from the centre like so\+:


\begin{DoxyCode}
new Chart(ctx).Doughnut(data, \{
    animateScale: true
\});
// This will create a chart with all of the default options, merged from the global config,
// and the Doughnut chart defaults but this particular instance will have `animateScale` set to `true`.
\end{DoxyCode}


We can also change these default values for each Doughnut type that is created, this object is available at {\ttfamily Chart.\+defaults.\+Doughnut}. Pie charts also have a clone of these defaults available to change at {\ttfamily Chart.\+defaults.\+Pie}, with the only difference being {\ttfamily percentage\+Inner\+Cutout} being set to 0.

\subsubsection*{Prototype methods}

\paragraph*{.get\+Segments\+At\+Event( event )}

Calling {\ttfamily get\+Segments\+At\+Event(event)} on your Chart instance passing an argument of an event, or j\+Query event, will return the segment elements that are at the same position of that event.


\begin{DoxyCode}
canvas.onclick = function(evt)\{
    var activePoints = myDoughnutChart.getSegmentsAtEvent(evt);
    // => activePoints is an array of segments on the canvas that are at the same position as the click
       event.
\};
\end{DoxyCode}


This functionality may be useful for implementing D\+OM based tooltips, or triggering custom behaviour in your application.

\paragraph*{.update( )}

Calling {\ttfamily update()} on your Chart instance will re-\/render the chart with any updated values, allowing you to edit the value of multiple existing points, then render those in one animated render loop.


\begin{DoxyCode}
myDoughnutChart.segments[1].value = 10;
// Would update the first dataset's value of 'Green' to be 10
myDoughnutChart.update();
// Calling update now animates the circumference of the segment 'Green' from 50 to 10.
// and transitions other segment widths
\end{DoxyCode}


\paragraph*{.add\+Data( segment\+Data, index )}

Calling {\ttfamily add\+Data(segment\+Data, index)} on your Chart instance passing an object in the same format as in the constructor. There is an optional second argument of \textquotesingle{}index\textquotesingle{}, this determines at what index the new segment should be inserted into the chart.


\begin{DoxyCode}
// An object in the same format as the original data source
myDoughnutChart.addData(\{
    value: 130,
    color: "#B48EAD",
    highlight: "#C69CBE",
    label: "Purple"
\});
// The new segment will now animate in.
\end{DoxyCode}


\paragraph*{.remove\+Data( index )}

Calling {\ttfamily remove\+Data(index)} on your Chart instance will remove segment at that particular index. If none is provided, it will default to the last segment.


\begin{DoxyCode}
myDoughnutChart.removeData();
// Other segments will update to fill the empty space left.
\end{DoxyCode}
 