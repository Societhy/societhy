Monkey-\/patch (hook) functions for debugging and stuff.

\subsection*{Getting Started}

This code should work just fine in Node.\+js\+:

First, install the module with\+: {\ttfamily npm install hooker}


\begin{DoxyCode}
var hooker = require('hooker');
hooker.hook(Math, "max", function() \{
  console.log(arguments.length + " arguments passed");
\});
Math.max(5, 6, 7) // logs: "3 arguments passed", returns 7
\end{DoxyCode}


Or in the browser\+:


\begin{DoxyCode}
<script src="dist/ba-hooker.min.js"></script>
<script>
hook(Math, "max", function() \{
  console.log(arguments.length + " arguments passed");
\});
Math.max(5, 6, 7) // logs: "3 arguments passed", returns 7
</script>
\end{DoxyCode}


In the browser, you can attach Hooker\textquotesingle{}s methods to any object.


\begin{DoxyCode}
<script>
this.exports = Bocoup.utils;
</script>
<script src="dist/ba-hooker.min.js"></script>
<script>
Bocoup.utils.hook(Math, "max", function() \{
  console.log(arguments.length + " arguments passed");
\});
Math.max(5, 6, 7) // logs: "3 arguments passed", returns 7
</script>
\end{DoxyCode}


\subsection*{Documentation}

\subsubsection*{hooker.\+hook}

Monkey-\/patch (hook) one or more methods of an object. \paragraph*{Signature\+:}

{\ttfamily hooker.\+hook(object, \mbox{[} props, \mbox{]} \mbox{[}options $\vert$ prehook\+Function\mbox{]})} \paragraph*{{\ttfamily props}}

The optional {\ttfamily props} argument can be a method name, array of method names or null. If null (or omitted), all enumerable methods of {\ttfamily object} will be hooked. \paragraph*{{\ttfamily options}}


\begin{DoxyItemize}
\item {\ttfamily pre} -\/ (Function) a pre-\/hook function to be executed before the original function. Arguments passed into the method will be passed into the pre-\/hook function as well.
\item {\ttfamily post} -\/ (Function) a post-\/hook function to be executed after the original function. The original function\textquotesingle{}s result is passed into the post-\/hook function as its first argument, followed by the method arguments.
\item {\ttfamily once} -\/ (Boolean) if true, auto-\/unhook the function after the first execution.
\item {\ttfamily pass\+Name} -\/ (Boolean) if true, pass the name of the method into the pre-\/hook function as its first arg (preceding all other arguments), and into the post-\/hook function as the second arg (after result but preceding all other arguments).
\end{DoxyItemize}

\paragraph*{Returns\+:}

An array of hooked method names.

\subsubsection*{hooker.\+unhook}

Un-\/monkey-\/patch (unhook) one or more methods of an object. \paragraph*{Signature\+:}

{\ttfamily hooker.\+unhook(object \mbox{[}, props \mbox{]})} \paragraph*{{\ttfamily props}}

The optional {\ttfamily props} argument can be a method name, array of method names or null. If null (or omitted), all methods of {\ttfamily object} will be unhooked. \paragraph*{Returns\+:}

An array of unhooked method names.

\subsubsection*{hooker.\+orig}

Get a reference to the original method from a hooked function. \paragraph*{Signature\+:}

{\ttfamily hooker.\+orig(object, props)}

\subsubsection*{hooker.\+override}

When a pre-\/ or post-\/hook returns the result of this function, the value passed will be used in place of the original function\textquotesingle{}s return value. Any post-\/hook override value will take precedence over a pre-\/hook override value. \paragraph*{Signature\+:}

{\ttfamily hooker.\+override(value)}

\subsubsection*{hooker.\+preempt}

When a pre-\/hook returns the result of this function, the value passed will be used in place of the original function\textquotesingle{}s return value, and the original function will N\+OT be executed. \paragraph*{Signature\+:}

{\ttfamily hooker.\+preempt(value)}

\subsubsection*{hooker.\+filter}

When a pre-\/hook returns the result of this function, the context and arguments passed will be applied into the original function. \paragraph*{Signature\+:}

{\ttfamily hooker.\+filter(context, arguments)}

\subsection*{Examples}

See the unit tests for more examples.


\begin{DoxyCode}
var hooker = require('hooker');
// Simple logging.
hooker.hook(Math, "max", function() \{
  console.log(arguments.length + " arguments passed");
\});
Math.max(5, 6, 7) // logs: "3 arguments passed", returns 7

hooker.unhook(Math, "max"); // (This is assumed between all further examples)
Math.max(5, 6, 7) // 7

// Returning hooker.override(value) overrides the original value.
hooker.hook(Math, "max", function() \{
  if (arguments.length === 0) \{
    return hooker.override(9000);
  \}
\});
Math.max(5, 6, 7) // 7
Math.max() // 9000

// Auto-unhook after one execution.
hooker.hook(Math, "max", \{
  once: true,
  pre: function() \{
    console.log("Init something here");
  \}
\});
Math.max(5, 6, 7) // logs: "Init something here", returns 7
Math.max(5, 6, 7) // 7

// Filter `this` and arguments through a pre-hook function.
hooker.hook(Math, "max", \{
  pre: function() \{
    var args = [].map.call(arguments, function(num) \{
      return num * 2;
    \});
    return hooker.filter(this, args); // thisValue, arguments
  \}
\});
Math.max(5, 6, 7) // 14

// Modify the original function's result with a post-hook function.
hooker.hook(Math, "max", \{
  post: function(result) \{
    return hooker.override(result * 100);
  \}
\});
Math.max(5, 6, 7) // 700

// Hook every Math method. Note: if Math's methods were enumerable, the second
// argument could be omitted. Since they aren't, an array of properties to hook
// must be explicitly passed. Non-method properties will be skipped.
// See a more generic example here: http://bit.ly/vvJlrS
hooker.hook(Math, Object.getOwnPropertyNames(Math), \{
  passName: true,
  pre: function(name) \{
    console.log("=> Math." + name, [].slice.call(arguments, 1));
  \},
  post: function(result, name) \{
    console.log("<= Math." + name, result);
  \}
\});

var result = Math.max(5, 6, 7);
// => Math.max [ 5, 6, 7 ]
// <= Math.max 7
result // 7

result = Math.ceil(3.456);
// => Math.ceil [ 3.456 ]
// <= Math.ceil 4
result // 4
\end{DoxyCode}


\subsection*{Contributing}

In lieu of a formal styleguide, take care to maintain the existing coding style. Add unit tests for any new or changed functionality. Lint and test your code using \href{https://github.com/cowboy/grunt}{\tt grunt}.

{\itshape Also, please don\textquotesingle{}t edit files in the \char`\"{}dist\char`\"{} subdirectory as they are generated via grunt. You\textquotesingle{}ll find source code in the \char`\"{}lib\char`\"{} subdirectory!}

\subsection*{Release History}

2012/01/09 -\/ v0.\+2.\+3 -\/ First official release.

\subsection*{License}

Copyright (c) 2012 \char`\"{}\+Cowboy\char`\"{} Ben Alman Licensed under the M\+IT license. \href{http://benalman.com/about/license/}{\tt http\+://benalman.\+com/about/license/} 