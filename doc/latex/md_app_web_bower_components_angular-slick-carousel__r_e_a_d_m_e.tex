\href{https://gitter.im/devmark/angular-slick-carousel?utm_source=badge&utm_medium=badge&utm_campaign=pr-badge&utm_content=badge}{\tt }

Angular directive for \href{http://kenwheeler.github.io/slick/}{\tt slick-\/carousel}

\subsection*{Summary}


\begin{DoxyItemize}
\item \href{#usage}{\tt Usage}
\item \href{#attributes&event}{\tt Attributes \& Event}
\item \href{#enable/disableslick}{\tt Enable/disable slick}
\item \href{#method}{\tt Method}
\item \href{#slidedata}{\tt Slide data}
\item \href{#globalconfig}{\tt Global Config}
\item \href{#faq}{\tt faq}
\item \href{#examples}{\tt Examples}
\item \href{#creator}{\tt Creator}
\end{DoxyItemize}

\subsection*{\#\# Usage }


\begin{DoxyItemize}
\item Using \href{http://bower.io/}{\tt bower} to install it. {\ttfamily bower install angular-\/slick-\/carousel}
\item Add {\ttfamily jquery}, {\ttfamily angular}, {\ttfamily slick-\/carousel} and {\ttfamily angular-\/slick-\/carousel} to your code.
\end{DoxyItemize}


\begin{DoxyCode}
<link rel="stylesheet" href="../bower\_components/slick-carousel/slick/slick.css">
<link rel="stylesheet" href="../bower\_components/slick-carousel/slick/slick-theme.css">

<script src="../bower\_components/jquery/jquery.js"></script>
<script src="../bower\_components/angular/angular.js"></script>
<script src="../bower\_components/slick-carousel/slick/slick.js"></script>
<script src="../bower\_components/angular-slick-carousel/dist/angular-slick.min.js"></script>
\end{DoxyCode}



\begin{DoxyItemize}
\item Add the sortable module as a dependency to your application module\+: {\ttfamily slick\+Carousel}
\end{DoxyItemize}


\begin{DoxyCode}
var myAppModule = angular.module('MyApp', ['slickCarousel'])
\end{DoxyCode}


This directive allows you to use the slick-\/carousel plugin as an angular directive. It can be specified in your H\+T\+ML as either a {\ttfamily $<$div$>$} attribute or a {\ttfamily $<$slick$>$} element.


\begin{DoxyCode}
<slick infinite=true slides-to-show=3 slides-to-scroll=3>
...
</slick>

<slick 
    settings="slickConfig" ng-if="numberLoaded">
</slick>
\end{DoxyCode}


\subsection*{Attributes \& Event}

{\ttfamily settings}\+: optional {\ttfamily Object} containing any of the slick options. Consult \href{http://kenwheeler.github.io/slick/#settings}{\tt here}.
\begin{DoxyItemize}
\item {\ttfamily enabled} should slick be enabled or not. Default to true. Example below
\item {\ttfamily method} optional containing slick method. discussed \href{#method}{\tt below} in detail
\item {\ttfamily event} optional containing slick event
\end{DoxyItemize}


\begin{DoxyCode}
$scope.slickConfig = \{
    enabled: true,
    autoplay: true,
    draggable: false,  
    autoplaySpeed: 3000,
    method: \{\},
    event: \{
        beforeChange: function (event, slick, currentSlide, nextSlide) \{
        \},
        afterChange: function (event, slick, currentSlide, nextSlide) \{
        \}
    \}
\};
\end{DoxyCode}
 \subsection*{Enable/disable slick}

Slick can be easily switched on and off by using {\ttfamily enabled} settings flag. 
\begin{DoxyCode}
$scope.slickConfig = \{
    enabled: true,
\}
$scope.toggleSlick = function() \{
  $scope.slickConfig.enabled = !$scope.slickConfig.enabled;
\}
\end{DoxyCode}
 
\begin{DoxyCode}
<slick settings="slickConfig">
 ...
</slick>
<button ng-click="toggleSlick()">Toggle</button>
\end{DoxyCode}


\subsection*{Method}


\begin{DoxyEnumerate}
\item All the functions in the plugin are exposed through a control attribute.
\item To utilize this architecture, and have two-\/way data-\/binding, define an empty control handle on scope\+: 
\begin{DoxyCode}
$scope.slickConfig = \{
    method: \{\}
\}
\end{DoxyCode}

\item Pass it as the value to control attribute. Now, you can call any plugin methods as shown in the example.
\end{DoxyEnumerate}


\begin{DoxyCode}
<button ng-click="slickConfig.method.slickGoTo(2)">slickGoTo(2)</button>
<button ng-click="slickConfig.method.slickPrev()">slickPrev()</button>
<button ng-click="slickConfig.method.slickNext()">slickNext()</button>
<button ng-click='slickConfig.method.slickAdd("<div>New</div>")'>slickAdd()</button>
<button ng-click='slickConfig.method.slickRemove(3)'>slickRemove(3)</button>
<button ng-click='slickConfig.method.slickPlay()'>slickPlay()</button>
<button ng-click='slickConfig.method.slickPause()'>slickPause()</button>
\end{DoxyCode}


\subsection*{Slide data}

For change slide content, you have to set {\ttfamily ng-\/if} to destroy and init it


\begin{DoxyItemize}
\item controller\+: 
\begin{DoxyCode}
$scope.number = [\{label: 1\}, \{label: 2\}, \{label: 3\}, \{label: 4\}, \{label: 5\}, \{label: 6\}, \{label: 7\},
       \{label: 8\}];
$scope.numberLoaded = true;
$scope.numberUpdate = function()\{
    $scope.numberLoaded = false; // disable slick

    //number update

    $scope.numberLoaded = true; // enable slick
\};
\end{DoxyCode}

\item html\+: 
\begin{DoxyCode}
<script type="text/ng-template" id="tpl.html">
    <h3>\{\{ i.label \}\}</h3>
</script>

<slick ng-if="numberLoaded">
    <div ng-repeat="i in number">
        <div class="" ng-include="'tpl.html'"></div>
    </div>
</slick>
\end{DoxyCode}

\end{DoxyItemize}

\#\# \hyperlink{class_global}{Global} config 
\begin{DoxyCode}
config(['slickCarouselConfig', function (slickCarouselConfig) \{
    slickCarouselConfig.dots = true;
    slickCarouselConfig.autoplay = false;
\}])
\end{DoxyCode}


\subsection*{F\+AQ}

Q\+: After change data, could i keep the current slide index? A\+: For this directive, this will destroy and init slick when updating data. You could get current index by event. example\+: 
\begin{DoxyCode}
$scope.currentIndex = 0;
$scope.slickConfig = \{
    event: \{
        afterChange: function (event, slick, currentSlide, nextSlide) \{
          $scope.currentIndex = currentSlide; // save current index each time
        \}
        init: function (event, slick) \{
          slick.slickGoTo($scope.currentIndex); // slide to correct index when init
        \}
    \}
\};
\end{DoxyCode}


\subsection*{Examples}

You need be running a server to see the samples\+:

Go to your terminal and run\+:


\begin{DoxyCode}
python -m SimpleHTTPServer
\end{DoxyCode}
 after this command you will be loading a python Server in you local machine in most the cases loads in the port 8000, you will be able to see the port when the server starts like that\+: 
\begin{DoxyCode}
Serving HTTP on 0.0.0.0 port 8000 ...
\end{DoxyCode}


so you can see the samples with this adress\+: \href{http://localhost:8000/examples/#/}{\tt http\+://localhost\+:8000/examples/\#/}

\subsection*{Creator}