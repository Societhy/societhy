\href{https://www.npmjs.com/package/clean-css}{\tt } \href{https://travis-ci.org/jakubpawlowicz/clean-css}{\tt } \href{https://ci.appveyor.com/project/jakubpawlowicz/clean-css/branch/master}{\tt } \href{https://david-dm.org/jakubpawlowicz/clean-css}{\tt } \href{https://david-dm.org/jakubpawlowicz/clean-css#info=devDependencies}{\tt }

\subsection*{What is clean-\/css?}

Clean-\/css is a fast and efficient \href{http://nodejs.org/}{\tt Node.\+js} library for minifying C\+SS files.

According to \href{http://goalsmashers.github.io/css-minification-benchmark/}{\tt tests} it is one of the best available.

\subsection*{Usage}

\subsubsection*{What are the requirements?}


\begin{DoxyCode}
Node.js 0.10+ (tested on CentOS, Ubuntu, OS X 10.6+, and Windows 7+) or io.js 3.0+
\end{DoxyCode}


\subsubsection*{How to install clean-\/css?}


\begin{DoxyCode}
npm install clean-css
\end{DoxyCode}


\subsubsection*{How to use clean-\/css C\+LI?}

Clean-\/css accepts the following command line arguments (please make sure you use {\ttfamily $<$source-\/file$>$} as the very last argument to avoid potential issues)\+:


\begin{DoxyCode}
cleancss [options] source-file, [source-file, ...]

-h, --help                     output usage information
-v, --version                  output the version number
-b, --keep-line-breaks         Keep line breaks
-c, --compatibility [ie7|ie8]  Force compatibility mode (see Readme for advanced examples)
-d, --debug                    Shows debug information (minification time & compression efficiency)
-o, --output [output-file]     Use [output-file] as output instead of STDOUT
-r, --root [root-path]         Set a root path to which resolve absolute @import rules
-s, --skip-import              Disable @import processing
-t, --timeout [seconds]        Per connection timeout when fetching remote @imports (defaults to 5 seconds)
--rounding-precision [n]       Rounds to `N` decimal places. Defaults to 2. -1 disables rounding
--s0                           Remove all special comments, i.e. /*! comment */
--s1                           Remove all special comments but the first one
--semantic-merging             Enables unsafe mode by assuming BEM-like semantic stylesheets (warning, this
       may break your styling!)
--skip-advanced                Disable advanced optimizations - ruleset reordering & merging
--skip-aggressive-merging      Disable properties merging based on their order
--skip-import-from [rules]     Disable @import processing for specified rules
--skip-media-merging           Disable @media merging
--skip-rebase                  Disable URLs rebasing
--skip-restructuring           Disable restructuring optimizations
--skip-shorthand-compacting    Disable shorthand compacting
--source-map                   Enables building input's source map
--source-map-inline-sources    Enables inlining sources inside source maps
\end{DoxyCode}


\paragraph*{Examples\+:}

To minify a {\bfseries public.\+css} file into {\bfseries public-\/min.\+css} do\+:


\begin{DoxyCode}
cleancss -o public-min.css public.css
\end{DoxyCode}


To minify the same {\bfseries public.\+css} into the standard output skip the {\ttfamily -\/o} parameter\+:


\begin{DoxyCode}
cleancss public.css
\end{DoxyCode}


More likely you would like to concatenate a couple of files. If you are on a Unix-\/like system\+:


\begin{DoxyCode}
cat one.css two.css three.css | cleancss -o merged-and-minified.css
\end{DoxyCode}


On Windows\+:


\begin{DoxyCode}
type one.css two.css three.css | cleancss -o merged-and-minified.css
\end{DoxyCode}


Or even gzip the result at once\+:


\begin{DoxyCode}
cat one.css two.css three.css | cleancss | gzip -9 -c > merged-minified-and-gzipped.css.gz
\end{DoxyCode}


\subsubsection*{How to use clean-\/css A\+PI?}


\begin{DoxyCode}
var CleanCSS = require('clean-css');
var source = 'a\{font-weight:bold;\}';
var minified = new CleanCSS().minify(source).styles;
\end{DoxyCode}


Clean\+C\+SS constructor accepts a hash as a parameter, i.\+e., {\ttfamily new Clean\+C\+S\+S(options)} with the following options available\+:


\begin{DoxyItemize}
\item {\ttfamily advanced} -\/ set to false to disable advanced optimizations -\/ selector \& property merging, reduction, etc.
\item {\ttfamily aggressive\+Merging} -\/ set to false to disable aggressive merging of properties.
\item {\ttfamily benchmark} -\/ turns on benchmarking mode measuring time spent on cleaning up (run {\ttfamily npm run bench} to see example)
\item {\ttfamily compatibility} -\/ enables compatibility mode, see \href{#how-to-set-a-compatibility-mode}{\tt below for more examples}
\item {\ttfamily debug} -\/ set to true to get minification statistics under {\ttfamily stats} property (see {\ttfamily test/custom-\/test.\+js} for examples)
\item {\ttfamily inliner} -\/ a hash of options for {\ttfamily @import} inliner, see \href{https://github.com/jakubpawlowicz/clean-css/blob/master/test/protocol-imports-test.js#L372}{\tt test/protocol-\/imports-\/test.\+js} for examples, or \href{https://github.com/jakubpawlowicz/clean-css/issues/612#issuecomment-119594185}{\tt this comment} for a proxy use case.
\item {\ttfamily keep\+Breaks} -\/ whether to keep line breaks (default is false)
\item {\ttfamily keep\+Special\+Comments} -\/ {\ttfamily $\ast$} for keeping all (default), {\ttfamily 1} for keeping first one only, {\ttfamily 0} for removing all
\item {\ttfamily media\+Merging} -\/ whether to merge {\ttfamily @media} at-\/rules (default is true)
\item {\ttfamily process\+Import} -\/ whether to process {\ttfamily @import} rules
\item {\ttfamily process\+Import\+From} -\/ a list of {\ttfamily @import} rules, can be `\mbox{[}\textquotesingle{}all\textquotesingle{}\mbox{]}{\ttfamily (default),}\mbox{[}\textquotesingle{}local\textquotesingle{}\mbox{]}{\ttfamily ,}\mbox{[}\textquotesingle{}remote\textquotesingle{}\mbox{]}{\ttfamily , or a blacklisted path e.\+g.}\mbox{[}\textquotesingle{}!fonts.googleapis.\+com\textquotesingle{}\mbox{]}{\ttfamily  $\ast$}rebase{\ttfamily -\/ set to false to skip U\+RL rebasing $\ast$}relative\+To{\ttfamily -\/ path to $\ast$$\ast$resolve$\ast$$\ast$ relative}{\ttfamily rules and U\+R\+Ls $\ast$}restructuring{\ttfamily -\/ set to false to disable restructuring in advanced optimizations $\ast$}root{\ttfamily -\/ path to $\ast$$\ast$resolve$\ast$$\ast$ absolute}{\ttfamily rules and $\ast$$\ast$rebase$\ast$$\ast$ relative U\+R\+Ls $\ast$}rounding\+Precision{\ttfamily -\/ rounding precision; defaults to}2{\ttfamily ;}-\/1{\ttfamily disables rounding $\ast$}semantic\+Merging{\ttfamily -\/ set to true to enable semantic merging mode which assumes B\+E\+M-\/like content (default is false as it\textquotesingle{}s highly likely this will break your stylesheets -\/ $\ast$$\ast$use with caution$\ast$$\ast$!) $\ast$}shorthand\+Compacting{\ttfamily -\/ set to false to skip shorthand compacting (default is true unless source\+Map is set when it\textquotesingle{}s false) $\ast$}source\+Map{\ttfamily -\/ exposes source map under}source\+Map{\ttfamily property, e.\+g.}new Clean\+C\+S\+S().minify(source).source\+Map` (default is false) If input styles are a product of C\+SS preprocessor (Less, Sass) an input source map can be passed as a string.
\item {\ttfamily source\+Map\+Inline\+Sources} -\/ set to true to inline sources inside a source map\textquotesingle{}s {\ttfamily sources\+Content} field (defaults to false) It is also required to process inlined sources from input source maps.
\item {\ttfamily target} -\/ path to a folder or an output file to which {\bfseries rebase} all U\+R\+Ls
\end{DoxyItemize}

The output of {\ttfamily minify} method (or the 2nd argument to passed callback) is a hash containing the following fields\+:


\begin{DoxyItemize}
\item {\ttfamily styles} -\/ optimized output C\+SS as a string
\item {\ttfamily source\+Map} -\/ output source map (if requested with {\ttfamily source\+Map} option)
\item {\ttfamily errors} -\/ a list of errors raised
\item {\ttfamily warnings} -\/ a list of warnings raised
\item {\ttfamily stats} -\/ a hash of statistic information (if requested with {\ttfamily debug} option)\+:
\begin{DoxyItemize}
\item {\ttfamily original\+Size} -\/ original content size (after import inlining)
\item {\ttfamily minified\+Size} -\/ optimized content size
\item {\ttfamily time\+Spent} -\/ time spent on optimizations
\item {\ttfamily efficiency} -\/ a ratio of output size to input size (e.\+g. 25\% if content was reduced from 100 bytes to 75 bytes)
\end{DoxyItemize}
\end{DoxyItemize}

\paragraph*{How to make sure remote {\ttfamily @import}s are processed correctly?}

In order to inline remote {\ttfamily @import} statements you need to provide a callback to minify method, e.\+g.\+:


\begin{DoxyCode}
var CleanCSS = require('clean-css');
var source = '@import url(http://path/to/remote/styles);';
new CleanCSS().minify(source, function (errors, minified) \{
  // minified.styles
\});
\end{DoxyCode}


This is due to a fact, that, while local files can be read synchronously, remote resources can only be processed asynchronously. If you don\textquotesingle{}t provide a callback, then remote {\ttfamily @import}s will be left intact.

\subsubsection*{How to use clean-\/css with build tools?}


\begin{DoxyItemize}
\item \href{https://github.com/broccolijs/broccoli#broccoli}{\tt Broccoli}\+: \href{https://github.com/shinnn/broccoli-clean-css}{\tt broccoli-\/clean-\/css}
\item \href{http://brunch.io/}{\tt Brunch}\+: \href{https://github.com/brunch/clean-css-brunch}{\tt clean-\/css-\/brunch}
\item \href{http://gruntjs.com}{\tt Grunt}\+: \href{https://github.com/gruntjs/grunt-contrib-cssmin}{\tt grunt-\/contrib-\/cssmin}
\item \href{http://gulpjs.com/}{\tt Gulp}\+: \href{https://github.com/jonathanepollack/gulp-minify-css}{\tt gulp-\/minify-\/css}
\item \href{http://gulpjs.com/}{\tt Gulp}\+: \href{https://github.com/jakubpawlowicz/clean-css/issues/342}{\tt using vinyl-\/map as a wrapper -\/ courtesy of }
\item \href{https://github.com/component/builder2.js}{\tt component-\/builder2}\+: \href{https://github.com/poying/builder-clean-css}{\tt builder-\/clean-\/css}
\item \href{http://metalsmith.io}{\tt Metalsmith}\+: \href{https://github.com/aymericbeaumet/metalsmith-clean-css}{\tt metalsmith-\/clean-\/css}
\item \href{https://github.com/lasso-js/lasso}{\tt Lasso}\+: \href{https://github.com/yomed/lasso-clean-css}{\tt lasso-\/clean-\/css}
\end{DoxyItemize}

\subsubsection*{What are the clean-\/css\textquotesingle{} dev commands?}

First clone the source, then run\+:


\begin{DoxyItemize}
\item {\ttfamily npm run bench} for clean-\/css benchmarks (see \href{https://github.com/jakubpawlowicz/clean-css/blob/master/test/bench.js}{\tt test/bench.\+js} for details)
\item {\ttfamily npm run browserify} to create the browser-\/ready clean-\/css version
\item {\ttfamily npm run check} to check JS sources with \href{https://github.com/jshint/jshint/}{\tt J\+S\+Hint}
\item {\ttfamily npm test} for the test suite
\end{DoxyItemize}

\subsection*{How to contribute to clean-\/css?}

See https\+://github.com/jakubpawlowicz/clean-\/css/blob/master/\+C\+O\+N\+T\+R\+I\+B\+U\+T\+I\+N\+G.\+md \char`\"{}\+C\+O\+N\+T\+R\+I\+B\+U\+T\+I\+N\+G.\+md\char`\"{}.

\subsection*{Tips \& Tricks}

\subsubsection*{How to preserve a comment block?}

Use the {\ttfamily /$\ast$!} notation instead of the standard one {\ttfamily /$\ast$}\+:


\begin{DoxyCode}
/*!
  Important comments included in minified output.
*/
\end{DoxyCode}


\subsubsection*{How to rebase relative image U\+R\+Ls?}

Clean-\/css will handle it automatically for you (since version 1.\+1) in the following cases\+:


\begin{DoxyItemize}
\item When using the C\+LI\+:
\begin{DoxyEnumerate}
\item Use an output path via {\ttfamily -\/o}/{\ttfamily -\/-\/output} to rebase U\+R\+Ls as relative to the output file.
\item Use a root path via {\ttfamily -\/r}/{\ttfamily -\/-\/root} to rebase U\+R\+Ls as absolute from the given root path.
\item If you specify both then {\ttfamily -\/r}/{\ttfamily -\/-\/root} takes precendence.
\end{DoxyEnumerate}
\item When using clean-\/css as a library\+:
\begin{DoxyEnumerate}
\item Use a combination of {\ttfamily relative\+To} and {\ttfamily target} options for relative rebase (same as 1 in C\+LI).
\item Use a combination of {\ttfamily relative\+To} and {\ttfamily root} options for absolute rebase (same as 2 in C\+LI).
\item {\ttfamily root} takes precendence over {\ttfamily target} as in C\+LI.
\end{DoxyEnumerate}
\end{DoxyItemize}

\subsubsection*{How to generate source maps?}

Source maps are supported since version 3.\+0.

Additionally to mapping original C\+SS files, clean-\/css also supports input source maps, so minified styles can be mapped into their \href{http://lesscss.org/}{\tt Less} or \href{http://sass-lang.com/}{\tt Sass} sources directly.

Source maps are generated using \href{https://github.com/mozilla/source-map/}{\tt source-\/map} module from Mozilla.

\paragraph*{Using C\+LI}

To generate a source map, use {\ttfamily -\/-\/source-\/map} switch, e.\+g.\+:


\begin{DoxyCode}
cleancss --source-map --output styles.min.css styles.css
\end{DoxyCode}


Name of the output file is required, so a map file, named by adding {\ttfamily .map} suffix to output file name, can be created (styles.\+min.\+css.\+map in this case).

\paragraph*{Using A\+PI}

To generate a source map, use {\ttfamily source\+Map\+: true} option, e.\+g.\+:


\begin{DoxyCode}
new CleanCSS(\{ sourceMap: true, target: pathToOutputDirectory \})
  .minify(source, function (minified) \{
    // access minified.sourceMap for SourceMapGenerator object
    // see https://github.com/mozilla/source-map/#sourcemapgenerator for more details
    // see https://github.com/jakubpawlowicz/clean-css/blob/master/bin/cleancss#L114 on how it's used in
       clean-css' CLI
\});
\end{DoxyCode}


Using A\+PI you can also pass an input source map directly\+:


\begin{DoxyCode}
new CleanCSS(\{ sourceMap: inputSourceMapAsString, target: pathToOutputDirectory \})
  .minify(source, function (minified) \{
    // access minified.sourceMap to access SourceMapGenerator object
    // see https://github.com/mozilla/source-map/#sourcemapgenerator for more details
    // see https://github.com/jakubpawlowicz/clean-css/blob/master/bin/cleancss#L114 on how it's used in
       clean-css' CLI
\});
\end{DoxyCode}


Or even multiple input source maps at once (available since version 3.\+1)\+:


\begin{DoxyCode}
new CleanCSS(\{ sourceMap: true, target: pathToOutputDirectory \}).minify(\{
  'path/to/source/1': \{
    styles: '...styles...',
    sourceMap: '...source-map...'
  \},
  'path/to/source/2': \{
    styles: '...styles...',
    sourceMap: '...source-map...'
  \}
\}, function (minified) \{
  // access minified.sourceMap as above
\});
\end{DoxyCode}


\subsubsection*{How to minify multiple files with A\+PI?}

\paragraph*{Passing an array}


\begin{DoxyCode}
new CleanCSS().minify(['path/to/file/one', 'path/to/file/two']);
\end{DoxyCode}


\paragraph*{Passing a hash}


\begin{DoxyCode}
new CleanCSS().minify(\{
  'path/to/file/one': \{
    styles: 'contents of file one'
  \},
  'path/to/file/two': \{
    styles: 'contents of file two'
  \}
\});
\end{DoxyCode}


\subsubsection*{How to set a compatibility mode?}

Compatibility settings are controlled by {\ttfamily -\/-\/compatibility} switch (C\+LI) and {\ttfamily compatibility} option (library mode).

In both modes the following values are allowed\+:


\begin{DoxyItemize}
\item {\ttfamily \textquotesingle{}ie7\textquotesingle{}} -\/ Internet Explorer 7 compatibility mode
\item {\ttfamily \textquotesingle{}ie8\textquotesingle{}} -\/ Internet Explorer 8 compatibility mode
\item `\textquotesingle{}\textquotesingle{}{\ttfamily or}\textquotesingle{}$\ast$\textquotesingle{}` (default) -\/ Internet Explorer 9+ compatibility mode
\end{DoxyItemize}

Since clean-\/css 3 a fine grained control is available over \href{https://github.com/jakubpawlowicz/clean-css/blob/master/lib/utils/compatibility.js}{\tt those settings}, with the following options available\+:


\begin{DoxyItemize}
\item `\textquotesingle{}\mbox{[}+-\/\mbox{]}colors.\+opacity\textquotesingle{}{\ttfamily -\/ -\/ turn on (+) / off (-\/)}rgba(){\ttfamily /}hsla(){\ttfamily declarations removal $\ast$}\textquotesingle{}\mbox{[}+-\/\mbox{]}properties.\+background\+Clip\+Merging\textquotesingle{}{\ttfamily -\/ turn on / off background-\/clip merging into shorthand $\ast$}\textquotesingle{}\mbox{[}+-\/\mbox{]}properties.\+background\+Origin\+Merging\textquotesingle{}{\ttfamily -\/ turn on / off background-\/origin merging into shorthand $\ast$}\textquotesingle{}\mbox{[}+-\/\mbox{]}properties.\+background\+Size\+Merging\textquotesingle{}{\ttfamily -\/ turn on / off background-\/size merging into shorthand $\ast$}\textquotesingle{}\mbox{[}+-\/\mbox{]}properties.\+colors\textquotesingle{}{\ttfamily -\/ turn on / off any color optimizations $\ast$}\textquotesingle{}\mbox{[}+-\/\mbox{]}properties.\+ie\+Bang\+Hack\textquotesingle{}{\ttfamily -\/ turn on / off IE bang hack removal $\ast$}\textquotesingle{}\mbox{[}+-\/\mbox{]}properties.\+ie\+Prefix\+Hack\textquotesingle{}{\ttfamily -\/ turn on / off IE prefix hack removal $\ast$}\textquotesingle{}\mbox{[}+-\/\mbox{]}properties.\+ie\+Suffix\+Hack\textquotesingle{}{\ttfamily -\/ turn on / off IE suffix hack removal $\ast$}\textquotesingle{}\mbox{[}+-\/\mbox{]}properties.\+merging\textquotesingle{}{\ttfamily -\/ turn on / off property merging based on understandability $\ast$}\textquotesingle{}\mbox{[}+-\/\mbox{]}properties.\+space\+After\+Closing\+Brace\textquotesingle{}{\ttfamily -\/ turn on / off removing space after closing brace -\/}url() no-\/repeat{\ttfamily into}url()no-\/repeat{\ttfamily  $\ast$}\textquotesingle{}\mbox{[}+-\/\mbox{]}properties.\+url\+Quotes\textquotesingle{}{\ttfamily -\/ turn on / off}url(){\ttfamily quoting $\ast$}\textquotesingle{}\mbox{[}+-\/\mbox{]}properties.\+zero\+Units\textquotesingle{}{\ttfamily -\/ turn on / off units removal after a}0{\ttfamily value $\ast$}\textquotesingle{}\mbox{[}+-\/\mbox{]}selectors.\+adjacent\+Space\textquotesingle{}{\ttfamily -\/ turn on / off extra space before}nav{\ttfamily element $\ast$}\textquotesingle{}\mbox{[}+-\/\mbox{]}selectors.\+ie7\+Hack\textquotesingle{}{\ttfamily -\/ turn on / off I\+E7 selector hack removal (}$\ast$+html...{\ttfamily ) $\ast$}\textquotesingle{}\mbox{[}+-\/\mbox{]}selectors.\+special\textquotesingle{}{\ttfamily -\/ a regular expression with all special, unmergeable selectors (leave it empty unless you know what you are doing) $\ast$}\textquotesingle{}\mbox{[}+-\/\mbox{]}units.\+ch\textquotesingle{}{\ttfamily -\/ turn on / off treating}ch{\ttfamily as a proper unit $\ast$}\textquotesingle{}\mbox{[}+-\/\mbox{]}units.\+in\textquotesingle{}{\ttfamily -\/ turn on / off treating}in{\ttfamily as a proper unit $\ast$}\textquotesingle{}\mbox{[}+-\/\mbox{]}units.\+pc\textquotesingle{}{\ttfamily -\/ turn on / off treating}pc{\ttfamily as a proper unit $\ast$}\textquotesingle{}\mbox{[}+-\/\mbox{]}units.\+pt\textquotesingle{}{\ttfamily -\/ turn on / off treating}pt{\ttfamily as a proper unit $\ast$}\textquotesingle{}\mbox{[}+-\/\mbox{]}units.\+rem\textquotesingle{}{\ttfamily -\/ turn on / off treating}rem{\ttfamily as a proper unit $\ast$}\textquotesingle{}\mbox{[}+-\/\mbox{]}units.\+vh\textquotesingle{}{\ttfamily -\/ turn on / off treating}vh{\ttfamily as a proper unit $\ast$}\textquotesingle{}\mbox{[}+-\/\mbox{]}units.\+vm\textquotesingle{}{\ttfamily -\/ turn on / off treating}vm{\ttfamily as a proper unit $\ast$}\textquotesingle{}\mbox{[}+-\/\mbox{]}units.\+vmax\textquotesingle{}{\ttfamily -\/ turn on / off treating}vmax{\ttfamily as a proper unit $\ast$}\textquotesingle{}\mbox{[}+-\/\mbox{]}units.\+vmin\textquotesingle{}{\ttfamily -\/ turn on / off treating}vmin{\ttfamily as a proper unit $\ast$}\textquotesingle{}\mbox{[}+-\/\mbox{]}units.\+vm\textquotesingle{}{\ttfamily -\/ turn on / off treating}vm` as a proper unit
\end{DoxyItemize}

For example, using {\ttfamily -\/-\/compatibility \textquotesingle{}ie8,+units.rem\textquotesingle{}} will ensure I\+E8 compatibility while enabling {\ttfamily rem} units so the following style {\ttfamily margin\+:0px 0rem} can be shortened to {\ttfamily margin\+:0}, while in pure I\+E8 mode it can\textquotesingle{}t be.

To pass a single off (-\/) switch in C\+LI please use the following syntax {\ttfamily -\/-\/compatibility $\ast$,-\/units.\+rem}.

In library mode you can also pass {\ttfamily compatibility} as a hash of options.

\subsubsection*{What advanced optimizations are applied?}

All advanced optimizations are dispatched \href{https://github.com/jakubpawlowicz/clean-css/blob/master/lib/selectors/advanced.js#L59}{\tt here}, and this is what they do\+:


\begin{DoxyItemize}
\item {\ttfamily recursively\+Optimize\+Blocks} -\/ does all the following operations on a block (think {\ttfamily @media} or {\ttfamily @keyframe} at-\/rules);
\item {\ttfamily recursively\+Optimize\+Properties} -\/ optimizes properties in rulesets and \char`\"{}flat at-\/rules\char`\"{} (like -\/face) by splitting them into components (e.\+g. {\ttfamily margin} into {\ttfamily margin-\/($\ast$)}), optimizing, and rebuilding them back. You may want to use {\ttfamily shorthand\+Compacting} option to control whether you want to turn multiple (long-\/hand) properties into a shorthand ones;
\item {\ttfamily remove\+Duplicates} -\/ gets rid of duplicate rulesets with exactly the same set of properties (think of including the same Sass / Less partial twice for no good reason);
\item {\ttfamily merge\+Adjacent} -\/ merges adjacent rulesets with the same selector or rules;
\item {\ttfamily reduce\+Non\+Adjacent} -\/ identifies which properties are overridden in same-\/selector non-\/adjacent rulesets, and removes them;
\item {\ttfamily merge\+Non\+Adjacent\+By\+Selector} -\/ identifies same-\/selector non-\/adjacent rulesets which can be moved (!) to be merged, requires all intermediate rulesets to not redefine the moved properties, or if redefined to be either more coarse grained (e.\+g. {\ttfamily margin} vs {\ttfamily margin-\/top}) or have the same value;
\item {\ttfamily merge\+Non\+Adjacent\+By\+Body} -\/ same as the one above but for same-\/rules non-\/adjacent rulesets;
\item {\ttfamily restructure} -\/ tries to reorganize different-\/selector different-\/rules rulesets so they take less space, e.\+g. {\ttfamily .one\{padding\+:0\}.two\{margin\+:0\}.one\{margin-\/bottom\+:3px\}} into {\ttfamily .two\{margin\+:0\}.one\{padding\+:0;margin-\/bottom\+:3px\}};
\item {\ttfamily remove\+Duplicate\+Media\+Queries} -\/ removes duplicated {\ttfamily @media} at-\/rules;
\item {\ttfamily merge\+Media\+Queries} -\/ merges non-\/adjacent {\ttfamily @media} at-\/rules by same rules as {\ttfamily merge\+Non\+Adjacent\+By$\ast$} above;
\end{DoxyItemize}

\subsection*{Acknowledgments (sorted alphabetically)}


\begin{DoxyItemize}
\item Anthony Barre (\href{https://github.com/abarre}{\tt }) for improvements to {\ttfamily @import} processing, namely introducing the {\ttfamily -\/-\/skip-\/import} / {\ttfamily process\+Import} options.
\item Simon Altschuler (\href{https://github.com/altschuler}{\tt }) for fixing {\ttfamily @import} processing inside comments.
\item Isaac (\href{https://github.com/facelessuser}{\tt }) for pointing out a flaw in clean-\/css\textquotesingle{} stateless mode.
\item Jan Michael Alonzo (\href{https://github.com/jmalonzo}{\tt }) for a patch removing node.\+js\textquotesingle{} old {\ttfamily sys} package.
\item Luke Page (\href{https://github.com/lukeapage}{\tt }) for suggestions and testing the source maps feature. Plus everyone else involved in \href{https://github.com/jakubpawlowicz/clean-css/issues/125}{\tt \#125} for pushing it forward.
\item Timur Kristóf (\href{https://github.com/Venemo}{\tt }) for an outstanding contribution of advanced property optimizer for 2.\+2 release.
\item Vincent Voyer (\href{https://github.com/vvo}{\tt }) for a patch with better empty element regex and for inspiring us to do many performance improvements in 0.\+4 release.
\item \href{https://github.com/XhmikosR}{\tt } for suggesting new features (option to remove special comments and strip out U\+R\+Ls quotation) and pointing out numerous improvements (J\+S\+Hint, media queries).
\end{DoxyItemize}

\subsection*{License}

Clean-\/css is released under the \href{https://github.com/jakubpawlowicz/clean-css/blob/master/LICENSE}{\tt M\+IT License}. 