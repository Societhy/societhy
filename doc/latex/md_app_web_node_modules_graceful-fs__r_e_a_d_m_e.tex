graceful-\/fs functions as a drop-\/in replacement for the fs module, making various improvements.

The improvements are meant to normalize behavior across different platforms and environments, and to make filesystem access more resilient to errors.

\subsection*{Improvements over fs module}

graceful-\/fs\+:


\begin{DoxyItemize}
\item keeps track of how many file descriptors are open, and by default limits this to 1024. Any further requests to open a file are put in a queue until new slots become available. If 1024 turns out to be too much, it decreases the limit further.
\item fixes {\ttfamily lchmod} for Node versions prior to 0.\+6.\+2.
\item implements {\ttfamily fs.\+lutimes} if possible. Otherwise it becomes a noop.
\item ignores {\ttfamily E\+I\+N\+V\+AL} and {\ttfamily E\+P\+E\+RM} errors in {\ttfamily chown}, {\ttfamily fchown} or {\ttfamily lchown} if the user isn\textquotesingle{}t root.
\item makes {\ttfamily lchmod} and {\ttfamily lchown} become noops, if not available.
\item retries reading a file if {\ttfamily read} results in E\+A\+G\+A\+IN error.
\end{DoxyItemize}

On Windows, it retries renaming a file for up to one second if {\ttfamily E\+A\+C\+C\+E\+SS} or {\ttfamily E\+P\+E\+RM} error occurs, likely because antivirus software has locked the directory.

\subsection*{Configuration}

The maximum number of open file descriptors that graceful-\/fs manages may be adjusted by setting {\ttfamily fs.\+M\+A\+X\+\_\+\+O\+P\+EN} to a different number. The default is 1024. 