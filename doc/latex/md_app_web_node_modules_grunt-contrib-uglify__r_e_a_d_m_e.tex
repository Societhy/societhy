\begin{quote}
Minify javascript files with Uglify\+JS \end{quote}


\subsection*{Getting Started}

If you haven\textquotesingle{}t used \href{http://gruntjs.com/}{\tt Grunt} before, be sure to check out the \href{http://gruntjs.com/getting-started}{\tt Getting Started} guide, as it explains how to create a \href{http://gruntjs.com/sample-gruntfile}{\tt Gruntfile} as well as install and use Grunt plugins. Once you\textquotesingle{}re familiar with that process, you may install this plugin with this command\+:


\begin{DoxyCode}
npm install grunt-contrib-uglify --save-dev
\end{DoxyCode}


Once the plugin has been installed, it may be enabled inside your Gruntfile with this line of Java\+Script\+:


\begin{DoxyCode}
grunt.loadNpmTasks('grunt-contrib-uglify');
\end{DoxyCode}


\subsection*{Uglify task}

{\itshape Run this task with the {\ttfamily grunt uglify} command.}

Task targets, files and options may be specified according to the grunt \href{http://gruntjs.com/configuring-tasks}{\tt Configuring tasks} guide.

\subsubsection*{Migrating from 2.\+x to 3.\+x}

Version {\ttfamily 3.\+x} introduced changes to configuring source maps. Accordingly, if you don\textquotesingle{}t use the source map options you should be able to upgrade seamlessly. If you do use source maps, see below.

\paragraph*{Removed options}

{\ttfamily source\+Mapping\+U\+RL} -\/ This is calculated automatically now {\ttfamily source\+Map\+Prefix} -\/ No longer necessary for the above reason

\paragraph*{Changed options}

{\ttfamily source\+Map} -\/ Only accepts a {\ttfamily Boolean} value. Generates a map with a default name for you {\ttfamily source\+Map\+Root} -\/ The location of your sources is now calculated for you when {\ttfamily source\+Map} is set to {\ttfamily true} but you can set manual source root if needed

\paragraph*{New options}

{\ttfamily source\+Map\+Name} -\/ Accepts a string or function to change the location or name of your map {\ttfamily source\+Map\+Include\+Sources} -\/ Embed the content of your source files directly into the map {\ttfamily expression} -\/ Accepts a {\ttfamily Boolean} value. Parse a single expression (J\+S\+ON or single functions) {\ttfamily quote\+Style} -\/ Accepts integers {\ttfamily 0} (default), {\ttfamily 1}, {\ttfamily 2}, {\ttfamily 3}. Enforce or preserve quotation mark style.

\subsubsection*{Options}

This task primarily delegates to \href{https://github.com/mishoo/UglifyJS2}{\tt Uglify\+J\+S2}, so please consider the \href{http://lisperator.net/uglifyjs/}{\tt Uglify\+JS documentation} as required reading for advanced configuration.

\paragraph*{mangle}

Type\+: {\ttfamily Boolean} {\ttfamily Object} Default\+: {\ttfamily \{\}}

Turn on or off mangling with default options. If an {\ttfamily Object} is specified, it is passed directly to {\ttfamily ast.\+mangle\+\_\+names()} {\itshape and} {\ttfamily ast.\+compute\+\_\+char\+\_\+frequency()} (mimicking command line behavior). \href{https://github.com/mishoo/UglifyJS2#mangler-options}{\tt View all options here}.

\paragraph*{compress}

Type\+: {\ttfamily Boolean} {\ttfamily Object} Default\+: {\ttfamily \{\}}

Turn on or off source compression with default options. If an {\ttfamily Object} is specified, it is passed as options to {\ttfamily Uglify\+J\+S.\+Compressor()}. \href{https://github.com/mishoo/UglifyJS2#compressor-options}{\tt View all options here}.

\paragraph*{beautify}

Type\+: {\ttfamily Boolean} {\ttfamily Object} Default\+: {\ttfamily false}

Turns on beautification of the generated source code. An {\ttfamily Object} will be merged and passed with the options sent to {\ttfamily Uglify\+J\+S.\+Output\+Stream()}. \href{https://github.com/mishoo/UglifyJS2#beautifier-options}{\tt View all options here}

\subparagraph*{expression}

Type\+: {\ttfamily Boolean} Default\+: {\ttfamily false}

Parse a single expression, rather than a program (for parsing J\+S\+ON)

\paragraph*{report}

Choices\+: {\ttfamily \textquotesingle{}min\textquotesingle{}}, {\ttfamily \textquotesingle{}gzip\textquotesingle{}} Default\+: {\ttfamily \textquotesingle{}min\textquotesingle{}}

Either report only minification result or report minification and gzip results. This is useful to see exactly how well clean-\/css is performing but using {\ttfamily \textquotesingle{}gzip\textquotesingle{}} will make the task take 5-\/10x longer to complete. \href{https://github.com/sindresorhus/maxmin#readme}{\tt Example output}.

\paragraph*{source\+Map}

Type\+: {\ttfamily Boolean} Default\+: {\ttfamily false}

If {\ttfamily true}, a source map file will be generated in the same directory as the {\ttfamily dest} file. By default it will have the same basename as the {\ttfamily dest} file, but with a {\ttfamily .map} extension.

\paragraph*{source\+Map\+Name}

Type\+: {\ttfamily String} {\ttfamily Function} Default\+: {\ttfamily undefined}

To customize the name or location of the generated source map, pass a string to indicate where to write the source map to. If a function is provided, the uglify destination is passed as the argument and the return value will be used as the file name.

\paragraph*{source\+Map\+In}

Type\+: {\ttfamily String} {\ttfamily Function} Default\+: {\ttfamily undefined}

The location of an input source map from an earlier compilation, e.\+g. from Coffee\+Script. If a function is provided, the uglify source is passed as the argument and the return value will be used as the source\+Map name. This only makes sense when there\textquotesingle{}s one source file.

\paragraph*{source\+Map\+Include\+Sources}

Type\+: {\ttfamily Boolean} Default\+: {\ttfamily false}

Pass this flag if you want to include the content of source files in the source map as sources\+Content property.

\subparagraph*{source\+Map\+Root}

Type\+: {\ttfamily String} Default\+: {\ttfamily undefined}

With this option you can customize root U\+RL that browser will use when looking for sources.

If the sources are not absolute U\+R\+Ls after prepending of the {\ttfamily source\+Map\+Root}, the sources are resolved relative to the source map.

\subparagraph*{enclose}

Type\+: {\ttfamily Object} Default\+: {\ttfamily undefined}

Wrap all of the code in a closure with a configurable arguments/parameters list. Each key-\/value pair in the {\ttfamily enclose} object is effectively an argument-\/parameter pair.

\paragraph*{wrap}

Type\+: {\ttfamily String} Default\+: {\ttfamily undefined}

Wrap all of the code in a closure, an easy way to make sure nothing is leaking. For variables that need to be public {\ttfamily exports} and {\ttfamily global} variables are made available. The value of wrap is the global variable exports will be available as.

\paragraph*{max\+Line\+Len}

Type\+: {\ttfamily Number} Default\+: {\ttfamily 32000}

Limit the line length in symbols. Pass max\+Line\+Len = 0 to disable this safety feature.

\paragraph*{A\+S\+C\+I\+I\+Only}

Type\+: {\ttfamily Boolean} Default\+: {\ttfamily false}

Enables to encode non-\/\+A\+S\+C\+II characters as .

\paragraph*{export\+All}

Type\+: {\ttfamily Boolean} Default\+: {\ttfamily false}

When using {\ttfamily wrap} this will make all global functions and variables available via the export variable.

\paragraph*{preserve\+Comments}

Type\+: {\ttfamily Boolean} {\ttfamily String} {\ttfamily Function} Default\+: {\ttfamily undefined} Options\+: {\ttfamily false} {\ttfamily \textquotesingle{}all\textquotesingle{}} {\ttfamily \textquotesingle{}some\textquotesingle{}}

Turn on preservation of comments.


\begin{DoxyItemize}
\item {\ttfamily false} will strip all comments
\item {\ttfamily \textquotesingle{}all\textquotesingle{}} will preserve all comments in code blocks that have not been squashed or dropped
\item {\ttfamily \textquotesingle{}some\textquotesingle{}} will preserve all comments that start with a bang ({\ttfamily !}) or include a closure compiler style directive ({\ttfamily @preserve} {\ttfamily @license} {\ttfamily @cc\+\_\+on})
\item {\ttfamily Function} specify your own comment preservation function. You will be passed the current node and the current comment and are expected to return either {\ttfamily true} or {\ttfamily false}
\end{DoxyItemize}

\paragraph*{banner}

Type\+: {\ttfamily String} Default\+: `\textquotesingle{}\textquotesingle{}`

This string will be prepended to the minified output. Template strings (e.\+g. {\ttfamily $<$\%= config.\+value \%$>$} will be expanded automatically.

\paragraph*{footer}

Type\+: {\ttfamily String} Default\+: `\textquotesingle{}\textquotesingle{}`

This string will be appended to the minified output. Template strings (e.\+g. {\ttfamily $<$\%= config.\+value \%$>$} will be expanded automatically.

\paragraph*{screw\+I\+E8}

Type\+: {\ttfamily Boolean} Default\+: {\ttfamily false}

Pass this flag if you don\textquotesingle{}t care about full compliance with Internet Explorer 6-\/8 quirks.

\paragraph*{mangle\+Properties}

Type\+: {\ttfamily Boolean} {\ttfamily Object} Default\+: {\ttfamily false}

Turn on or off property mangling with default options. If an {\ttfamily Object} is specified, it is passed directly to {\ttfamily ast.\+mangle\+\_\+properties()} (mimicking command line behavior). \href{https://github.com/mishoo/UglifyJS2#mangler-options}{\tt View all options here}.

\paragraph*{reserve\+D\+O\+M\+Properties}

Type\+: {\ttfamily Boolean} Default\+: {\ttfamily false}

Use this flag in conjunction with {\ttfamily mangle\+Properties} to prevent built-\/in browser object properties from being mangled.

\paragraph*{exceptions\+Files}

Type\+: {\ttfamily Array} Default\+: {\ttfamily \mbox{[}\mbox{]}}

Use this with {\ttfamily mangle\+Properties} to pass one or more J\+S\+ON files containing a list of variables and object properties that should not be mangled. See the \href{https://www.npmjs.com/package/uglify-js}{\tt Uglify\+JS docs} for more info on the file syntax.

\paragraph*{name\+Cache}

Type\+: {\ttfamily String} Default\+: `\textquotesingle{}\textquotesingle{}`

A string that is a path to a J\+S\+ON cache file that uglify will create and use to coordinate symbol mangling between multiple runs of uglify. Note\+: this generated file uses the same J\+S\+ON format as the {\ttfamily exceptions\+Files} files.

\paragraph*{quote\+Style}

Type\+: {\ttfamily Integer} Default\+: {\ttfamily 0}

Preserve or enforce quotation mark style.


\begin{DoxyItemize}
\item {\ttfamily 0} will use single or double quotes such as to minimize the number of bytes (prefers double quotes when both will do)
\item {\ttfamily 1} will always use single quotes
\item {\ttfamily 2} will always use double quotes
\item {\ttfamily 3} will preserve original quotation marks
\end{DoxyItemize}

\subsubsection*{Usage examples}

\paragraph*{Basic compression}

This configuration will compress and mangle the input files using the default options.


\begin{DoxyCode}
// Project configuration.
grunt.initConfig(\{
  uglify: \{
    my\_target: \{
      files: \{
        'dest/output.min.js': ['src/input1.js', 'src/input2.js']
      \}
    \}
  \}
\});
\end{DoxyCode}


\paragraph*{No mangling}

Specify {\ttfamily mangle\+: false} to prevent changes to your variable and function names.


\begin{DoxyCode}
// Project configuration.
grunt.initConfig(\{
  uglify: \{
    options: \{
      mangle: false
    \},
    my\_target: \{
      files: \{
        'dest/output.min.js': ['src/input.js']
      \}
    \}
  \}
\});
\end{DoxyCode}


\paragraph*{Reserved identifiers}

You can specify identifiers to leave untouched with an {\ttfamily except} array in the {\ttfamily mangle} options.


\begin{DoxyCode}
// Project configuration.
grunt.initConfig(\{
  uglify: \{
    options: \{
      mangle: \{
        except: ['jQuery', 'Backbone']
      \}
    \},
    my\_target: \{
      files: \{
        'dest/output.min.js': ['src/input.js']
      \}
    \}
  \}
\});
\end{DoxyCode}


\paragraph*{Source maps}

Generate a source map by setting the {\ttfamily source\+Map} option to {\ttfamily true}. The generated source map will be in the same directory as the destination file. Its name will be the basename of the destination file with a {\ttfamily .map} extension. Override these defaults with the {\ttfamily source\+Map\+Name} attribute.


\begin{DoxyCode}
// Project configuration.
grunt.initConfig(\{
  uglify: \{
    my\_target: \{
      options: \{
        sourceMap: true,
        sourceMapName: 'path/to/sourcemap.map'
      \},
      files: \{
        'dest/output.min.js': ['src/input.js']
      \}
    \}
  \}
\});
\end{DoxyCode}


\paragraph*{Advanced source maps}

Set the {\ttfamily source\+Map\+Include\+Sources} option to {\ttfamily true} to embed your sources directly into the map. To include a source map from a previous compilation pass it as the value of the {\ttfamily source\+Map\+In} option.


\begin{DoxyCode}
// Project configuration.
grunt.initConfig(\{
  uglify: \{
    my\_target: \{
      options: \{
        sourceMap: true,
        sourceMapIncludeSources: true,
        sourceMapIn: 'example/coffeescript-sourcemap.js', // input sourcemap from a previous compilation
      \},
      files: \{
        'dest/output.min.js': ['src/input.js'],
      \},
    \},
  \},
\});
\end{DoxyCode}


Refer to the \href{http://lisperator.net/uglifyjs/codegen#source-map}{\tt Uglify\+JS Source\+Map Documentation} for more information.

\paragraph*{Turn off console warnings}

Specify {\ttfamily drop\+\_\+console\+: true} as part of the {\ttfamily compress} options to discard calls to {\ttfamily console.$\ast$} functions. This will suppress warning messages in the console.


\begin{DoxyCode}
// Project configuration.
grunt.initConfig(\{
  uglify: \{
    options: \{
      compress: \{
        drop\_console: true
      \}
    \},
    my\_target: \{
      files: \{
        'dest/output.min.js': ['src/input.js']
      \}
    \}
  \}
\});
\end{DoxyCode}


\paragraph*{Beautify}

Specify {\ttfamily beautify\+: true} to beautify your code for debugging/troubleshooting purposes. Pass an object to manually configure any other output options passed directly to {\ttfamily Uglify\+J\+S.\+Output\+Stream()}.

See \href{http://lisperator.net/uglifyjs/codegen}{\tt Uglify\+JS Codegen documentation} for more information.

{\itshape Note that manual configuration will require you to explicitly set {\ttfamily beautify\+: true} if you want traditional, beautified output.}


\begin{DoxyCode}
// Project configuration.
grunt.initConfig(\{
  uglify: \{
    my\_target: \{
      options: \{
        beautify: true
      \},
      files: \{
        'dest/output.min.js': ['src/input.js']
      \}
    \},
    my\_advanced\_target: \{
      options: \{
        beautify: \{
          width: 80,
          beautify: true
        \}
      \},
      files: \{
        'dest/output.min.js': ['src/input.js']
      \}
    \}
  \}
\});
\end{DoxyCode}


\paragraph*{Banner comments}

In this example, running {\ttfamily grunt uglify\+:my\+\_\+target} will prepend a banner created by interpolating the {\ttfamily banner} template string with the config object. Here, those properties are the values imported from the {\ttfamily package.\+json} file (which are available via the {\ttfamily pkg} config property) plus today\textquotesingle{}s date.

{\itshape Note\+: you don\textquotesingle{}t have to use an external J\+S\+ON file. It\textquotesingle{}s also valid to create the {\ttfamily pkg} object inline in the config. That being said, if you already have a J\+S\+ON file, you might as well reference it.}


\begin{DoxyCode}
// Project configuration.
grunt.initConfig(\{
  pkg: grunt.file.readJSON('package.json'),
  uglify: \{
    options: \{
      banner: '/*! <%= pkg.name %> - v<%= pkg.version %> - ' +
        '<%= grunt.template.today("yyyy-mm-dd") %> */'
    \},
    my\_target: \{
      files: \{
        'dest/output.min.js': ['src/input.js']
      \}
    \}
  \}
\});
\end{DoxyCode}


\paragraph*{Conditional compilation}

You can also enable Uglify\+JS conditional compilation. This is commonly used to remove debug code blocks for production builds.

See \href{http://lisperator.net/uglifyjs/compress#global-defs}{\tt Uglify\+JS global definitions documentation} for more information.


\begin{DoxyCode}
// Project configuration.
grunt.initConfig(\{
  uglify: \{
    options: \{
      compress: \{
        global\_defs: \{
          "DEBUG": false
        \},
        dead\_code: true
      \}
    \},
    my\_target: \{
      files: \{
        'dest/output.min.js': ['src/input.js']
      \}
    \}
  \}
\});
\end{DoxyCode}


\paragraph*{Compiling all files in a folder dynamically}

This configuration will compress and mangle the files dynamically.


\begin{DoxyCode}
// Project configuration.
grunt.initConfig(\{
  uglify: \{
    my\_target: \{
      files: [\{
          expand: true,
          cwd: 'src/js',
          src: '**/*.js',
          dest: 'dest/js'
      \}]
    \}
  \}
\});
\end{DoxyCode}


\paragraph*{Turn on object property name mangling}

This configuration will turn on object property name mangling, but not mangle built-\/in browser object properties. Additionally, variables and object properties listed in the {\ttfamily my\+Exceptions\+File.\+json} will be mangled. For more info, on the format of the exception file format please see the \href{https://www.npmjs.com/package/uglify-js}{\tt Uglify\+JS docs}.


\begin{DoxyCode}
// Project configuration.
grunt.initConfig(\{
  uglify: \{
    options: \{
      mangleProperties: true,
      reserveDOMCache: true,
      exceptionsFiles: [ 'myExceptionsFile.json' ]
    \},
    my\_target: \{
      files: \{
        'dest/output.min.js': ['src/input.js']
      \}
    \}
  \}
\});
\end{DoxyCode}


\paragraph*{Turn on use of name mangling cache}

Turn on use of name mangling cache to coordinate mangled symbols between outputted uglify files. uglify will the generate a J\+S\+ON cache file with the name provided in the options. Note\+: this generated file uses the same J\+S\+ON format as the {\ttfamily exceptions\+Files} files.


\begin{DoxyCode}
// Project configuration.
grunt.initConfig(\{
  uglify: \{
    options: \{
      nameCache: '.tmp/grunt-uglify-cache.json',
    \},
    my\_target: \{
      files: \{
        'dest/output1.min.js': ['src/input1.js'],
        'dest/output2.min.js': ['src/input2.js']
      \}
    \}
  \}
\});
\end{DoxyCode}


\subsection*{Release History}


\begin{DoxyItemize}
\item 2016-\/01-\/29   v0.11.\+1   switch to lodash $^\wedge$4.0.\+1 switch to grunt-\/contrib-\/clean $^\wedge$0.7.\+0 switch to grunt-\/contrib-\/jshint $^\wedge$0.12.\+0
\item 2015-\/11-\/20   v0.11.\+0   switch to uglify $\sim$2.6.\+0
\item 2015-\/11-\/12   v0.10.\+1   switch to uglify $\sim$2.5
\item 2015-\/10-\/27   v0.10.\+0   bump to uglify $^\wedge$2.5
\item 2015-\/08-\/24   v0.9.\+2   bump to uglify $^\wedge$2.4.\+24
\item 2015-\/04-\/07   v0.9.\+1   more fixes for mangle options
\item 2015-\/04-\/07   v0.9.\+0   added hook into uglify\textquotesingle{}s mangling functionality
\item 2015-\/03-\/30   v0.8.\+1   lock uglify to 2.\+4.\+17 due to breaking changes
\item 2015-\/02-\/19   v0.8.\+0   \+Add {\ttfamily screw\+I\+E8} option. Fix issue with explicit {\ttfamily compress} in node 0.\+12.\+0.
\item 2014-\/12-\/23   v0.7.\+0   \+Adds source\+Map\+Root options. Updates readme descriptions. Removes reference to cleancss.
\item 2014-\/09-\/17   v0.6.\+0   \+Output fixes. A\+S\+C\+I\+I\+Only option. Other fixes.
\item 2014-\/07-\/25   v0.5.\+1   \+Chalk updates. Output updates.
\item 2014-\/03-\/01   v0.4.\+0   remove grunt-\/lib-\/contrib dependency and add more colors
\item 2014-\/02-\/27   v0.3.\+3   remove unnecessary calls to {\ttfamily grunt.\+template.\+process}
\item 2014-\/01-\/22   v0.3.\+2   fix handling of {\ttfamily source\+Map\+Include\+Sources} option.
\item 2014-\/01-\/20   v0.3.\+1   fix relative path issue in sourcemaps
\item 2014-\/01-\/16   v0.3.\+0   refactor sourcemap support
\item 2013-\/11-\/09   v0.2.\+7   prepending banner if source\+Map option not set, addresses
\item 2013-\/11-\/08   v0.2.\+6   merged 45, 53, 85 (105 by way of duping 53) Added support for banners in uglified files with sourcemaps Updated docs
\item 2013-\/10-\/28   v0.2.\+5   \+Added warning for banners when using sourcemaps
\item 2013-\/09-\/02   v0.2.\+4   updated sourcemap format via /83
\item 2013-\/06-\/10   v0.2.\+3   added footer option
\item 2013-\/05-\/31   v0.2.\+2   \+Reverted /56 due to /58 until \href{https://code.google.com/p/chromium/issues/detail?id=239660&q=sourcemappingurl&colspec=ID%20Pri%20M%20Iteration%20ReleaseBlock%20Cr%20Status%20Owner%20Summary%20OS%20Modified}{\tt chrome/239660} \href{https://bugzilla.mozilla.org/show_bug.cgi?id=870361}{\tt firefox/870361} drop
\item 2013-\/05-\/22   v0.2.\+1   \+Bumped uglify to $\sim$2.3.\+5 /55 /40 Changed sourcemapping\+Url syntax /56 Disabled sorting of names for consistent mangling /44 Updated docs for source\+Map\+Root /47 /25
\item 2013-\/03-\/14   v0.2.\+0   \+No longer report gzip results by default. Support {\ttfamily report} option.
\item 2013-\/01-\/30   v0.1.\+2   \+Added better error reporting Support for dynamic names of multiple sourcemaps
\item 2013-\/02-\/15   v0.1.\+1   \+First official release for Grunt 0.\+4.\+0.
\item 2013-\/01-\/18   v0.1.\+1rc6   \+Updating grunt/gruntplugin dependencies to rc6. Changing in-\/development grunt/gruntplugin dependency versions from tilde version ranges to specific versions.
\item 2013-\/01-\/09   v0.1.\+1rc5   \+Updating to work with grunt v0.\+4.\+0rc5. Switching back to this.\+files api.
\item 2012-\/11-\/28   v0.1.\+0   \+Work in progress, not yet officially released. 


\end{DoxyItemize}

Task submitted by \href{http://benalman.com}{\tt \char`\"{}\+Cowboy\char`\"{} Ben Alman}

{\itshape This file was generated on Tue Feb 02 2016 11\+:38\+:37.} 