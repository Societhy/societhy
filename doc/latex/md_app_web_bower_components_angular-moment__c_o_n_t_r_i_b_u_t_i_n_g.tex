Contributing to {\ttfamily angular-\/moment} is fairly easy. This document shows you how to get the project, run all provided tests and generate a production ready build.

It also covers provided grunt tasks, that help you developing on {\ttfamily angular-\/moment}.

\subsection*{Dependencies}

To make sure, that the following instructions work, please install the following dependencies on you machine\+:


\begin{DoxyItemize}
\item Node.\+js
\item npm
\item Git
\end{DoxyItemize}

\subsection*{Installation}

To get the source of {\ttfamily angular-\/moment} clone the git repository via\+:

{\ttfamily git clone \href{https://github.com/urish/angular-moment}{\tt https\+://github.\+com/urish/angular-\/moment}}

This will clone the complete source to your local machine. Navigate to the project folder and install all needed dependencies via {\bfseries npm}\+:

{\ttfamily npm install}

To complete the installation, install the frontend (bower) dependencies by running the following command\+:

{\ttfamily npm run bower}

Well done! angular-\/moment is now installed and ready to be built.

\subsection*{Building}

{\ttfamily angular-\/moment} comes with a few {\bfseries grunt tasks} which help you to automate the development process. The following grunt tasks are provided\+:

\paragraph*{grunt test}

{\ttfamily npm test} executes (as you might thought) the unit tests, which are located in {\ttfamily tests.\+js}. The task uses the {\bfseries karma} test runner to executes the tests with the {\bfseries jasmine testing framework}. This task also checks the coding using {\bfseries jshint}.

\paragraph*{grunt build}

{\ttfamily npm run build} updates the minified version of the code (angular-\/moment.\+min.\+js). It also checks the code using {\bfseries jshint}.

\subsection*{Contributing/\+Submitting changes}


\begin{DoxyItemize}
\item Checkout a new branch based on {\ttfamily master} and name it to what you intend to do\+:
\begin{DoxyItemize}
\item Example\+: ```` \$ git checkout -\/b B\+R\+A\+N\+C\+H\+\_\+\+N\+A\+ME ````
\item Use one branch per fix/feature
\end{DoxyItemize}
\item Make your changes
\begin{DoxyItemize}
\item Make sure to provide a spec for unit tests (in {\ttfamily tests.\+js})
\item Run your tests with {\ttfamily npm test}
\item When all tests pass, everything\textquotesingle{}s fine
\end{DoxyItemize}
\item Commit your changes
\begin{DoxyItemize}
\item Please provide a git message which explains what you\textquotesingle{}ve done
\item Commit to the forked repository
\end{DoxyItemize}
\item Make a pull request
\end{DoxyItemize}

If you follow these instructions, your PR will land pretty safety in the main repo! 