

 title\+: Getting started \subsection*{anchor\+: getting-\/started }

\subsubsection*{Include Chart.\+js}

First we need to include the Chart.\+js library on the page. The library occupies a global variable of {\ttfamily Chart}.


\begin{DoxyCode}
<script src="Chart.js"></script>
\end{DoxyCode}


Alternatively, if you\textquotesingle{}re using an A\+MD loader for Java\+Script modules, that is also supported in the Chart.\+js core. Please note\+: the library will still occupy a global variable of {\ttfamily Chart}, even if it detects {\ttfamily define} and {\ttfamily define.\+amd}. If this is a problem, you can call {\ttfamily no\+Conflict} to restore the global Chart variable to it\textquotesingle{}s previous owner.


\begin{DoxyCode}
// Using requirejs
require(['path/to/Chartjs'], function(Chart)\{
    // Use Chart.js as normal here.

    // Chart.noConflict restores the Chart global variable to it's previous owner
    // The function returns what was previously Chart, allowing you to reassign.
    var Chartjs = Chart.noConflict();

\});
\end{DoxyCode}


You can also grab Chart.\+js using bower\+:


\begin{DoxyCode}
bower install Chart.js --save
\end{DoxyCode}


\subsubsection*{Creating a chart}

To create a chart, we need to instantiate the {\ttfamily Chart} class. To do this, we need to pass in the 2d context of where we want to draw the chart. Here\textquotesingle{}s an example.


\begin{DoxyCode}
<canvas id="myChart" width="400" height="400"></canvas>
\end{DoxyCode}



\begin{DoxyCode}
// Get the context of the canvas element we want to select
var ctx = document.getElementById("myChart").getContext("2d");
var myNewChart = new Chart(ctx).PolarArea(data);
\end{DoxyCode}


We can also get the context of our canvas with j\+Query. To do this, we need to get the D\+OM node out of the j\+Query collection, and call the {\ttfamily get\+Context(\char`\"{}2d\char`\"{})} method on that.


\begin{DoxyCode}
// Get context with jQuery - using jQuery's .get() method.
var ctx = $("#myChart").get(0).getContext("2d");
// This will get the first returned node in the jQuery collection.
var myNewChart = new Chart(ctx);
\end{DoxyCode}


After we\textquotesingle{}ve instantiated the Chart class on the canvas we want to draw on, Chart.\+js will handle the scaling for retina displays.

With the Chart class set up, we can go on to create one of the charts Chart.\+js has available. In the example below, we would be drawing a Polar area chart.


\begin{DoxyCode}
new Chart(ctx).PolarArea(data, options);
\end{DoxyCode}


We call a method of the name of the chart we want to create. We pass in the data for that chart type, and the options for that chart as parameters. Chart.\+js will merge the global defaults with chart type specific defaults, then merge any options passed in as a second argument after data.

\subsubsection*{\hyperlink{class_global}{Global} chart configuration}

This concept was introduced in Chart.\+js 1.\+0 to keep configuration D\+RY, and allow for changing options globally across chart types, avoiding the need to specify options for each instance, or the default for a particular chart type.


\begin{DoxyCode}
Chart.defaults.global = \{
    // Boolean - Whether to animate the chart
    animation: true,

    // Number - Number of animation steps
    animationSteps: 60,

    // String - Animation easing effect
    animationEasing: "easeOutQuart",

    // Boolean - If we should show the scale at all
    showScale: true,

    // Boolean - If we want to override with a hard coded scale
    scaleOverride: false,

    // ** Required if scaleOverride is true **
    // Number - The number of steps in a hard coded scale
    scaleSteps: null,
    // Number - The value jump in the hard coded scale
    scaleStepWidth: null,
    // Number - The scale starting value
    scaleStartValue: null,

    // String - Colour of the scale line
    scaleLineColor: "rgba(0,0,0,.1)",

    // Number - Pixel width of the scale line
    scaleLineWidth: 1,

    // Boolean - Whether to show labels on the scale
    scaleShowLabels: true,

    // Interpolated JS string - can access value
    scaleLabel: "<%=value%>",

    // Boolean - Whether the scale should stick to integers, not floats even if drawing space is there
    scaleIntegersOnly: true,

    // Boolean - Whether the scale should start at zero, or an order of magnitude down from the lowest
       value
    scaleBeginAtZero: false,

    // String - Scale label font declaration for the scale label
    scaleFontFamily: "'Helvetica Neue', 'Helvetica', 'Arial', sans-serif",

    // Number - Scale label font size in pixels
    scaleFontSize: 12,

    // String - Scale label font weight style
    scaleFontStyle: "normal",

    // String - Scale label font colour
    scaleFontColor: "#666",

    // Boolean - whether or not the chart should be responsive and resize when the browser does.
    responsive: false,

    // Boolean - whether to maintain the starting aspect ratio or not when responsive, if set to false,
       will take up entire container
    maintainAspectRatio: true,

    // Boolean - Determines whether to draw tooltips on the canvas or not
    showTooltips: true,

    // Function - Determines whether to execute the customTooltips function instead of drawing the built in
       tooltips (See [Advanced - External Tooltips](#advanced-usage-custom-tooltips))
    customTooltips: false,

    // Array - Array of string names to attach tooltip events
    tooltipEvents: ["mousemove", "touchstart", "touchmove"],

    // String - Tooltip background colour
    tooltipFillColor: "rgba(0,0,0,0.8)",

    // String - Tooltip label font declaration for the scale label
    tooltipFontFamily: "'Helvetica Neue', 'Helvetica', 'Arial', sans-serif",

    // Number - Tooltip label font size in pixels
    tooltipFontSize: 14,

    // String - Tooltip font weight style
    tooltipFontStyle: "normal",

    // String - Tooltip label font colour
    tooltipFontColor: "#fff",

    // String - Tooltip title font declaration for the scale label
    tooltipTitleFontFamily: "'Helvetica Neue', 'Helvetica', 'Arial', sans-serif",

    // Number - Tooltip title font size in pixels
    tooltipTitleFontSize: 14,

    // String - Tooltip title font weight style
    tooltipTitleFontStyle: "bold",

    // String - Tooltip title font colour
    tooltipTitleFontColor: "#fff",

    // Number - pixel width of padding around tooltip text
    tooltipYPadding: 6,

    // Number - pixel width of padding around tooltip text
    tooltipXPadding: 6,

    // Number - Size of the caret on the tooltip
    tooltipCaretSize: 8,

    // Number - Pixel radius of the tooltip border
    tooltipCornerRadius: 6,

    // Number - Pixel offset from point x to tooltip edge
    tooltipXOffset: 10,
    \{% raw %\}
    // String - Template string for single tooltips
    tooltipTemplate: "<%if (label)\{%><%=label%>: <%\}%><%= value %>",
    \{% endraw %\}
    // String - Template string for multiple tooltips
    multiTooltipTemplate: "<%= value %>",

    // Function - Will fire on animation progression.
    onAnimationProgress: function()\{\},

    // Function - Will fire on animation completion.
    onAnimationComplete: function()\{\}
\}
\end{DoxyCode}


If for example, you wanted all charts created to be responsive, and resize when the browser window does, the following setting can be changed\+:


\begin{DoxyCode}
Chart.defaults.global.responsive = true;
\end{DoxyCode}


Now, every time we create a chart, {\ttfamily options.\+responsive} will be {\ttfamily true}. 