Add notification popups to any element \href{http://jemonjam.com/angular-notification-icons}{\tt http\+://jemonjam.\+com/angular-\/notification-\/icons}

\href{https://travis-ci.org/jacob-meacham/angular-notification-icons}{\tt } \href{https://coveralls.io/r/jacob-meacham/angular-notification-icons?branch=develop}{\tt } \href{https://codeclimate.com/github/jacob-meacham/grunt-lcov-merge}{\tt }




\begin{DoxyCode}
<notification-icon count='scopeVariable'>
  <i class="fa fa-envelope-o fa-3x"></i>
</notification-icon>
\end{DoxyCode}


\subsection*{Getting Started}

\subsubsection*{1. Install components}

Bower\+: 
\begin{DoxyCode}
bower install angular-notification-icons --save
\end{DoxyCode}


npm\+: 
\begin{DoxyCode}
npm install angular-notification-icons --save
\end{DoxyCode}


\subsubsection*{2. Add css and scripts}

Bower\+: 
\begin{DoxyCode}
<link rel="stylesheet"
       href="bower\_components/angular-notification-icons/dist/angular-notification-icons.min.css">

<script src="bower\_components/angular/angular.js"></script>
<script src="bower\_components/angular-animate/angular-animate.js"></script> 
<script src="bower\_components/angular-notification-icons/dist/angular-notification-icons.min.js"></script>
\end{DoxyCode}


npm\+: 
\begin{DoxyCode}
<link rel="stylesheet"
       href="node\_modules/angular-notification-icons/dist/angular-notification-icons.min.css">

<script src="bower\_components/angular/angular.js"></script>
<script src="bower\_components/angular-animate/angular-animate.js"></script> 
<script src="bower\_components/angular-notification-icons/dist/angular-notification-icons.min.js"></script>
\end{DoxyCode}


\#\#\# 3. Add a dependency to your app 
\begin{DoxyCode}
angular.module('MyApp', ['angular-notification-icons', 'ngAnimate']); // ngAnimate is only required if you
       want animations
\end{DoxyCode}


\#\#\# 4. Add a notification-\/icon element around any other element 
\begin{DoxyCode}
<notification-icon count='scopeVariable'>
  ...
</notification-icon>
\end{DoxyCode}


angular-\/notification-\/icons is an angular directive that adds a notification popup on top of any element. The counter is tied to a scope variable and updating the count is as easy as updating the scope variable. angular-\/notification-\/icons comes with a number of canned animations and a default style, but it is easy to add your own styles or custom animations. angular-\/notification-\/icons can also optionally listen for D\+OM events and clear the count on a D\+OM event.

angular-\/notification-\/icons has been tested with angular 1.\+3.\+x, 1.\+4.\+x, and 1.\+5.\+x. It will probably work with most recent versions of angular and angular-\/animate.

\subsection*{Demo App}

To run the demo app, run {\ttfamily npm install}, {\ttfamily bower install} and then {\ttfamily gulp serve}.

\subsection*{Webpack and E\+S6}

angular-\/notification-\/icons can be used in a webpack Angular application. To the top of your main application, add\+: 
\begin{DoxyCode}
import angular from 'angular'
import 'angular-animate'
import 'angular-notification-icons'
import 'angular-notification-icons/dist/angular-notification-icons.css'
// Can also use less with a less-loader:
// import 'angular-notification-icons/src/less/angular-notification-icons.less'
\end{DoxyCode}


\subsection*{Basic Usage}

\subsubsection*{Counter}

The only required attribute for angular-\/notification-\/icons is \textquotesingle{}count\textquotesingle{}. This uses two-\/way binding to bind a scope variable to the isolate scope. This makes updating the count very simple, since your controller has full control over how it\textquotesingle{}s set. 
\begin{DoxyCode}
<notification-icon count="myScopeVariable">
    ...
</notification-icon>
\end{DoxyCode}
 When my\+Scope\+Variable is $<$= 0, the notification icon will not be visible. Once my\+Scope\+Variable $>$ 0, the notification will show up.

\href{http://jemonjam.com/angular-notification-icons#basic}{\tt Live Demo}

\subsubsection*{Built-\/in Animations}

angular-\/notification-\/icons comes with a few prebuilt animations for your use. Note that these are only available if you are using the ng\+Animate module


\begin{DoxyItemize}
\item bounce
\item fade
\item grow
\item shake
\end{DoxyItemize}

There are three separate animation events\+: appear, update, and disappear. Appear is triggered when the counter goes from 0 to non-\/zero. Update is trigger when the counter increments or decrements but does not go to or from zero. Disappear is triggered when the counter goes from non-\/zero to zero. The default animation for appear and update is grow, and there is no default set for disappear. A common case is to use the same animation for appear and update, and you can use the \textquotesingle{}animation\textquotesingle{} attribute for this case.


\begin{DoxyCode}
<notification-icon count="myCount" animation='bounce'>
    ...
</notification-icon>
\end{DoxyCode}
 This will create a notification that bounces when appearing and when the counter is updated. All three animation events can also be set explicitly\+:


\begin{DoxyCode}
<notification-icon count="myCount" appear-animation='bounce' update-animation='shake'
       disappear-animation='fade'>
    ...
</notification-icon>
\end{DoxyCode}
 This will create a notification that bounces when appearing, shakes when updates, and fades away when disappearing. Because all of these attributes do not use two-\/way binding, if you\textquotesingle{}re using a variable for the animation, you\textquotesingle{}ll want to use \{\{my\+Variable\}\} when binding.

\href{http://jemonjam.com/angular-notification-icons#animations}{\tt Live Demo}

\subsubsection*{D\+OM Events}

angular-\/notification-\/icons can respond to D\+OM events to clear the counter. This clears the scope variable and runs an \$apply. Your controller can \$watch the variable if you want to react to clearing the counter.


\begin{DoxyCode}
<notification-icon count="myCount" clear-trigger='click'>
    ...
</notification-icon>
\end{DoxyCode}
 Will cause the count to be cleared upon click. Any D\+OM event name is valid as a clear-\/trigger. Because clear-\/trigger does not use two-\/way binding, if you\textquotesingle{}re using a variable as the trigger, you\textquotesingle{}ll want to use \{\{my\+Variable\}\} when binding.

\href{http://jemonjam.com/angular-notification-icons#dom-events}{\tt Live Demo}

\subsection*{Customizing}

angular-\/notification-\/icons was designed to be very simple to customize so that it fits the feel of your app.

\subsubsection*{Adding Custom Style}

Adding custom style is done via C\+SS. When the directive is created, it adds a few elements to the D\+OM 
\begin{DoxyCode}
<notification-icon>
    <div class="angular-notification-icons-container">
        <div class="angular-notification-icons-icon overlay">
            <div class="notification-inner">
                
            </div>
        </div>
    </div>
</notification-icon>
\end{DoxyCode}


You can add styling at any level. For instance, if you just want to change the look of the notifaction icon, you can add to your app\textquotesingle{}s css\+:


\begin{DoxyCode}
.angular-notification-icons-icon \{
  left: -10px;
  background: yellow;
  color: black;
  width: 30px;
  height: 30px;
  font-weight: bolder;
  font-size: 1.2em;
\}
\end{DoxyCode}
 Which will make the notification icon appear on the left with a yellow background and bold, larger text. \href{http://jemonjam.com/angular-notification-icons#custom-style}{\tt Live Demo}

\subsubsection*{Adding Custom Animations}

Adding a custom animation is as simple as adding custom styles. angular-\/notification-\/icons uses the standard \href{https://docs.angularjs.org/guide/animations}{\tt angular-\/animate} module for providing animations. This means that you can use either C\+SS keyframes or C\+SS transitions to build animations.


\begin{DoxyCode}
.angular-notification-icons-icon.my-custom-animation \{
  transition:0.5s linear all;
\}

.angular-notification-icons-icon.my-custom-animation-add \{
  background: black;
  color: white;
\}

.angular-notification-icons-icon.my-custom-animation-add-active \{
  background: yellow;
  color: black;
\}

.angular-notification-icons-icon.my-custom-keyframe-animation \{
    animation: custom\_keyframe\_animation 0.5s;
\}
@keyframes custom\_keyframe\_animation \{
  from \{
    opacity: 0;
  \}
  to \{
    opacity: 1;
  \}
\}
\end{DoxyCode}


Adding your animation is as simple as specifying it by name on the directive 
\begin{DoxyCode}
<notification-icon count='myCount' animation='my-custom-animation'
       disappear-animation='my-custom-keyframe-animation'>
    ...
</notification-icon>
\end{DoxyCode}
 \href{http://jemonjam.com/angular-notification-icons#custom-style}{\tt Live Demo}

\subsection*{Advanced Usage}

\subsubsection*{hide\+Count}

If you don\textquotesingle{}t want the count number appear, you can hide the count using the \textquotesingle{}hide-\/count\textquotesingle{} attribute 
\begin{DoxyCode}
<notification-icon count='myCount' hide-count='true'>
    ...
</notification-icon>
\end{DoxyCode}
 When my\+Count $>$ 0, the notification icon will be visible, but the number will be hidden. When my\+Count $<$= 0, the icon will be hidden as normal.

\href{http://jemonjam.com/angular-notification-icons#hide-count}{\tt Live Demo}

\subsubsection*{always\+Show}

If you {\itshape always} want the count number to appear, even when 0 or negative, you can add the \textquotesingle{}always-\/show\textquotesingle{} attribute 
\begin{DoxyCode}
<notification-icon count='myCount' always-show='true'>
  ...
</notification-icon>
\end{DoxyCode}


\href{http://jemonjam.com/angular-notification-icons#always-show}{\tt Live Demo}

\subsubsection*{Pill shape}

When the number of notifications grows large enough, the icon changes to a pill shape. This is achieved by adding the css class wide-\/icon to the icon\textquotesingle{}s div. By default, the shape transitions to a pill once the count is greater than or equal to 100, but is configurable via the attribute \textquotesingle{}wide-\/threshold\textquotesingle{}. 
\begin{DoxyCode}
<notification-icon count='myCount' wide-threshold='10'>
    ...
</notification-icon>
\end{DoxyCode}
 This will change the shape to a pill once my\+Count $>$= 10.

\href{http://jemonjam.com/angular-notification-icons#pill}{\tt Live Demo}

\subsection*{Helping Out}

Pull requests are gladly accepted! Be sure to run {\ttfamily npm run build} to ensure that your PR is lint-\/free and all the tests pass. 