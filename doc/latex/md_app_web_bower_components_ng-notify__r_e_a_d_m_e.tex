A simple, lightweight module for displaying notifications in your Angular\+JS app.

Both JS and C\+SS files combine for $\sim$4.5 k\+Bs.

I\+E9+ (Angular\+JS v1.\+3.\+x no longer supports I\+E8) and the latest versions of Chrome, Fire\+Fox and Safari have been tested and are supported. If you do run across any issues, please submit a \href{https://github.com/matowens/ng-notify/issues}{\tt new issue} and I\textquotesingle{}ll take a look -\/ or better yet -\/ submit a PR with the bug fix and I\textquotesingle{}ll merge it in.

You can check out the vitals and demo here\+: \href{http://matowens.github.io/ng-notify}{\tt http\+://matowens.\+github.\+io/ng-\/notify}

\section*{Implementation }

\subsubsection*{Requirements}

Angular\+JS is the only dependency. Animation is achieved with pure JS, j\+Query not necessary.

\subsubsection*{Installation}

You can install ng-\/notify with Bower. \begin{DoxyVerb}bower install ng-notify --save
\end{DoxyVerb}


You can also install ng-\/notify with npm. \begin{DoxyVerb}npm install ng-notify --save
\end{DoxyVerb}


As of v0.\+6.\+0, ng-\/notify is now available via the js\+Delivr C\+DN if you\textquotesingle{}d prefer to go down that route. \begin{DoxyVerb}//cdn.jsdelivr.net/angular.ng-notify/{version.number.here}/ng-notify.min.js
//cdn.jsdelivr.net/angular.ng-notify/{version.number.here}/ng-notify.min.css
\end{DoxyVerb}


For example\+: \begin{DoxyVerb}//cdn.jsdelivr.net/angular.ng-notify/0.6.0/ng-notify.min.js
//cdn.jsdelivr.net/angular.ng-notify/0.6.0/ng-notify.min.css
\end{DoxyVerb}


And as always, you can download the source files straight from this repo -\/ they\textquotesingle{}re located in the {\ttfamily dist} dir. Be sure to include the minified version of both js and css files.

\subsubsection*{Usage}

After including {\itshape ng-\/notify.\+min.\+js} and {\itshape ng-\/notify.\+min.\+css}, inject the ng-\/notify provider into your project.


\begin{DoxyCode}
var app = angular.module('demo', ['ngNotify']);
\end{DoxyCode}


Now you can trigger notifications from anywhere in your app. To display a notification, just use the {\ttfamily set} method.


\begin{DoxyCode}
ngNotify.set('Your notification message goes here!');
\end{DoxyCode}


To specify the {\itshape type} of notification to display, provide the optional {\itshape type} param. (For available types, check the \href{#definitions}{\tt definitions} below.)


\begin{DoxyCode}
ngNotify.set('Your error message goes here!', 'error');
\end{DoxyCode}


\section*{Advanced Usage }

\subsubsection*{Default Configuration}

You can override the default options for all notifications by using the {\ttfamily config} method. None of these options are required. (For available options, check the \href{#definitions}{\tt definitions} below.)


\begin{DoxyCode}
ngNotify.config(\{
    theme: 'pure',
    position: 'bottom',
    duration: 3000,
    type: 'info',
    sticky: false,
    button: true,
    html: false
\});
\end{DoxyCode}


Default configuration options can be set during the {\ttfamily run()} block. If your app utilizes a global controller, the config options could be set there just as well. For a discussion and working example on this topic, checkout \href{https://github.com/matowens/ng-notify/issues/16#issuecomment-104492193}{\tt this comment}.

\subsubsection*{Individual Configurations}

You can also pass an object of options to individual notifications. You can pass through any combination of our available options here as well. (For available options, check the \href{#definitions}{\tt definitions} below.) For example\+:


\begin{DoxyCode}
ngNotify.set('Your first message.', \{
    position: 'top',
    sticky: true
\});

ngNotify.set('Your second message.', \{
    type: 'error',
    duration: 2000
\});

ngNotify.set('Your third message.', 'error'); // Original use case still works, too.

ngNotify.set('Your <i>fourth</i> message.', \{
    theme: 'pitchy',
    html: true
\});
\end{DoxyCode}


\subsubsection*{Sticky Notifications}

Sticky notifications allow you to set a persistent notification that doesn\textquotesingle{}t fade away. To do this, simply set the {\ttfamily sticky} attribute to true\+:


\begin{DoxyCode}
ngNotify.set('This is sticky.', \{
    sticky: true
\});
\end{DoxyCode}


This will give the user the option of closing the notification themselves. If you need to dismiss a notification manually, you can do so with the {\ttfamily dismiss} method like this\+:


\begin{DoxyCode}
ngNotify.dismiss();
\end{DoxyCode}


If you\textquotesingle{}d prefer to dismiss the notification programmtically and prevent the user from doing so, you can add an option to remove the button\+:


\begin{DoxyCode}
ngNotify.set('This is sticky without a button.', \{
    sticky: true,
    button: false
\});
\end{DoxyCode}


{\itshape Any time a notification is set to sticky, the duration attribute will be ignored since the notification will not be automatically fading out.}

\subsubsection*{H\+T\+ML Notifications}

H\+T\+ML notifications will allow you to display messages with H\+T\+ML content in them. To do this, you\textquotesingle{}ll need to set the {\ttfamily html} attribute to true\+:


\begin{DoxyCode}
ngNotify.set('This has <b>HTML</b> content!', \{
    html: true
\});
\end{DoxyCode}


You can also set H\+T\+ML notifications to be enabled for all of your notifications by adding it the ng\+Notify config like so\+:


\begin{DoxyCode}
ngNotify.config(\{
    html: true
\});
\end{DoxyCode}
 In order for H\+T\+ML notifications to display, you are required to include the \href{https://docs.angularjs.org/api/ngSanitize}{\tt ng\+Sanitize} script in your app (eg, via Google C\+DN, Bower, or code.\+angular.\+org). There\textquotesingle{}s no need to add it as a dependency to ng\+Notify. If ng\+Notify has found the ng\+Sanitize script, it will add it as a dependency to the ng\+Notify module dynamically. Once included, you just need to toggle the {\ttfamily html} attribute to true and the module will handle the rest.

If you don\textquotesingle{}t have ng\+Sanitize included and you do set {\ttfamily html} to true, ng\+Notify will gracefully degrade back to the default message display and print a debug message to remind you in your browser\textquotesingle{}s console.

\subsubsection*{Roll Your Own}

There are two additional methods that allow you to create your own types and themes.

\subparagraph*{Custom Notification Types}

Creating a custom type will allow you to add additional types of notifications to use throughout your application. To create a new type, use the {\ttfamily add\+Type} method. The first param is the {\itshape name} you\textquotesingle{}ll use to reference your new type. The second param is the {\itshape class} you\textquotesingle{}ll use to style your new notification type.


\begin{DoxyCode}
ngNotify.addType('notice', 'my-notice-type');
\end{DoxyCode}


Then you can set any of your notifications up to use that type as you would any other, triggering it by using the name you gave it.


\begin{DoxyCode}
ngNotify.set('This notification is using our new type!', 'notice');
\end{DoxyCode}


To style your new type, pick a color you\textquotesingle{}d like to use and set it to the background color of your new style.


\begin{DoxyCode}
.my-notice-type
    background-color: #ABC123
\end{DoxyCode}


\subparagraph*{Custom Themes}

Creating a custom theme will allow you to build an entirely new spectrum of notification messages utilizing the existing notification types. To create a new theme, use the {\ttfamily add\+Theme} method. The first param is the {\itshape name} you\textquotesingle{}ll use to reference your new theme. The second param is the {\itshape class} you\textquotesingle{}ll use to style your new theme\textquotesingle{}s notification types.


\begin{DoxyCode}
ngNotify.addTheme('newTheme', 'my-new-theme');
\end{DoxyCode}


Now you can activate your new theme via the config method, using the name you previously assigned to it.


\begin{DoxyCode}
ngNotify.config(\{
    theme: 'newTheme'
\});
\end{DoxyCode}


To style your new theme, pick a collection of colors you\textquotesingle{}d like to use for each notification type and set them to each type\textquotesingle{}s background color.


\begin{DoxyCode}
.my-new-theme.ngn-info
    background-color: #0033CC

.my-new-theme.ngn-error
    background-color: #FF0000

.my-new-theme.ngn-success
    background-color: #00CC00

.my-new-theme.ngn-warn
    background-color: #FF9900

.my-new-theme.ngn-grimace
    background-color: #660099
\end{DoxyCode}


\subparagraph*{Custom Styles}

The position, size, color, alignment and more are all styled based on the notification\textquotesingle{}s classes and are all specified in the C\+SS file. See the \href{#styles}{\tt style definitions} below to see which classes can be used to override any of the styles within your own application.

\section*{Definitions }

\subsubsection*{Methods}

\paragraph*{set(message, type)}

displays a notification message and sets the formatting/behavioral options for this one notification.
\begin{DoxyItemize}
\item {\bfseries message}\+: {\itshape string} -\/ {\itshape required} -\/ the message to display in your notification.
\item {\bfseries type}\+: {\itshape string} -\/ {\itshape optional} -\/ the type of notification to display.
\begin{DoxyItemize}
\item info $\ast$(default)$\ast$
\item error
\item success
\item warn
\item grimace
\end{DoxyItemize}
\end{DoxyItemize}

\paragraph*{config(options)}

sets default settings for all notifications to take into account when displaying.
\begin{DoxyItemize}
\item {\bfseries options} -\/ {\itshape object} -\/ an object of options that overrides the default settings.
\begin{DoxyItemize}
\item {\bfseries theme}\+: {\itshape string} -\/ {\itshape optional} -\/ sets the theme to use, altering the styles for each notification type.
\begin{DoxyItemize}
\item pure $\ast$(default)$\ast$
\item prime
\item pastel
\item pitchy
\end{DoxyItemize}
\item {\bfseries type}\+: {\itshape string} -\/ {\itshape optional} -\/ sets the default notification type when a type isn\textquotesingle{}t explicitly set.
\begin{DoxyItemize}
\item info $\ast$(default)$\ast$
\item error
\item success
\item warn
\item grimace
\end{DoxyItemize}
\item {\bfseries position}\+: {\itshape string} -\/ {\itshape optional} -\/ sets where the notifications will be displayed at in the app.
\begin{DoxyItemize}
\item bottom $\ast$(default)$\ast$
\item top
\end{DoxyItemize}
\item {\bfseries duration}\+: {\itshape integer} -\/ {\itshape optional} -\/ the duration the notification stays visible to the user, in milliseconds.
\item {\bfseries sticky}\+: {\itshape bool} -\/ {\itshape optional} -\/ determines whether or not the message will fade at the end of the duration or if the message will persist until the user dismisses it themselves. when true, duration will not be set, even if it has a value. $\ast$(false by default)$\ast$.
\item {\bfseries button}\+: {\itshape bool} -\/ {\itshape optional} -\/ determines whether or not the dismiss button will show on sticky notifications. when true, the button will display. when false, the button wil not display. sticky must be set to true in order to control the visibility of the dismiss button. $\ast$(true by default)$\ast$.
\end{DoxyItemize}
\end{DoxyItemize}

\paragraph*{dismiss()}

manually dismisses any sticky notifications that may still be set.

\paragraph*{add\+Type(name, class)}

allows a dev to create a new type of notification to use in their app.
\begin{DoxyItemize}
\item {\bfseries name}\+: {\itshape string} -\/ {\itshape required} -\/ the name used to trigger this notification type in the {\itshape set} method.
\item {\bfseries class}\+: {\itshape string} -\/ {\itshape required} -\/ the class used to target this type in the stylesheet.
\end{DoxyItemize}

\paragraph*{add\+Theme(name, class)}

allows a dev to create a whole new set of styles for each notification type.
\begin{DoxyItemize}
\item {\bfseries name}\+: {\itshape string} -\/ {\itshape required} -\/ the name used when setting the theme in the {\itshape config} object.
\item {\bfseries class}\+: {\itshape string} -\/ {\itshape required} -\/ the class used to target this theme in the stylesheet.
\end{DoxyItemize}

\subsubsection*{Styles}


\begin{DoxyItemize}
\item {\bfseries primary}\+: the class that\textquotesingle{}s present on every notification and controls all of the primary styles.
\begin{DoxyItemize}
\item $\ast$.ngn$\ast$
\end{DoxyItemize}
\item {\bfseries position}\+: purely responsible for where notifications are displayed. {\itshape default is set to bottom, one is present on every notification.}
\begin{DoxyItemize}
\item $\ast$.ngn-\/top$\ast$
\item $\ast$.ngn-\/bottom$\ast$
\end{DoxyItemize}
\item {\bfseries types}\+: default classes for setting each notification type\textquotesingle{}s background color. {\itshape default is set to info, one is present on every notification.}
\begin{DoxyItemize}
\item $\ast$.ngn-\/info$\ast$
\item $\ast$.ngn-\/error$\ast$
\item $\ast$.ngn-\/success$\ast$
\item $\ast$.ngn-\/warn$\ast$
\item $\ast$.ngn-\/grimace$\ast$
\end{DoxyItemize}
\item {\bfseries themes}\+: theme specific classes that are chained together with type classes to override default background colors. {\itshape not always present, default excludes all of these.}
\begin{DoxyItemize}
\item $\ast$.ngn-\/prime$\ast$
\item $\ast$.ngn-\/pastel$\ast$
\item $\ast$.ngn-\/pitchy$\ast$
\end{DoxyItemize}
\item {\bfseries sticky}\+: styles responsible for displaying the dismissal button for sticky notifications.
\begin{DoxyItemize}
\item $\ast$.ngn-\/sticky$\ast$ -\/ displays the dismissal button when sticky is enabled.
\item $\ast$.ngn-\/dismiss$\ast$ -\/ styles the dismissal button.
\end{DoxyItemize}
\end{DoxyItemize}

\section*{Development }

If you\textquotesingle{}ve forked or cloned the project and would like to make any sort of adjustments, there are few items to make note of. First, your system will need to have the following bits in place\+:


\begin{DoxyItemize}
\item Node \& N\+PM
\item Grunt
\item Ruby
\item Sass
\end{DoxyItemize}

Second, there are a few grunt tasks that you\textquotesingle{}ll be able to leverage to help validate and prepare your changes for use.

You can fire off a {\ttfamily grunt} or {\ttfamily grunt build} command manually at any time to lint, minify, and setup your demo (built in the \+\_\+gh-\/pages dir) for testing.


\begin{DoxyCode}
grunt (or grunt build)
\end{DoxyCode}


Also, you can run {\ttfamily grunt dev} to lint, minify, and prep your demo for testing. Once the build is complete, it\textquotesingle{}ll also fire off a {\ttfamily watch} so that any changes that are made to the the sass, js, and demo files will automatically trigger the build script to update your project.


\begin{DoxyCode}
grunt dev
\end{DoxyCode}


To run through the configured unit tests, you can run {\ttfamily grunt test}. This will fire off a series of tests that check that all default options are set correctly, all configurable options are able to be set correctly, and that all methods carry out the functionality that they\textquotesingle{}re supposed to. These tests should let you know if any of the updates that you\textquotesingle{}ve made have negatively effected any preexisting functionality. Also, when the tests complete, there will be a test coverage report generated and stored in the {\ttfamily coverage} directory.


\begin{DoxyCode}
grunt test
\end{DoxyCode}


Next, you\textquotesingle{}ll want to do all of your development within three locations. If you add changes anywhere else, they\textquotesingle{}re likely to be overwritten during the build process. These locations are\+:

{\ttfamily src/scripts/ng-\/notify.\+js} -\/ for any script modifications.

{\ttfamily src/styles/ng-\/notify.\+sass} -\/ for any style modifications.

{\ttfamily demo/$\ast$} -\/ for any modifications to the demo.

Lastly, once you\textquotesingle{}ve made your changes and run through the appropriate grunt tasks, your changes should be baked and ready for you to consume -\/ located in the {\ttfamily dist} directory as minified js and css files. 