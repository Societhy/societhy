\begin{quote}
Terminal string styling done right \end{quote}


\href{https://travis-ci.org/sindresorhus/chalk}{\tt } 

\href{https://github.com/Marak/colors.js}{\tt colors.\+js} is currently the most popular string styling module, but it has serious deficiencies like extending String.\+prototype which causes all kinds of \href{https://github.com/yeoman/yo/issues/68}{\tt problems}. Although there are other ones, they either do too much or not enough.

{\bfseries Chalk is a clean and focused alternative.}



\subsection*{Why}


\begin{DoxyItemize}
\item Highly performant
\item Doesn\textquotesingle{}t extend String.\+prototype
\item Expressive A\+PI
\item Ability to nest styles
\item Clean and focused
\item Auto-\/detects color support
\item Actively maintained
\item \href{https://npmjs.org/browse/depended/chalk}{\tt Used by 1000+ modules}
\end{DoxyItemize}

\subsection*{Install}


\begin{DoxyCode}
$ npm install --save chalk
\end{DoxyCode}


\subsection*{Usage}

Chalk comes with an easy to use composable A\+PI where you just chain and nest the styles you want.


\begin{DoxyCode}
var chalk = require('chalk');

// style a string
console.log(  chalk.blue('Hello world!')  );

// combine styled and normal strings
console.log(  chalk.blue('Hello'), 'World' + chalk.red('!')  );

// compose multiple styles using the chainable API
console.log(  chalk.blue.bgRed.bold('Hello world!')  );

// pass in multiple arguments
console.log(  chalk.blue('Hello', 'World!', 'Foo', 'bar', 'biz', 'baz')  );

// nest styles
console.log(  chalk.red('Hello', chalk.underline.bgBlue('world') + '!')  );

// nest styles of the same type even (color, underline, background)
console.log(  chalk.green('I am a green line ' + chalk.blue('with a blue substring') + ' that becomes green
       again!')  );
\end{DoxyCode}


Easily define your own themes.


\begin{DoxyCode}
var chalk = require('chalk');
var error = chalk.bold.red;
console.log(error('Error!'));
\end{DoxyCode}


Take advantage of console.\+log \href{http://nodejs.org/docs/latest/api/console.html#console_console_log_data}{\tt string substitution}.


\begin{DoxyCode}
var name = 'Sindre';
console.log(chalk.green('Hello %s'), name);
//=> Hello Sindre
\end{DoxyCode}


\subsection*{A\+PI}

\subsubsection*{chalk.{\ttfamily $<$style$>$\mbox{[}.$<$style$>$...\mbox{]}(string, \mbox{[}string...\mbox{]})}}

Example\+: `chalk.red.\+bold.\+underline(\textquotesingle{}Hello\textquotesingle{}, \textquotesingle{}world\textquotesingle{});`

Chain \href{#styles}{\tt styles} and call the last one as a method with a string argument. Order doesn\textquotesingle{}t matter.

Multiple arguments will be separated by space.

\subsubsection*{chalk.\+enabled}

Color support is automatically detected, but you can override it.

\subsubsection*{chalk.\+supports\+Color}

Detect whether the terminal \href{https://github.com/sindresorhus/supports-color}{\tt supports color}.

Can be overridden by the user with the flags {\ttfamily -\/-\/color} and {\ttfamily -\/-\/no-\/color}.

Used internally and handled for you, but exposed for convenience.

\subsubsection*{chalk.\+styles}

Exposes the styles as \href{https://github.com/sindresorhus/ansi-styles}{\tt A\+N\+SI escape codes}.

Generally not useful, but you might need just the {\ttfamily .open} or {\ttfamily .close} escape code if you\textquotesingle{}re mixing externally styled strings with yours.


\begin{DoxyCode}
var chalk = require('chalk');

console.log(chalk.styles.red);
//=> \{open: '\(\backslash\)u001b[31m', close: '\(\backslash\)u001b[39m'\}

console.log(chalk.styles.red.open + 'Hello' + chalk.styles.red.close);
\end{DoxyCode}


\subsubsection*{chalk.\+has\+Color(string)}

Check whether a string \href{https://github.com/sindresorhus/has-ansi}{\tt has color}.

\subsubsection*{chalk.\+strip\+Color(string)}

\href{https://github.com/sindresorhus/strip-ansi}{\tt Strip color} from a string.

Can be useful in combination with {\ttfamily .supports\+Color} to strip color on externally styled text when it\textquotesingle{}s not supported.

Example\+:


\begin{DoxyCode}
var chalk = require('chalk');
var styledString = getText();

if (!chalk.supportsColor) \{
    styledString = chalk.stripColor(styledString);
\}
\end{DoxyCode}


\subsection*{Styles}

\subsubsection*{General}


\begin{DoxyItemize}
\item {\ttfamily reset}
\item {\ttfamily bold}
\item {\ttfamily dim}
\item {\ttfamily italic} $\ast$(not widely supported)$\ast$
\item {\ttfamily underline}
\item {\ttfamily inverse}
\item {\ttfamily hidden}
\item {\ttfamily strikethrough} $\ast$(not widely supported)$\ast$
\end{DoxyItemize}

\subsubsection*{Text colors}


\begin{DoxyItemize}
\item {\ttfamily black}
\item {\ttfamily red}
\item {\ttfamily green}
\item {\ttfamily yellow}
\item {\ttfamily blue}
\item {\ttfamily magenta}
\item {\ttfamily cyan}
\item {\ttfamily white}
\item {\ttfamily gray}
\end{DoxyItemize}

\subsubsection*{Background colors}


\begin{DoxyItemize}
\item {\ttfamily bg\+Black}
\item {\ttfamily bg\+Red}
\item {\ttfamily bg\+Green}
\item {\ttfamily bg\+Yellow}
\item {\ttfamily bg\+Blue}
\item {\ttfamily bg\+Magenta}
\item {\ttfamily bg\+Cyan}
\item {\ttfamily bg\+White}
\end{DoxyItemize}

\subsection*{License}

M\+IT © \href{http://sindresorhus.com}{\tt Sindre Sorhus} 