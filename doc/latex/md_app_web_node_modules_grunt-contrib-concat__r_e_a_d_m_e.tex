\begin{quote}
Concatenate files. \end{quote}


\subsection*{Getting Started}

This plugin requires Grunt {\ttfamily $>$=0.\+4.\+0}

If you haven\textquotesingle{}t used \href{http://gruntjs.com/}{\tt Grunt} before, be sure to check out the \href{http://gruntjs.com/getting-started}{\tt Getting Started} guide, as it explains how to create a \href{http://gruntjs.com/sample-gruntfile}{\tt Gruntfile} as well as install and use Grunt plugins. Once you\textquotesingle{}re familiar with that process, you may install this plugin with this command\+:


\begin{DoxyCode}
npm install grunt-contrib-concat --save-dev
\end{DoxyCode}


Once the plugin has been installed, it may be enabled inside your Gruntfile with this line of Java\+Script\+:


\begin{DoxyCode}
grunt.loadNpmTasks('grunt-contrib-concat');
\end{DoxyCode}


\subsection*{Concat task}

{\itshape Run this task with the {\ttfamily grunt concat} command.}

Task targets, files and options may be specified according to the Grunt \href{http://gruntjs.com/configuring-tasks}{\tt Configuring tasks} guide.

\subsubsection*{Options}

\paragraph*{separator}

Type\+: {\ttfamily String} Default\+: {\ttfamily grunt.\+util.\+linefeed}

Concatenated files will be joined on this string. If you\textquotesingle{}re post-\/processing concatenated Java\+Script files with a minifier, you may need to use a semicolon `\textquotesingle{};\textquotesingle{}` as the separator.

\paragraph*{banner}

Type\+: {\ttfamily String} Default\+: empty string

This string will be prepended to the beginning of the concatenated output. It is processed using \href{https://github.com/gruntjs/grunt-docs/blob/master/grunt.template.md#grunttemplateprocess}{\tt grunt.\+template.\+process}, using the default options.

\+\_\+(Default processing options are explained in the \href{https://github.com/gruntjs/grunt-docs/blob/master/grunt.template.md#grunttemplateprocess}{\tt grunt.\+template.\+process} documentation)\+\_\+

\paragraph*{footer}

Type\+: {\ttfamily String} Default\+: empty string

This string will be appended to the end of the concatenated output. It is processed using \href{https://github.com/gruntjs/grunt-docs/blob/master/grunt.template.md#grunttemplateprocess}{\tt grunt.\+template.\+process}, using the default options.

\+\_\+(Default processing options are explained in the \href{https://github.com/gruntjs/grunt-docs/blob/master/grunt.template.md#grunttemplateprocess}{\tt grunt.\+template.\+process} documentation)\+\_\+

\paragraph*{strip\+Banners}

Type\+: {\ttfamily Boolean} {\ttfamily Object} Default\+: {\ttfamily false}

Strip Java\+Script banner comments from source files.


\begin{DoxyItemize}
\item {\ttfamily false} -\/ No comments are stripped.
\item {\ttfamily true} -\/ {\ttfamily /$\ast$ ... $\ast$/} block comments are stripped, but {\itshape N\+OT} {\ttfamily /$\ast$! ... $\ast$/} comments.
\item {\ttfamily options} object\+:
\begin{DoxyItemize}
\item By default, behaves as if {\ttfamily true} were specified.
\item {\ttfamily block} -\/ If true, {\itshape all} block comments are stripped.
\item {\ttfamily line} -\/ If true, any contiguous {\itshape leading} {\ttfamily //} line comments are stripped.
\end{DoxyItemize}
\end{DoxyItemize}

\paragraph*{process}

Type\+: {\ttfamily Boolean} {\ttfamily Object} {\ttfamily Function} Default\+: {\ttfamily false}

Process source files before concatenating, either as https\+://github.com/gruntjs/grunt-\/docs/blob/master/grunt.\+template.\+md \char`\"{}templates\char`\"{} or with a custom function.


\begin{DoxyItemize}
\item {\ttfamily false} -\/ No processing will occur.
\item {\ttfamily true} -\/ Process source files using \href{https://github.com/gruntjs/grunt-docs/blob/master/grunt.template.md#grunttemplateprocess}{\tt grunt.\+template.\+process} defaults.
\item {\ttfamily data} object -\/ Process source files using \href{https://github.com/gruntjs/grunt-docs/blob/master/grunt.template.md#grunttemplateprocess}{\tt grunt.\+template.\+process}, using the specified options.
\item {\ttfamily function(src, filepath)} -\/ Process source files using the given function, called once for each file. The returned value will be used as source code.
\end{DoxyItemize}

\+\_\+(Default processing options are explained in the \href{https://github.com/gruntjs/grunt-docs/blob/master/grunt.template.md#grunttemplateprocess}{\tt grunt.\+template.\+process} documentation)\+\_\+

\paragraph*{source\+Map}

Type\+: {\ttfamily Boolean} Default\+: {\ttfamily false}

Set to true to create a source map. The source map will be created alongside the destination file, and share the same file name with the {\ttfamily .map} extension appended to it.

\paragraph*{source\+Map\+Name}

Type\+: {\ttfamily String} {\ttfamily Function} Default\+: {\ttfamily undefined}

To customize the name or location of the generated source map, pass a string to indicate where to write the source map to. If a function is provided, the concat destination is passed as the argument and the return value will be used as the file name.

\paragraph*{source\+Map\+Style}

Type\+: {\ttfamily String} Default\+: {\ttfamily embed}

Determines the type of source map that is generated. The default value, {\ttfamily embed}, places the content of the sources directly into the map. {\ttfamily link} will reference the original sources in the map as links. {\ttfamily inline} will store the entire map as a data U\+RI in the destination file.

\subsubsection*{Usage Examples}

\paragraph*{Concatenating with a custom separator}

In this example, running {\ttfamily grunt concat\+:dist} (or {\ttfamily grunt concat} because {\ttfamily concat} is a \href{http://gruntjs.com/creating-tasks#multi-tasks}{\tt multi task}) will concatenate the three specified source files (in order), joining files with {\ttfamily ;} and writing the output to {\ttfamily dist/built.\+js}.


\begin{DoxyCode}
// Project configuration.
grunt.initConfig(\{
  concat: \{
    options: \{
      separator: ';',
    \},
    dist: \{
      src: ['src/intro.js', 'src/project.js', 'src/outro.js'],
      dest: 'dist/built.js',
    \},
  \},
\});
\end{DoxyCode}


\paragraph*{Banner comments}

In this example, running {\ttfamily grunt concat\+:dist} will first strip any preexisting banner comment from the {\ttfamily src/project.\+js} file, then concatenate the result with a newly-\/generated banner comment, writing the output to {\ttfamily dist/built.\+js}.

This generated banner will be the contents of the {\ttfamily banner} template string interpolated with the config object. In this case, those properties are the values imported from the {\ttfamily package.\+json} file (which are available via the {\ttfamily pkg} config property) plus today\textquotesingle{}s date.

{\itshape Note\+: you don\textquotesingle{}t have to use an external J\+S\+ON file. It\textquotesingle{}s also valid to create the {\ttfamily pkg} object inline in the config. That being said, if you already have a J\+S\+ON file, you might as well reference it.}


\begin{DoxyCode}
// Project configuration.
grunt.initConfig(\{
  pkg: grunt.file.readJSON('package.json'),
  concat: \{
    options: \{
      stripBanners: true,
      banner: '/*! <%= pkg.name %> - v<%= pkg.version %> - ' +
        '<%= grunt.template.today("yyyy-mm-dd") %> */',
    \},
    dist: \{
      src: ['src/project.js'],
      dest: 'dist/built.js',
    \},
  \},
\});
\end{DoxyCode}


\paragraph*{Multiple targets}

In this example, running {\ttfamily grunt concat} will build two separate files. One \char`\"{}basic\char`\"{} version, with the main file essentially just copied to {\ttfamily dist/basic.\+js}, and another \char`\"{}with\+\_\+extras\char`\"{} concatenated version written to {\ttfamily dist/with\+\_\+extras.\+js}.

While each concat target can be built individually by running {\ttfamily grunt concat\+:basic} or {\ttfamily grunt concat\+:extras}, running {\ttfamily grunt concat} will build all concat targets. This is because {\ttfamily concat} is a \href{http://gruntjs.com/creating-tasks#multi-tasks}{\tt multi task}.


\begin{DoxyCode}
// Project configuration.
grunt.initConfig(\{
  concat: \{
    basic: \{
      src: ['src/main.js'],
      dest: 'dist/basic.js',
    \},
    extras: \{
      src: ['src/main.js', 'src/extras.js'],
      dest: 'dist/with\_extras.js',
    \},
  \},
\});
\end{DoxyCode}


\paragraph*{Multiple files per target}

Like the previous example, in this example running {\ttfamily grunt concat} will build two separate files. One \char`\"{}basic\char`\"{} version, with the main file essentially just copied to {\ttfamily dist/basic.\+js}, and another \char`\"{}with\+\_\+extras\char`\"{} concatenated version written to {\ttfamily dist/with\+\_\+extras.\+js}.

This example differs in that both files are built under the same target.

Using the {\ttfamily files} object, you can have list any number of source-\/destination pairs.


\begin{DoxyCode}
// Project configuration.
grunt.initConfig(\{
  concat: \{
    basic\_and\_extras: \{
      files: \{
        'dist/basic.js': ['src/main.js'],
        'dist/with\_extras.js': ['src/main.js', 'src/extras.js'],
      \},
    \},
  \},
\});
\end{DoxyCode}


\paragraph*{Dynamic filenames}

Filenames can be generated dynamically by using {\ttfamily $<$\%= \%$>$} delimited underscore templates as filenames.

In this example, running {\ttfamily grunt concat\+:dist} generates a destination file whose name is generated from the {\ttfamily name} and {\ttfamily version} properties of the referenced {\ttfamily package.\+json} file (via the {\ttfamily pkg} config property).


\begin{DoxyCode}
// Project configuration.
grunt.initConfig(\{
  pkg: grunt.file.readJSON('package.json'),
  concat: \{
    dist: \{
      src: ['src/main.js'],
      dest: 'dist/<%= pkg.name %>-<%= pkg.version %>.js',
    \},
  \},
\});
\end{DoxyCode}


\paragraph*{Advanced dynamic filenames}

In this more involved example, running {\ttfamily grunt concat} will build two separate files (because {\ttfamily concat} is a \href{http://gruntjs.com/creating-tasks#multi-tasks}{\tt multi task}). The destination file paths will be expanded dynamically based on the specified templates, recursively if necessary.

For example, if the {\ttfamily package.\+json} file contained {\ttfamily \{\char`\"{}name\char`\"{}\+: \char`\"{}awesome\char`\"{}, \char`\"{}version\char`\"{}\+: \char`\"{}1.\+0.\+0\char`\"{}\}}, the files {\ttfamily dist/awesome/1.\+0.\+0/basic.js} and {\ttfamily dist/awesome/1.\+0.\+0/with\+\_\+extras.js} would be generated.


\begin{DoxyCode}
// Project configuration.
grunt.initConfig(\{
  pkg: grunt.file.readJSON('package.json'),
  dirs: \{
    src: 'src/files',
    dest: 'dist/<%= pkg.name %>/<%= pkg.version %>',
  \},
  concat: \{
    basic: \{
      src: ['<%= dirs.src %>/main.js'],
      dest: '<%= dirs.dest %>/basic.js',
    \},
    extras: \{
      src: ['<%= dirs.src %>/main.js', '<%= dirs.src %>/extras.js'],
      dest: '<%= dirs.dest %>/with\_extras.js',
    \},
  \},
\});
\end{DoxyCode}


\paragraph*{Invalid or Missing Files Warning}

If you would like the {\ttfamily concat} task to warn if a given file is missing or invalid be sure to set {\ttfamily nonull} to {\ttfamily true}\+:


\begin{DoxyCode}
grunt.initConfig(\{
  concat: \{
    missing: \{
      src: ['src/invalid\_or\_missing\_file'],
      dest: 'compiled.js',
      nonull: true,
    \},
  \},
\});
\end{DoxyCode}


See \href{http://gruntjs.com/configuring-tasks#files}{\tt configuring files for a task} for how to configure file globbing in Grunt.

\paragraph*{Custom process function}

If you would like to do any custom processing before concatenating, use a custom process function\+:


\begin{DoxyCode}
grunt.initConfig(\{
  concat: \{
    dist: \{
      options: \{
        // Replace all 'use strict' statements in the code with a single one at the top
        banner: "'use strict';\(\backslash\)n",
        process: function(src, filepath) \{
          return '// Source: ' + filepath + '\(\backslash\)n' +
            src.replace(/(^|\(\backslash\)n)[ \(\backslash\)t]*('use strict'|"use strict");?\(\backslash\)s*/g, '$1');
        \},
      \},
      files: \{
        'dist/built.js': ['src/project.js'],
      \},
    \},
  \},
\});
\end{DoxyCode}


\subsection*{Release History}


\begin{DoxyItemize}
\item 2015-\/02-\/20   v0.5.\+1   \+Fix path issues with Source Maps on Windows.
\item 2014-\/07-\/19   v0.5.\+0   \+Adds source\+Map option.
\item 2014-\/03-\/21   v0.4.\+0   \+R\+E\+A\+D\+ME updates. Output updates.
\item 2013-\/04-\/25   v0.3.\+0   \+Add option to process files with a custom function.
\item 2013-\/04-\/08   v0.2.\+0   \+Don\textquotesingle{}t normalize separator to allow user to set LF even on a Windows environment.
\item 2013-\/02-\/22   v0.1.\+3   \+Support footer option.
\item 2013-\/02-\/15   v0.1.\+2   \+First official release for Grunt 0.\+4.\+0.
\item 2013-\/01-\/18   v0.1.\+2rc6   \+Updating grunt/gruntplugin dependencies to rc6. Changing in-\/development grunt/gruntplugin dependency versions from tilde version ranges to specific versions.
\item 2013-\/01-\/09   v0.1.\+2rc5   \+Updating to work with grunt v0.\+4.\+0rc5. Switching back to this.\+files api.
\item 2012-\/11-\/13   v0.1.\+1   \+Switch to this.\+file api internally.
\item 2012-\/10-\/03   v0.1.\+0   \+Work in progress, not yet officially released. 


\end{DoxyItemize}

Task submitted by \href{http://benalman.com/}{\tt \char`\"{}\+Cowboy\char`\"{} Ben Alman}

{\itshape This file was generated on Fri Feb 20 2015 10\+:39\+:55.} 