This is the Java\+Script S\+DK for \href{https://oauth.io}{\tt O\+Auth.\+io}. O\+Auth.\+io allows you to integrate {\bfseries 100+ providers} really easily in your web app, without worrying about each provider\textquotesingle{}s O\+Auth specific implementation.

\section*{Installation }

\subsection*{Getting the S\+DK }

To get the S\+DK, you can \+:


\begin{DoxyItemize}
\item download the zip file from this repository
\item get it via Bower or npm (for browserify)
\end{DoxyItemize}

{\bfseries Zip file}

Just copy the dist/oauth.\+js or dist/oauth.\+min.\+js to your project.

{\bfseries Bower}


\begin{DoxyCode}
$ bower install oauthio-web
\end{DoxyCode}


{\bfseries npm for browserify}


\begin{DoxyCode}
$ npm install oauthio-web
\end{DoxyCode}


\subsection*{Integrating in your project }

In the {\ttfamily $<$head$>$} of your H\+T\+ML, include O\+Auth.\+js

{\ttfamily $<$script src=\char`\"{}/path/to/oauth.\+js\char`\"{}$>$$<$/script$>$}

In your Javascript, add this line to initialize O\+Auth.\+js

`\+O\+Auth.initialize(\textquotesingle{}your\+\_\+app\+\_\+public\+\_\+key\textquotesingle{});`

\section*{Usage }

To connect your user using facebook, 2 methods\+:

\subsection*{Popup mode }


\begin{DoxyCode}
//Using popup (option 1)
OAuth.popup('facebook')
.done(function(result) \{
  //use result.access\_token in your API request 
  //or use result.get|post|put|del|patch|me methods (see below)
\})
.fail(function (err) \{
  //handle error with err
\});
\end{DoxyCode}


\subsection*{Redirection mode }


\begin{DoxyCode}
//Using redirection (option 2)
OAuth.redirect('facebook', "callback/url");
\end{DoxyCode}


In callback url \+:


\begin{DoxyCode}
OAuth.callback('facebook')
.done(function(result) \{
    //use result.access\_token in your API request
    //or use result.get|post|put|del|patch|me methods (see below)
\})
.fail(function (err) \{
    //handle error with err
\});
\end{DoxyCode}


\subsection*{Making requests }

You can make requests to the provider\textquotesingle{}s A\+PI manually with the access token you got from the {\ttfamily popup} or {\ttfamily callback} methods, or use the request methods stored in the {\ttfamily result} object.

{\bfseries G\+ET Request}

To make a G\+ET request, you have to call the {\ttfamily result.\+get} method like this \+:


\begin{DoxyCode}
//Let's say the /me endpoint on the provider API returns a JSON object
//with the field "name" containing the name "John Doe"
OAuth.popup(provider)
.done(function(result) \{
    result.get('/me')
    .done(function (response) \{
        //this will display "John Doe" in the console
        console.log(response.name);
    \})
    .fail(function (err) \{
        //handle error with err
    \});
\})
.fail(function (err) \{
    //handle error with err
\});
\end{DoxyCode}


{\bfseries P\+O\+ST Request}

To make a P\+O\+ST request, you have to call the {\ttfamily result.\+post} method like this \+:


\begin{DoxyCode}
//Let's say the /message endpoint on the provider waits for
//a POST request containing the fields "user\_id" and "content"
//and returns the field "id" containing the id of the sent message 
OAuth.popup(provider)
.done(function(result) \{
    result.post('/message', \{
        data: \{
            user\_id: 93,
            content: 'Hello Mr. 93 !'
        \}
    \})
    .done(function (response) \{
        //this will display the id of the message in the console
        console.log(response.id);
    \})
    .fail(function (err) \{
        //handle error with err
    \});
\})
.fail(function (err) \{
    //handle error with err
\});
\end{DoxyCode}


{\bfseries P\+UT Request}

To make a P\+UT request, you have to call the {\ttfamily result.\+post} method like this \+:


\begin{DoxyCode}
//Let's say the /profile endpoint on the provider waits for
//a PUT request to update the authenticated user's profile 
//containing the field "name" and returns the field "name" 
//containing the new name
OAuth.popup(provider)
.done(function(result) \{
    result.put('/message', \{
        data: \{
            name: "John Williams Doe III"
        \}
    \})
    .done(function (response) \{
        //this will display the new name in the console
        console.log(response.name);
    \})
    .fail(function (err) \{
        //handle error with err
    \});
\})
.fail(function (err) \{
    //handle error with err
\});
\end{DoxyCode}


{\bfseries P\+A\+T\+CH Request}

To make a P\+A\+T\+CH request, you have to call the {\ttfamily result.\+patch} method like this \+:


\begin{DoxyCode}
//Let's say the /profile endpoint on the provider waits for
//a PATCH request to update the authenticated user's profile 
//containing the field "name" and returns the field "name" 
//containing the new name
OAuth.popup(provider)
.done(function(result) \{
    result.patch('/message', \{
        data: \{
            name: "John Williams Doe III"
        \}
    \})
    .done(function (response) \{
        //this will display the new name in the console
        console.log(response.name);
    \})
    .fail(function (err) \{
        //handle error with err
    \});
\})
.fail(function (err) \{
    //handle error with err
\});
\end{DoxyCode}


{\bfseries D\+E\+L\+E\+TE Request}

To make a D\+E\+L\+E\+TE request, you have to call the {\ttfamily result.\+del} method like this \+:


\begin{DoxyCode}
//Let's say the /picture?id=picture\_id endpoint on the provider waits for
//a DELETE request to delete a picture with the id "84"
//and returns true or false depending on the user's rights on the picture
OAuth.popup(provider)
.done(function(result) \{
    result.del('/picture?id=84')
    .done(function (response) \{
        //this will display true if the user was authorized to delete
        //the picture
        console.log(response);
    \})
    .fail(function (err) \{
        //handle error with err
    \});
\})
.fail(function (err) \{
    //handle error with err
\});
\end{DoxyCode}


{\bfseries Me() Request}

The {\ttfamily me()} request is an O\+Auth.\+io feature that allows you, when the provider is supported, to retrieve a unified object describing the authenticated user. That can be very useful when you need to login a user via several providers, but don\textquotesingle{}t want to handle a different response each time.

To use the {\ttfamily me()} feature, do like the following (the example works for Facebook, Github, Twitter and many other providers in this case) \+:


\begin{DoxyCode}
//provider can be 'facebook', 'twitter', 'github', or any supported
//provider that contain the fields 'firstname' and 'lastname' 
//or an equivalent (e.g. "FirstName" or "first-name")
var provider = 'facebook';

OAuth.popup(provider)
.done(function(result) \{
    result.me()
    .done(function (response) \{
        console.log('Firstname: ', response.firstname);
        console.log('Lastname: ', response.lastname);
    \})
    .fail(function (err) \{
        //handle error with err
    \});
\})
.fail(function (err) \{
    //handle error with err
\});
\end{DoxyCode}


{\itshape Filtering the results}

You can filter the results of the {\ttfamily me()} method by passing an array of fields you need \+:


\begin{DoxyCode}
//...
result.me(['firstname', 'lastname', 'email'/*, ...*/])
//...
\end{DoxyCode}


\section*{Contributing }

You are welcome to fork and make pull requests. We will be happy to review them and include them in the code if they bring nice improvements \+:)

\section*{Testing the S\+DK }

To test the S\+DK, you first need to install the npm modules {\ttfamily jasmine-\/node} and {\ttfamily istanbul} (to get the tests coverage) \+:


\begin{DoxyCode}
$ sudo npm install -g jasmine-node@2.0.0 istanbul
\end{DoxyCode}


Then you can run the testsuite from the S\+DK root directory \+:


\begin{DoxyCode}
$ jasmine-node --verbose tests/unit/spec
\end{DoxyCode}


Once you\textquotesingle{}ve installed {\ttfamily istanbul}, you can run the following command to get coverage information \+:


\begin{DoxyCode}
$ npm test
\end{DoxyCode}


The coverage report is generated in the {\ttfamily coverage} folder. You can have a nice H\+T\+ML render of the report in {\ttfamily coverage/lcof-\/report/index.\+html}

\section*{License }

This S\+DK is published under the Apache2 License.

More information in \href{http://oauth.io/#/docs}{\tt oauth.\+io documentation} 