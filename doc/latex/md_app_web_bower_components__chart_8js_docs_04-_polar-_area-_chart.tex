

 title\+: Polar Area Chart \subsection*{anchor\+: polar-\/area-\/chart }

\subsubsection*{Introduction}

Polar area charts are similar to pie charts, but each segment has the same angle -\/ the radius of the segment differs depending on the value.

This type of chart is often useful when we want to show a comparison data similar to a pie chart, but also show a scale of values for context.

 $<$canvas width=\char`\"{}250\char`\"{} height=\char`\"{}125\char`\"{}$>$$<$/canvas$>$ 

\subsubsection*{Example usage}


\begin{DoxyCode}
new Chart(ctx).PolarArea(data, options);
\end{DoxyCode}


\subsubsection*{Data structure}


\begin{DoxyCode}
var data = [
    \{
        value: 300,
        color:"#F7464A",
        highlight: "#FF5A5E",
        label: "Red"
    \},
    \{
        value: 50,
        color: "#46BFBD",
        highlight: "#5AD3D1",
        label: "Green"
    \},
    \{
        value: 100,
        color: "#FDB45C",
        highlight: "#FFC870",
        label: "Yellow"
    \},
    \{
        value: 40,
        color: "#949FB1",
        highlight: "#A8B3C5",
        label: "Grey"
    \},
    \{
        value: 120,
        color: "#4D5360",
        highlight: "#616774",
        label: "Dark Grey"
    \}

];
\end{DoxyCode}
 As you can see, for the chart data you pass in an array of objects, with a value and a colour. The value attribute should be a number, while the color attribute should be a string. Similar to C\+SS, for this string you can use H\+EX notation, R\+GB, R\+G\+BA or H\+SL.

\subsubsection*{Chart options}

These are the customisation options specific to Polar Area charts. These options are merged with the \href{#getting-started-global-chart-configuration}{\tt global chart configuration options}, and form the options of the chart.


\begin{DoxyCode}
\{
    //Boolean - Show a backdrop to the scale label
    scaleShowLabelBackdrop : true,

    //String - The colour of the label backdrop
    scaleBackdropColor : "rgba(255,255,255,0.75)",

    // Boolean - Whether the scale should begin at zero
    scaleBeginAtZero : true,

    //Number - The backdrop padding above & below the label in pixels
    scaleBackdropPaddingY : 2,

    //Number - The backdrop padding to the side of the label in pixels
    scaleBackdropPaddingX : 2,

    //Boolean - Show line for each value in the scale
    scaleShowLine : true,

    //Boolean - Stroke a line around each segment in the chart
    segmentShowStroke : true,

    //String - The colour of the stroke on each segement.
    segmentStrokeColor : "#fff",

    //Number - The width of the stroke value in pixels
    segmentStrokeWidth : 2,

    //Number - Amount of animation steps
    animationSteps : 100,

    //String - Animation easing effect.
    animationEasing : "easeOutBounce",

    //Boolean - Whether to animate the rotation of the chart
    animateRotate : true,

    //Boolean - Whether to animate scaling the chart from the centre
    animateScale : false,
    \{% raw %\}
    //String - A legend template
    legendTemplate : "<ul class=\(\backslash\)"<%=name.toLowerCase()%>-legend\(\backslash\)"><% for (var i=0; i<segments.length;
       i++)\{%><li><span
       style=\(\backslash\)"background-color:<%=segments[i].fillColor%>\(\backslash\)"></span><%if(segments[i].label)\{%><%=segments[i].label%><%\}%></li><%\}%></ul>"
    \{% endraw %\}
\}
\end{DoxyCode}


You can override these for your {\ttfamily Chart} instance by passing a second argument into the {\ttfamily Polar\+Area} method as an object with the keys you want to override.

For example, we could have a polar area chart with a black stroke on each segment like so\+:


\begin{DoxyCode}
new Chart(ctx).PolarArea(data, \{
    segmentStrokeColor: "#000000"
\});
// This will create a chart with all of the default options, merged from the global config,
// and the PolarArea chart defaults but this particular instance will have `segmentStrokeColor` set to
       `"#000000"`.
\end{DoxyCode}


We can also change these defaults values for each Polar\+Area type that is created, this object is available at {\ttfamily Chart.\+defaults.\+Polar\+Area}.

\subsubsection*{Prototype methods}

\paragraph*{.get\+Segments\+At\+Event( event )}

Calling {\ttfamily get\+Segments\+At\+Event(event)} on your Chart instance passing an argument of an event, or j\+Query event, will return the segment elements that are at that the same position of that event.


\begin{DoxyCode}
canvas.onclick = function(evt)\{
    var activePoints = myPolarAreaChart.getSegmentsAtEvent(evt);
    // => activePoints is an array of segments on the canvas that are at the same position as the click
       event.
\};
\end{DoxyCode}


This functionality may be useful for implementing D\+OM based tooltips, or triggering custom behaviour in your application.

\paragraph*{.update( )}

Calling {\ttfamily update()} on your Chart instance will re-\/render the chart with any updated values, allowing you to edit the value of multiple existing points, then render those in one animated render loop.


\begin{DoxyCode}
myPolarAreaChart.segments[1].value = 10;
// Would update the first dataset's value of 'Green' to be 10
myPolarAreaChart.update();
// Calling update now animates the position of Green from 50 to 10.
\end{DoxyCode}


\paragraph*{.add\+Data( segment\+Data, index )}

Calling {\ttfamily add\+Data(segment\+Data, index)} on your Chart instance passing an object in the same format as in the constructor. There is an option second argument of \textquotesingle{}index\textquotesingle{}, this determines at what index the new segment should be inserted into the chart.


\begin{DoxyCode}
// An object in the same format as the original data source
myPolarAreaChart.addData(\{
    value: 130,
    color: "#B48EAD",
    highlight: "#C69CBE",
    label: "Purple"
\});
// The new segment will now animate in.
\end{DoxyCode}


\paragraph*{.remove\+Data( index )}

Calling {\ttfamily remove\+Data(index)} on your Chart instance will remove segment at that particular index. If none is provided, it will default to the last segment.


\begin{DoxyCode}
myPolarAreaChart.removeData();
// Other segments will update to fill the empty space left.
\end{DoxyCode}
 