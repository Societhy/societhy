\begin{quote}
Run grunt tasks concurrently \end{quote}


Running slow tasks like Coffee and Sass concurrently can potentially improve your build time significantly. This task is also useful if you need to run multiple blocking tasks like {\ttfamily nodemon} and {\ttfamily watch} at once, as seen in the example config.



This task is similar to grunt-\/parallel, but more focused by leaving out support for shell scripts which results in a leaner config. It also has a smaller dependency size and pads the output of concurrent tasks, as seen above.

\subsection*{Getting Started}

If you haven\textquotesingle{}t used \href{http://gruntjs.com}{\tt grunt} before, be sure to check out the \href{https://github.com/gruntjs/grunt/wiki/Getting-started}{\tt Getting Started} guide, as it explains how to create a \href{https://github.com/gruntjs/grunt/wiki/Getting-started}{\tt gruntfile} as well as install and use grunt plugins. Once you\textquotesingle{}re familiar with that process, install this plugin with this command\+:


\begin{DoxyCode}
npm install grunt-concurrent --save-dev
\end{DoxyCode}


Once the plugin has been installed, it may be enabled inside your Gruntfile with this line of Java\+Script\+:


\begin{DoxyCode}
grunt.loadNpmTasks('grunt-concurrent');
\end{DoxyCode}


{\itshape Tip\+: the \href{https://github.com/sindresorhus/load-grunt-tasks}{\tt load-\/grunt-\/tasks} module makes it easier to load multiple grunt tasks.}

\subsection*{Documentation}

See the \href{Gruntfile.js}{\tt Gruntfile} in this repo for a full example.

Just specify the tasks you want to run concurrently as an array in a target of this task as shown below.

\subsubsection*{Example config}

This will first run the Coffee and Sass tasks at the same time, then the J\+S\+Hint and Mocha tasks at the same time.


\begin{DoxyCode}
grunt.initConfig(\{
    concurrent: \{
        target1: ['coffee', 'sass'],
        target2: ['jshint', 'mocha']
    \}
\});

grunt.loadNpmTasks('grunt-concurrent');
grunt.registerTask('default', ['concurrent:target1', 'concurrent:target2']);
\end{DoxyCode}


\subsection*{Options}

\subsubsection*{limit}

Type\+: {\ttfamily Number} Default\+: Number of C\+PU cores (`require(\textquotesingle{}os\textquotesingle{}).cpus().length`) with a minimum of 2

Limit of how many tasks that are run concurrently.

\subsubsection*{log\+Concurrent\+Output}

Type\+: {\ttfamily Boolean} Default\+: {\ttfamily false}

You can optionally log the output of your concurrent tasks by specifying the {\ttfamily log\+Concurrent\+Output} option. Here is an example config which runs \href{https://github.com/ChrisWren/grunt-nodemon}{\tt grunt-\/nodemon} to launch and monitor a node server and \href{https://github.com/gruntjs/grunt-contrib-watch}{\tt grunt-\/contrib-\/watch} to watch for asset changes all in one terminal tab\+:


\begin{DoxyCode}
grunt.initConfig(\{
    concurrent: \{
        target: \{
            tasks: ['nodemon', 'watch'],
            options: \{
                logConcurrentOutput: true
            \}
        \}
    \}
\});

grunt.loadNpmTasks('grunt-concurrent');
grunt.registerTask('default', ['concurrent:target']);
\end{DoxyCode}


{\itshape Note the output will be messy when combining certain tasks. This option is best used with tasks that don\textquotesingle{}t exit like watch and nodemon to monitor the output of long-\/running concurrent tasks.}

\subsection*{License}

M\+IT © \href{http://sindresorhus.com}{\tt Sindre Sorhus} 