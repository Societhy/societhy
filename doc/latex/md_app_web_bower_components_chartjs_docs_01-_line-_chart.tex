

 title\+: Line Chart \subsection*{anchor\+: line-\/chart }

\subsubsection*{Introduction}

A line chart is a way of plotting data points on a line.

Often, it is used to show trend data, and the comparison of two data sets.

 $<$canvas width=\char`\"{}250\char`\"{} height=\char`\"{}125\char`\"{}$>$$<$/canvas$>$ 

\#\#\#\+Example usage 
\begin{DoxyCode}
var myLineChart = new Chart(ctx).Line(data, options);
\end{DoxyCode}
 \subsubsection*{Data structure}


\begin{DoxyCode}
var data = \{
    labels: ["January", "February", "March", "April", "May", "June", "July"],
    datasets: [
        \{
            label: "My First dataset",
            fillColor: "rgba(220,220,220,0.2)",
            strokeColor: "rgba(220,220,220,1)",
            pointColor: "rgba(220,220,220,1)",
            pointStrokeColor: "#fff",
            pointHighlightFill: "#fff",
            pointHighlightStroke: "rgba(220,220,220,1)",
            data: [65, 59, 80, 81, 56, 55, 40]
        \},
        \{
            label: "My Second dataset",
            fillColor: "rgba(151,187,205,0.2)",
            strokeColor: "rgba(151,187,205,1)",
            pointColor: "rgba(151,187,205,1)",
            pointStrokeColor: "#fff",
            pointHighlightFill: "#fff",
            pointHighlightStroke: "rgba(151,187,205,1)",
            data: [28, 48, 40, 19, 86, 27, 90]
        \}
    ]
\};
\end{DoxyCode}


The line chart requires an array of labels for each of the data points. This is shown on the X axis. The data for line charts is broken up into an array of datasets. Each dataset has a colour for the fill, a colour for the line and colours for the points and strokes of the points. These colours are strings just like C\+SS. You can use R\+G\+BA, R\+GB, H\+EX or H\+SL notation.

The label key on each dataset is optional, and can be used when generating a scale for the chart.

\subsubsection*{Chart options}

These are the customisation options specific to Line charts. These options are merged with the \href{#getting-started-global-chart-configuration}{\tt global chart configuration options}, and form the options of the chart.


\begin{DoxyCode}
\{

    ///Boolean - Whether grid lines are shown across the chart
    scaleShowGridLines : true,

    //String - Colour of the grid lines
    scaleGridLineColor : "rgba(0,0,0,.05)",

    //Number - Width of the grid lines
    scaleGridLineWidth : 1,

    //Boolean - Whether to show horizontal lines (except X axis)
    scaleShowHorizontalLines: true,

    //Boolean - Whether to show vertical lines (except Y axis)
    scaleShowVerticalLines: true,

    //Boolean - Whether the line is curved between points
    bezierCurve : true,

    //Number - Tension of the bezier curve between points
    bezierCurveTension : 0.4,

    //Boolean - Whether to show a dot for each point
    pointDot : true,

    //Number - Radius of each point dot in pixels
    pointDotRadius : 4,

    //Number - Pixel width of point dot stroke
    pointDotStrokeWidth : 1,

    //Number - amount extra to add to the radius to cater for hit detection outside the drawn point
    pointHitDetectionRadius : 20,

    //Boolean - Whether to show a stroke for datasets
    datasetStroke : true,

    //Number - Pixel width of dataset stroke
    datasetStrokeWidth : 2,

    //Boolean - Whether to fill the dataset with a colour
    datasetFill : true,
    \{% raw %\}
    //String - A legend template
    legendTemplate : "<ul class=\(\backslash\)"<%=name.toLowerCase()%>-legend\(\backslash\)"><% for (var i=0; i<datasets.length;
       i++)\{%><li><span
       style=\(\backslash\)"background-color:<%=datasets[i].strokeColor%>\(\backslash\)"></span><%if(datasets[i].label)\{%><%=datasets[i].label%><%\}%></li><%\}%></ul>"
    \{% endraw %\}
\};
\end{DoxyCode}


You can override these for your {\ttfamily Chart} instance by passing a second argument into the {\ttfamily Line} method as an object with the keys you want to override.

For example, we could have a line chart without bezier curves between points by doing the following\+:


\begin{DoxyCode}
new Chart(ctx).Line(data, \{
    bezierCurve: false
\});
// This will create a chart with all of the default options, merged from the global config,
// and the Line chart defaults, but this particular instance will have `bezierCurve` set to false.
\end{DoxyCode}


We can also change these defaults values for each Line type that is created, this object is available at {\ttfamily Chart.\+defaults.\+Line}.

\subsubsection*{Prototype methods}

\paragraph*{.get\+Points\+At\+Event( event )}

Calling {\ttfamily get\+Points\+At\+Event(event)} on your Chart instance passing an argument of an event, or j\+Query event, will return the point elements that are at that the same position of that event.


\begin{DoxyCode}
canvas.onclick = function(evt)\{
    var activePoints = myLineChart.getPointsAtEvent(evt);
    // => activePoints is an array of points on the canvas that are at the same position as the click
       event.
\};
\end{DoxyCode}


This functionality may be useful for implementing D\+OM based tooltips, or triggering custom behaviour in your application.

\paragraph*{.update( )}

Calling {\ttfamily update()} on your Chart instance will re-\/render the chart with any updated values, allowing you to edit the value of multiple existing points, then render those in one animated render loop.


\begin{DoxyCode}
myLineChart.datasets[0].points[2].value = 50;
// Would update the first dataset's value of 'March' to be 50
myLineChart.update();
// Calling update now animates the position of March from 90 to 50.
\end{DoxyCode}


\paragraph*{.add\+Data( values\+Array, label )}

Calling {\ttfamily add\+Data(values\+Array, label)} on your Chart instance passing an array of values for each dataset, along with a label for those points.


\begin{DoxyCode}
// The values array passed into addData should be one for each dataset in the chart
myLineChart.addData([40, 60], "August");
// This new data will now animate at the end of the chart.
\end{DoxyCode}


\paragraph*{.remove\+Data( )}

Calling {\ttfamily remove\+Data()} on your Chart instance will remove the first value for all datasets on the chart.


\begin{DoxyCode}
myLineChart.removeData();
// The chart will remove the first point and animate other points into place
\end{DoxyCode}
 